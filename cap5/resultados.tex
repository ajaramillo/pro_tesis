\Pag{Resultados}
\sectionm{Resultados}
\subsection{Plataforma Ejecuci\'on}
\quad Describiremos a continuaci\'on, las principales caracter\'isticas del equipo utilizado para la ejecuci\'on del programa.

\begin{description}
 \item[Nombre Servidor y ubicaci\'on : ] Kudi. C.E.M.C.C. Universidad de La Frontera. Temuco.
\item[Procesador : ] Intel XEON CPU, E5506. Velocidad reloj 2.13GHZ.
\item[N\textdegree{} N\'ucleos :] 4 n\'ucleos, 8 Hilos de procesamiento.
\item[Memoria RAM : ] 16GB.
\item[Cach\'e L1 : ] 32KB por N\'ucleo.
\item[Cach\'e L2 : ] 256KB por N\'ucleo.
\item[Cach\'e L3 : ] 4MB compartida.
\item[S.O. : ] GNU/Linux, distribuci\'on Fedora Core 13, Kernel 2.6.34.7-66.fc13.x86\_64.
\end{description}

\subsection{Resultados}

\subsubsection{Resultados anteriores}
En las figuras \ref{fig:niv_at_012}, \ref{fig:niv_at_34} y \ref{fig:modos_ab} se presentan la evoluci\'on del sistema obtenida en (\cite{gino} para el caso $\dimA=3$, $\dimB=3$ sin ruido t\'ermico.

%%%%%%%%%%%%%%%%%%%%%%%%%%%%%%%%%%%%%%%% RESULTADOS GINO %%%%%%%%%%%%%%%%%%%%%%%%%%%%%%%%%%%%%%%%%
\begin{figure}[!ht]
\centering
 \includegraphics[scale=0.75]{cap5/imagen_gino_p123.eps}\caption{Evoluci\'on de niveles at\'omicos 0, 1 y 2 en trabajo anterior.}\label{fig:niv_at_012}
 \includegraphics[scale=0.75]{cap5/imagen_gino_p45.eps}\caption{Evoluci\'on de niveles at\'omicos 3 y 4 en trabajo anterior.}\label{fig:niv_at_34}
\end{figure}
\begin{figure}[!ht]
\centering
 \includegraphics[scale=0.75]{cap5/imagen_gino_pAB.eps}\caption{Evoluci\'on poblacional modos $a$ y $b$ en trabajo anterior.}\label{fig:modos_ab}
\end{figure}

\subsubsection{Resultados niveles at\'omicos y poblaciones de los modos}
\quad Ahora, procedemos a presentar los resultados, en forma gr\'afica, de las curvas resultantes de la evoluci\'on del sistema para diferentes niveles de ruido, esto es, alrededor de $\bar{n}=0\ldotp 01$ que corresponde al ruido t\'ermico encontrado en la literatura, para temperaturas del orden $[\mu K]$.
\clearpage
%%%%%%%%%%%%%%%%%%%%%%%%%%%%%%%%%%%%%%%% RESULTADOS TRABAJO Niv 0, 1 y 2 %%%%%%%%%%%%%%%%%%%%%%%%%%%%%%%%%%%%%%%%%
\begin{figure}[ht]
\centering
\includegraphics[scale=0.75]{cap5/imagen_4_0.001_p123.eps}\caption{Evoluci\'on niveles at\'omicos 0, 1 y 2, $\bar{n}=0\ldotp 001$.}\label{fig:niveles_123_4_0001}
\includegraphics[scale=0.75]{cap5/imagen_5_0.005_p123.eps}\caption{Evoluci\'on niveles at\'omicos 0, 1 y 2, $\bar{n}=0\ldotp 005$.}\label{fig:niveles_123_5_0005}
\end{figure}
\begin{figure}[ht]
\centering
\includegraphics[scale=0.75]{cap5/imagen_6_0.01_p123.eps}\caption{Evoluci\'on niveles at\'omicos 0, 1 y 2, $\bar{n}=0\ldotp 01$.}\label{fig:niveles_123_6_001}
\includegraphics[scale=0.75]{cap5/imagen_7_0.05_p123.eps}\caption{Evoluci\'on niveles at\'omicos 0, 1 y 2, $\bar{n}=0\ldotp 05$.}\label{fig:niveles_123_7_005}
\end{figure}
\begin{figure}[ht]
\centering
\includegraphics[scale=0.75]{cap5/imagen_10_0.1_p123.eps}\caption{Evoluci\'on niveles at\'omicos 0, 1 y 2, $\bar{n}=0\ldotp 1$.}\label{fig:niveles_123_10_01}
\includegraphics[scale=0.75]{cap5/imagen_13_0.5_p123.eps}\caption{Evoluci\'on niveles at\'omicos 0, 1 y 2, $\bar{n}=0\ldotp 5$.}\label{fig:niveles_123_13_05}
\end{figure}
%%%%%%%%%%%%%%%%%%%%%%%%%%%%%%%%%%%%%%%% RESULTADOS TRABAJO Niv 0, 1 y 2 %%%%%%%%%%%%%%%%%%%%%%%%%%%%%%%%%%%%%%%%%


%%%%%%%%%%%%%%%%%%%%%%%%%%%%%%%%%%%%%%%% RESULTADOS TRABAJO Niv 3 y 4 %%%%%%%%%%%%%%%%%%%%%%%%%%%%%%%%%%%%%%%%%
\begin{figure}[ht]
\centering
\includegraphics[scale=0.75]{cap5/imagen_4_0.001_p45.eps}\caption{Evoluci\'on niveles at\'omicos 3 y 4, $\bar{n}=0\ldotp 001$.}\label{fig:niveles_45_4_0001}
\end{figure}
\begin{figure}[ht]
\centering
\includegraphics[scale=0.75]{cap5/imagen_5_0.005_p45.eps}\caption{Evoluci\'on niveles at\'omicos 3 y 4, $\bar{n}=0\ldotp 005$.}\label{fig:niveles_45_5_0005}
\end{figure}
\begin{figure}[ht]
\centering
\includegraphics[scale=0.75]{cap5/imagen_6_0.01_p45.eps}\caption{Evoluci\'on niveles at\'omicos 3 y 4, $\bar{n}=0\ldotp 01$.}\label{fig:niveles_45_6_001}
\includegraphics[scale=0.75]{cap5/imagen_7_0.05_p45.eps}\caption{Evoluci\'on niveles at\'omicos 3 y 4, $\bar{n}=0\ldotp 05$.}\label{fig:niveles_45_7_005}
\end{figure}
\begin{figure}[ht]
\centering
\includegraphics[scale=0.75]{cap5/imagen_10_0.1_p45.eps}\caption{Evoluci\'on niveles at\'omicos 3 y 4, $\bar{n}=0\ldotp 1$.}\label{fig:niveles_45_10_01}
\includegraphics[scale=0.75]{cap5/imagen_13_0.5_p45.eps}\caption{Evoluci\'on niveles at\'omicos 3 y 4, $\bar{n}=0\ldotp 5$.}\label{fig:niveles_45_13_05}
\end{figure}
%%%%%%%%%%%%%%%%%%%%%%%%%%%%%%%%%%%%%%%% RESULTADOS TRABAJO Niv 3 y 4 %%%%%%%%%%%%%%%%%%%%%%%%%%%%%%%%%%%%%%%%%


%%%%%%%%%%%%%%%%%%%%%%%%%%%%%%%%%%%%%%%% RESULTADOS TRABAJO <nA> <nB> %%%%%%%%%%%%%%%%%%%%%%%%%%%%%%%%%%%%%%%%%
\begin{figure}[ht]
\centering
 \includegraphics[scale=0.75]{cap5/imagen_4_0.001_pAB.eps}\caption{Evoluci\'on poblacional modos $a$ y $b$, $\bar{n}=0\ldotp 001$.}\label{fig:modos_ab_4_0001}
 \includegraphics[scale=0.75]{cap5/imagen_5_0.005_pAB.eps}\caption{Evoluci\'on poblacional modos $a$ y $b$, $\bar{n}=0\ldotp 005$.}\label{fig:modos_ab_5_0005}
\end{figure}
\begin{figure}[ht]
\centering
 \includegraphics[scale=0.75]{cap5/imagen_6_0.01_pAB.eps}\caption{Evoluci\'on poblacional modos $a$ y $b$, $\bar{n}=0\ldotp 01$.}\label{fig:modos_ab_6_001}
 \includegraphics[scale=0.75]{cap5/imagen_7_0.05_pAB.eps}\caption{Evoluci\'on poblacional modos $a$ y $b$, $\bar{n}=0\ldotp 05$.}\label{fig:modos_ab_7_005}
\end{figure}
\begin{figure}[ht]
\centering
 \includegraphics[scale=0.75]{cap5/imagen_10_0.1_pAB.eps}\caption{Evoluci\'on poblacional modos $a$ y $b$, $\bar{n}=0\ldotp 1$.}\label{fig:modos_ab_10_01}
 \includegraphics[scale=0.75]{cap5/imagen_13_0.5_pAB.eps}\caption{Evoluci\'on poblacional modos $a$ y $b$, $\bar{n}=0\ldotp 5$.}\label{fig:modos_ab_13_05}
\end{figure}
%%%%%%%%%%%%%%%%%%%%%%%%%%%%%%%%%%%%%%%% RESULTADOS TRABAJO <nA> <nB> %%%%%%%%%%%%%%%%%%%%%%%%%%%%%%%%%%%%%%%%%
\clearpage
\subsubsection{Espectros obtenidos}
\qquad Ahora presentamos los espectros (normalizados) para las curvas de poblaci\'on de ambos modos, $a$ y $b$ del campo. Se aprecia c\'omo a medida que el ruido aumenta es m\'as dif\'icil poder discriminar si ocurri\'o la emisi\'on de un fot\'on o s\'olo hay ruido en la cavidad, para esto comp\'arese la respectiva curva con la presentada en la Figura \ref{fig:espectro_ruidopuro}, que corresponde al espectro de un ruido puro. En las figuras, $DFT$ denota que la curva presentada corresponde a la Transformada Discreta de Fourier (Discrete Fourier Transform).
%%%%%%%%%%%%%%%%%%%%%%%%%%%%%%%%%%%%%%% RESULTADOS TRABAJO espectros nbar=0 (gino) %%%%%%%%%%%%%%%%%%%%%%%%
\begin{figure}[ht]
\hspace*{-1.6cm}
\begin{minipage}{0.52 \linewidth}
\centering
 \includegraphics[scale=0.6]{cap5/imagen_2_0.0_fft_nA.eps}
\caption{Espectro modo $a$, trabajo anterior sin riudo t\'ermico.}\label{fig:espectro_2_0}
\end{minipage}
\hspace*{1.5cm}
\begin{minipage}{0.52 \linewidth}
\centering
 \includegraphics[scale=0.6]{cap5/imagen_2_0.0_fft_nB.eps}
\caption{Espectro modo $b$, trabajo anterior sin riudo t\'ermico.}\label{fig:espectro_2_0}
\end{minipage}
\end{figure}
%%%%%%%%%%%%%%%%%%%%%%%%%%%%%%%%%%%%%%%% RESULTADOS TRABAJO espectros nbar=0 (gino) %%%%%%%%%%%%%%%%%%%%%%%%
\begin{figure}[ht]
\centering
\begin{minipage}{0.52 \linewidth}
\centering
 \includegraphics[scale=0.6]{cap5/imagen_espectro_ruidopuro.eps}
\caption{Espectro ruido puro normalizado.}\label{fig:espectro_ruidopuro}
\end{minipage}
\end{figure}
%%%%%%%%%%%%%%%%%%%%%%%%%%%%%%%%%%%%%%%% RESULTADOS TRABAJO espectro nbar=0.001 modos a y b%%%%%%%%%%%%%%%%%%%%%%%%
\begin{figure}[ht]
\hspace*{-1.6cm}
\begin{minipage}{0.52 \linewidth}
\centering
 \includegraphics[scale=0.6]{cap5/imagen_4_0.001_fft_nA.eps}
\caption{Espectro modo $a$, $\bar{n}=0\ldotp 001$.}\label{fig:espectro_4_0001}
\end{minipage}
\hspace*{1.5cm}
\begin{minipage}{0.52 \linewidth}
\centering
 \includegraphics[scale=0.6]{cap5/imagen_4_0.001_fft_nB.eps}
\caption{Espectro modo $b$, $\bar{n}=0\ldotp 001$.}\label{fig:espectro_4_0001}
\end{minipage}
\end{figure}
%%%%%%%%%%%%%%%%%%%%%%%%%%%%%%%%%%%%%%%% RESULTADOS TRABAJO espectro nbar=0.001 modos a y b%%%%%%%%%%%%%%%%%%%%%%%%
%%%%%%%%%%%%%%%%%%%%%%%%%%%%%%%%%%%%%%%% RESULTADOS TRABAJO espectro nbar=0.005 modos a y b%%%%%%%%%%%%%%%%%%%%%%%%
\begin{figure}[ht]
\hspace*{-1.6cm}
\begin{minipage}{0.52 \linewidth}
\centering
 \includegraphics[scale=0.6]{cap5/imagen_5_0.005_fft_nA.eps}
\caption{Espectro modo $a$, $\bar{n}=0\ldotp 005$.}\label{fig:espectro_5_0005}
\end{minipage}
\hspace*{1.5cm}
\begin{minipage}{0.52 \linewidth}
\centering
 \includegraphics[scale=0.6]{cap5/imagen_5_0.005_fft_nB.eps}
\caption{Espectro modo $b$, $\bar{n}=0\ldotp 005$.}\label{fig:espectro_5_0005}
\end{minipage}
\end{figure}
%%%%%%%%%%%%%%%%%%%%%%%%%%%%%%%%%%%%%%%% RESULTADOS TRABAJO espectro nbar=0.005 modos a y b%%%%%%%%%%%%%%%%%%%%%%%%
%%%%%%%%%%%%%%%%%%%%%%%%%%%%%%%%%%%%%%%% RESULTADOS TRABAJO espectro nbar=0.01 modos a y b%%%%%%%%%%%%%%%%%%%%%%%%
\begin{figure}[ht]
\hspace*{-1.6cm}
\begin{minipage}{0.52 \linewidth}
\centering
 \includegraphics[scale=0.6]{cap5/imagen_6_0.01_fft_nA.eps}
\caption{Espectro modo $a$, $\bar{n}=0\ldotp 01$.}\label{fig:espectro_6_001}
\end{minipage}
\hspace*{1.5cm}
\begin{minipage}{0.52 \linewidth}
\centering
 \includegraphics[scale=0.6]{cap5/imagen_6_0.01_fft_nB.eps}
\caption{Espectro modo $b$, $\bar{n}=0\ldotp 01$.}\label{fig:espectro_6_001}
\end{minipage}
\end{figure}
% %%%%%%%%%%%%%%%%%%%%%%%%%%%%%%%%%%%%%%%% RESULTADOS TRABAJO espectro nbar=0.01 modos a y b%%%%%%%%%%%%%%%%%%%%%%%%
%%%%%%%%%%%%%%%%%%%%%%%%%%%%%%%%%%%%%%%% RESULTADOS TRABAJO espectro nbar=0.05 modos a y b%%%%%%%%%%%%%%%%%%%%%%%%
\begin{figure}[ht]
\hspace*{-1.6cm}
\begin{minipage}{0.52 \linewidth}
\centering
 \includegraphics[scale=0.6]{cap5/imagen_7_0.05_fft_nA.eps}
\caption{Espectro modo $a$, $\bar{n}=0\ldotp 05$.}\label{fig:espectro_7_005}
\end{minipage}
\hspace*{1.5cm}
\begin{minipage}{0.52 \linewidth}
\centering
 \includegraphics[scale=0.6]{cap5/imagen_7_0.05_fft_nB.eps}
\caption{Espectro modo $b$, $\bar{n}=0\ldotp 05$.}\label{fig:espectro_7_005}
\end{minipage}
\end{figure}
%%%%%%%%%%%%%%%%%%%%%%%%%%%%%%%%%%%%%%%% RESULTADOS TRABAJO espectro nbar=0.05 modos a y b%%%%%%%%%%%%%%%%%%%%%%%%
%%%%%%%%%%%%%%%%%%%%%%%%%%%%%%%%%%%%%%%% RESULTADOS TRABAJO espectro nbar=0.1 modos a y b%%%%%%%%%%%%%%%%%%%%%%%%
\begin{figure}[ht]
\hspace*{-1.6cm}
\begin{minipage}{0.52 \linewidth}
\centering
 \includegraphics[scale=0.6]{cap5/imagen_10_0.1_fft_nA.eps}
\caption{Espectro modo $a$, $\bar{n}=0\ldotp 1$.}\label{fig:espectro_10_01}
\end{minipage}
\hspace*{1.5cm}
\begin{minipage}{0.52 \linewidth}
\centering
 \includegraphics[scale=0.6]{cap5/imagen_10_0.1_fft_nB.eps}
\caption{Espectro modo $b$, $\bar{n}=0\ldotp 1$.}\label{fig:espectro_10_01}
\end{minipage}
\end{figure}
%%%%%%%%%%%%%%%%%%%%%%%%%%%%%%%%%%%%%%% RESULTADOS TRABAJO espectro nbar=0.1 modos a y b%%%%%%%%%%%%%%%%%%%%%%%%
% %%%%%%%%%%%%%%%%%%%%%%%%%%%%%%%%%%%%%%% RESULTADOS TRABAJO espectro nbar=0.5 modos a y b%%%%%%%%%%%%%%%%%%%%%%%%
\clearpage
\begin{figure}[h]
\hspace*{-1.6cm}
\begin{minipage}{0.52 \linewidth}
\centering
 \includegraphics[scale=0.6]{cap5/imagen_13_0.5_fft_nA.eps}
\caption{Espectro modo $a$, $\bar{n}=0\ldotp 5$.}\label{fig:espectro_13_05}
\end{minipage}
\hspace*{1.5cm}
\begin{minipage}{0.52 \linewidth}
\centering
 \includegraphics[scale=0.6]{cap5/imagen_13_0.5_fft_nB.eps}
\caption{Espectro modo $b$, $\bar{n}=0\ldotp 5$.}\label{fig:espectro_13_05}
\end{minipage}
\end{figure}
%%%%%%%%%%%%%%%%%%%%%%%%%%%%%%%%%%%%%%% RESULTADOS TRABAJO espectro nbar=0.5 modos a y b%%%%%%%%%%%%%%%%%%%%%%%%

\subsubsection{Tiempos de ejecuci\'on} \qquad En la Tabla \ref{tabla:tiempos_modos} presentamos los tiempos\footnote{En esta secci\'on, los intervalos de tiempo ser\'an denotados seg\'un ``DDdHHhMMmSSs``, donde DD es la cantidad de d\'ias, HH horas, MM minutos y SS segundos.} de ejecuci\'on resultantes para diferentes truncamientos de los espacios asociados a los modos del campo, tomando siempre ($\dimA=\dimB=2\ldots 10)$.\\
% \begin{table}[h]
% \centering
% \begin{tabular}{|c|c|c|c|c|c|c|c|c|c|}
% \hline &\multicolumn{9}{|c|}{$\dimA=\dimB$}\\
% \hline $\bar{n}$& 2&3&4&5&6&7&8&9&10\\
% \hline $0\ldotp001$& $18$ &$69$ & $229$ & $562$ & $1135$ & $2532$ & $5792$ & $19090$ & $28813$\\
% \hline $0\ldotp 005$& $18$ &$67$ & $236$ & $594$ & $1328$ & $2260$ & $10178$ & $10296$ & $27449$\\
% \hline $0\ldotp 01$& $20$ &$75$ & $228$ & $581$ & $1148$ & $1958$ & $7800$ & $14641$ & $19991$\\
% \hline $0\ldotp 05$& $22$ &$79$ & $261$ & $593$ & $1272$ & $2556$ & $17517$ & $20969$ & $32860$\\
% \hline
% \end{tabular}\caption{Tiempo (seg) de ejecuci\'on para diferentes dimensiones de los espacios modales.}\label{tabla:tiempos_modos}
% \end{table}
\quad En las siguientes tables, denotaremos por $R_f$ el Algoritmo \ref{algo:main_int_f} asociado al c\'alculo de la funci\'on objetivo a integrar; $R_e$ denotar\'a al Algoritmo \ref{algo:main_int_err}, asociado al c\'alculo del error relativo y $R_c$ que denotar\'a al Algoritmo \ref{algo:main_int_calculos} asociado al c\'alculo de las cantidades f\'isicas a medir. En la Tabla \ref{tabla:tiempos_gino}, mostramos los tiempos de ejecuci\'on, del Programa original en FORTRAN. Este programa corresponde a la \emph{traducci\'on literal} del programa utilizado en \cite{gino} y escrito en MATLAB, para la resoluci\'on del problema del presente trabajo, con $\dimA=\dimB=2$ y sin ruido t\'ermico. Adem\'as se muestran tambi\'en las proporciones (en porcentaje) de los tiempos de demora de cada una de las rutinas principales definidas anteriormente. Tambi\'en, en la Tabla \ref{tabla:cantidad_memoria_rho} se muestra la cantidad de memoria utilizada para almacenar $\rho$, el color azul, verde y rojo, indican que para el tama\~no de los espacios modales asociado, la matriz $\rho$, el programa entero, se pueden almacenar totalmente en la cach\'e L1, L2 o L3 respectivamente.\\

\quad Adem\'as, en la Tabla \ref{tabla:numero_correcciones}, fijando $\dimA=\dimB=7$ y haciendo variar $\bar{n}$, se observa c\'omo var\'ia el n\'umero de iteraciones que realiza el m\'etodo corrector en la integraci\'on.

\begin{table}[h]
\centering
\begin{tabular}{|c|c|c|c|c|c|c|c|c|c|c|c|c|c|}
\hline $\dimA=\dimB$ & 2 &3 & 4& 5& 6&7 & 8&9 &10&11&12\\
\hline$\rho$ &\colorbox{green}{16}&\colorbox{blue}{50}&\colorbox{blue}{122}&\colorbox{blue}{253}&\colorbox{red}{469}&\colorbox{red}{800}&    \colorbox{red}{1281}&\colorbox{red}{1953}&\colorbox{red}{2860}&\colorbox{red}{4050}&    5578\\
\hline Total &\colorbox{red}{1012}&    \colorbox{red}{1276}&    \colorbox{red}{1796}&    \colorbox{red}{2596}&    \colorbox{red}{3868}&    5784&    8640&    12288&    17408&    24576&    33792\\
\hline
\end{tabular} \caption{Memoria utilizada por la matriz $\rho$ en y el programa completo en KB. }\label{tabla:cantidad_memoria_rho}
\end{table}

\begin{table}[h]
\centering
\begin{tabular}{|c|c|c|c|c|c|c|c|c|c|c|}
\hline &\multicolumn{9}{|c|}{$\dimA=\dimB$}&\\
\hline $\bar{n}$& 2&3&4&5&6&7&8&9&10&Total\\
\hline $0\ldotp001$& 17s &1m02s & \colorbox{red}{3m01s} & 6m54s & 15m40s & 35m53s& 1h31m & 2h03m & 4h16m &8h53m\\
\hline $0\ldotp 005$&18s &1m03s & 3m04s & \colorbox{red}{6m56s} & 15m26s & 36m16s & 1h38m & 2h51m & 4h34m & 10h06m\\
\hline $0\ldotp 01$& 19s &1m10s & 3m21s & 7m45s & \colorbox{red}{15m31s} & 38m00s & 1h35m & 2h54m & 5h02m&10h37m\\
\hline $0\ldotp 05$& 20s &1m11s & 3m25s & 7m41s & 15m36s & \colorbox{red}{45m40s} & 1h59m & 3h09m & 4h44m&11h06m\\
 \hline $0\ldotp 1$& 20s &1m14s & 3m32s & 8m02s & 16m58s & 46m09s & 1h41m & 3h58m & \colorbox{red}{5h55m}&12h51m\\
 \hline $0\ldotp 5$& 22s &1m20s & 3m51s & 8m55s & 18m42s & 50m27s & 2h10m & 4h06m & 6h16m&13h55m\\
\hline
\end{tabular}\caption{Tiempos de ejecuci\'on para diferentes dimensiones de los espacios modales.}\label{tabla:tiempos_modos}
\end{table}

\begin{table}[h]
 \centering
\begin{tabular}{|c|c|c|c|c|c|c|c|c|}
\hline $\dimA=\dimB$ & 2 & 3 & 4& 5& 6&7 & 8&9 \\
\hline Tiempo & 24m & 2h18m & 8h & 1d & 3d9h&14d19h & 21d & 38d \\
\hline $R_f$ &$90\ldotp 6\%$ & $92\ldotp 4\%$ &$91\ldotp 1\%$ &$91\ldotp 3\%$ &$94\ldotp 0\%$ &$93\ldotp 8\%$ &$94\ldotp 2\%$ & $88\ldotp 9\%$ \\
\hline $R_e$ &$3\ldotp 9\%$ &$3\ldotp 7\%$ &$3\ldotp 7\%$ &$3\ldotp 8\%$ &$2\ldotp 2\%$ &$2\ldotp 6\%$ &$2\ldotp 1\%$ & $3\ldotp 6\%$\\
\hline $R_c$ &$4\ldotp 7\%$ & $4\ldotp 6\%$&$4\ldotp 6\%$ &$4\ldotp 8\%$ &$3\ldotp 4\%$ &$3\ldotp 6\%$ &$3\ldotp 2\%$ &$7\ldotp 3\%$ \\
\hline
\end{tabular}\caption{Tiempos de ejecuci\'on y porcentajes por rutina del trabajo anterior \cite{gino}.}\label{tabla:tiempos_gino}
\end{table}
\begin{table}[h]
 \centering
\begin{tabular}{|c|c|c|c|c|c|c|c|c|}
\hline $\dimA=\dimB$ & 2 & 3 & 4& 5& 6&7 & 8&9 \\
\hline Tiempo & 12s & 37s & 1m48s & 4m03s & 10m8s& 21m26s & 52m44s & 1h42m \\
\hline $R_f$ &$86\ldotp 6\%$ & $89\ldotp 5\%$ &$90\ldotp 2\%$ &$89\ldotp 8\%$ &$85\ldotp 5\%$ &$84\ldotp 8\%$ &$85\ldotp 8\%$ & $87\ldotp 7\%$ \\
\hline $R_e$ &$4\ldotp 0\%$ &$3\ldotp 0\%$ &$3\ldotp 2\%$ &$3\ldotp 2\%$ &$4\ldotp 8\%$ &$4\ldotp 8\%$ &$3\ldotp 8\%$ & $2\ldotp 8\%$\\
\hline $R_c$ &$3\ldotp 8\%$ & $3\ldotp 0\%$&$2\ldotp 8\%$ &$2\ldotp 8\%$ &$3\ldotp 0\%$ &$3\ldotp 3\%$ &$4\ldotp 6\%$ &$5\ldotp 0\%$ \\
\hline
\end{tabular}\caption{Tiempos de ejecuci\'on y porcentajes por rutina del trabajo actual sin ruido t\'ermico.}\label{tabla:tiempos_actual_sinruido}
\end{table}

\begin{table}[h]
 \centering
\begin{tabular}{|c|c|c|c|c|c|c|c|}
\hline &\multicolumn{7}{|c|}{$\bar{n}$}\\
\hline $n$ & $0\ldotp0$ &$0\ldotp001$ &$0\ldotp005$ &$0\ldotp01$ &$0\ldotp05$ &$0\ldotp1$ &$0\ldotp5$\\
\hline 0 & 3654 & 0 & 0 & 0 & 0 &0 &0 \\
\hline 1 & 4513 & 0 & 0 & 0 & 0 &0 &0 \\
\hline 2 & 3853 & 11896 & 10974 & 0 & 0 &0 &0 \\
\hline 3 & 3627 & 3743 & 4630 & 15560 & 14910 &10906 &1492\\
\hline 4 & 4353 & 4361 & 4396 & 4440 & 5090 &9094 &17573 \\
\hline 5 & 0 & 0 & 0 & 0 & 0 &0 &935 \\
\hline
\end{tabular}\caption{N\'umero de correcciones $n$, en el m\'etodo de integraci\'on, $\dimA=\dimB=7$.}\label{tabla:numero_correcciones}
\end{table}
\clearpage
\subsubsection{Proporciones de demora los Algoritmos}\label{sec:prop_demora_algoritmos}

\qquad Ahora, otros datos de inter\'es son las proporciones, sobre el tiempo total, de demora de cada subrutina importante identificada en (\ref{sec:codigos_generales}). Resumimos \'estas proporciones de tiempo de ejecuci\'on en la Tabla \ref{tabla:tiempos_prop_por_rutina}.\\

\begin{table}[h]
\centering
\begin{tabular}{|c|c|c|c|c|c|c|c|c|c|c|}
\hline \multicolumn{11}{|c|}{$\bar{n}=0\ldotp 001$}\\
\hline $\dimA=\dimB$ & 2&3&4&5&6&7&8&9&10&Prom.\\
\hline $R_f$ & $ 90\ldotp 2$ & $ 92\ldotp 7$ & $ 93\ldotp 6$ & $ 93\ldotp 8$ & $ 91\ldotp 2$ & $ 89\ldotp 1$ & $ 89\ldotp 6$ & $ 90\ldotp 7$ & $ 90\ldotp 2$ & $ 91\ldotp 2$\\
\hline $R_e$ & $  3\ldotp 0$ & $  2\ldotp 6$ & $  2\ldotp 2$ & $  2\ldotp 1$ & $  3\ldotp 0$ & $  3\ldotp 4$ & $  2\ldotp 9$ & $  2\ldotp 2$ & $  2\ldotp 3$ & $  2\ldotp 6$\\
\hline $R_c$ & $  2\ldotp 6$ & $  1\ldotp 8$ & $  1\ldotp 7$ & $  1\ldotp 6$ & $  1\ldotp 7$ & $  2\ldotp 5$ & $  2\ldotp 9$ & $  3\ldotp 7$ & $  3\ldotp 7$ & $  2\ldotp 5$\\
\hline
\hline \multicolumn{11}{|c|}{$\bar{n}=0\ldotp 005$}\\
\hline $\dimA=\dimB$ & 2&3&4&5&6&7&8&9&10&Prom.\\
\hline $R_f$ & $ 90\ldotp 5$ & $ 92\ldotp 7$ & $ 93\ldotp 5$ & $ 93\ldotp 8$ & $ 92\ldotp 3$ & $ 89\ldotp 3$ & $ 90\ldotp 0$ & $ 89\ldotp 6$ & $ 90\ldotp 3$ & $ 91\ldotp 3$\\
\hline $R_e$ & $  2\ldotp 7$ & $  2\ldotp 5$ & $  2\ldotp 2$ & $  2\ldotp 1$ & $  2\ldotp 6$ & $  3\ldotp 4$ & $  2\ldotp 8$ & $  2\ldotp 6$ & $  2\ldotp 2$ & $  2\ldotp 6$\\
\hline $R_c$ & $  2\ldotp 5$ & $  1\ldotp 7$ & $  1\ldotp 7$ & $  1\ldotp 5$ & $  1\ldotp 7$ & $  2\ldotp 4$ & $  2\ldotp 7$ & $  3\ldotp 6$ & $  4\ldotp 1$ & $  2\ldotp 4$\\
\hline
\hline \multicolumn{11}{|c|}{$\bar{n}=0\ldotp 01$}\\
\hline $\dimA=\dimB$ & 2&3&4&5&6&7&8&9&10&Prom.\\
\hline $R_f$ & $ 90\ldotp 5$ & $ 92\ldotp 8$ & $ 93\ldotp 7$ & $ 93\ldotp 9$ & $ 94\ldotp 3$ & $ 90\ldotp 0$ & $ 89\ldotp 2$ & $ 90\ldotp 7$ & $ 90\ldotp 3$ & $ 91\ldotp 7$\\
\hline $R_e$ & $  3\ldotp 0$ & $  2\ldotp 6$ & $  2\ldotp 3$ & $  2\ldotp 1$ & $  2\ldotp 0$ & $  3\ldotp 4$ & $  3\ldotp 2$ & $  2\ldotp 5$ & $  2\ldotp 6$ & $  2\ldotp 6$\\
\hline $R_c$ & $  2\ldotp 2$ & $  1\ldotp 6$ & $  1\ldotp 5$ & $  1\ldotp 4$ & $  1\ldotp 5$ & $  2\ldotp 1$ & $  2\ldotp 6$ & $  3\ldotp 1$ & $  3\ldotp 3$ & $  2\ldotp 1$\\
\hline
\hline \multicolumn{11}{|c|}{$\bar{n}=0\ldotp 05$}\\
\hline $\dimA=\dimB$ & 2&3&4&5&6&7&8&9&10&Prom.\\
\hline $R_f$ & $ 90\ldotp 5$ & $ 90\ldotp 8$ & $ 93\ldotp 7$ & $ 94\ldotp 2$ & $ 94\ldotp 3$ & $ 87\ldotp 9$ & $ 89\ldotp 0$ & $ 89\ldotp 8$ & $ 87\ldotp 8$ & $ 90\ldotp 9$\\
\hline $R_e$ & $  3\ldotp 0$ & $  3\ldotp 8$ & $  2\ldotp 3$ & $  2\ldotp 1$ & $  2\ldotp 0$ & $  4\ldotp 2$ & $  3\ldotp 4$ & $  2\ldotp 8$ & $  2\ldotp 3$ & $  2\ldotp 9$\\
\hline $R_c$ & $  1\ldotp 9$ & $  1\ldotp 7$ & $  1\ldotp 4$ & $  1\ldotp 4$ & $  1\ldotp 5$ & $  2\ldotp 1$ & $  2\ldotp 4$ & $  3\ldotp 2$ & $  3\ldotp 2$ & $  2\ldotp 1$\\
\hline
\hline \multicolumn{11}{|c|}{$\bar{n}=0\ldotp 1$}\\
\hline $\dimA=\dimB$ & 2&3&4&5&6&7&8&9&10&Prom.\\
\hline $R_f$ & $ 90\ldotp 5$ & $ 93\ldotp 1$ & $ 93\ldotp 7$ & $ 94\ldotp 0$ & $ 92\ldotp 5$ & $ 89\ldotp 3$ & $ 90\ldotp 7$ & $ 88\ldotp 8$ & $ 87\ldotp 4$ & $ 91\ldotp 1$\\
\hline $R_e$ & $  3\ldotp 0$ & $  2\ldotp 5$ & $  2\ldotp 3$ & $  2\ldotp 2$ & $  2\ldotp 6$ & $  3\ldotp 6$ & $  2\ldotp 9$ & $  3\ldotp 2$ & $  2\ldotp 7$ & $  2\ldotp 8$\\
\hline $R_c$ & $  2\ldotp 1$ & $  1\ldotp 6$ & $  1\ldotp 4$ & $  1\ldotp 3$ & $  1\ldotp 4$ & $  2\ldotp 0$ & $  2\ldotp 4$ & $  2\ldotp 9$ & $  3\ldotp 2$ & $  2\ldotp 0$\\
\hline
\hline \multicolumn{11}{|c|}{$\bar{n}=0\ldotp 5$}\\
\hline $\dimA=\dimB$ & 2&3&4&5&6&7&8&9&10&Prom.\\
\hline $R_f$ & $ 90\ldotp 8$ & $ 92\ldotp 8$ & $ 93\ldotp 8$ & $ 94\ldotp 0$ & $ 92\ldotp 7$ & $ 89\ldotp 9$ & $ 89\ldotp 2$ & $ 88\ldotp 9$ & $ 88\ldotp 3$ & $ 91\ldotp 2$\\
\hline $R_e$ & $  3\ldotp 0$ & $  2\ldotp 9$ & $  2\ldotp 4$ & $  2\ldotp 2$ & $  2\ldotp 7$ & $  3\ldotp 4$ & $  3\ldotp 4$ & $  3\ldotp 3$ & $  2\ldotp 8$ & $  2\ldotp 9$\\
\hline $R_c$ & $  2\ldotp 0$ & $  1\ldotp 4$ & $  1\ldotp 3$ & $  1\ldotp 2$ & $  1\ldotp 3$ & $  2\ldotp 0$ & $  2\ldotp 4$ & $  2\ldotp 6$ & $  2\ldotp 8$ & $  1\ldotp 9$\\
\hline
\end{tabular}\caption{Porcentajes de tiempos de ejecuci\'on para diferentes dimensiones de los espacios modales.}\label{tabla:tiempos_prop_por_rutina}
\end{table}

\clearpage
\qquad En la Tabla \ref{tabla:proporciones_tiempo_F_lim_conmu}: '$R_f$' denota al Algoritmo \ref{algo:main_int_f}, mientras que 'Li', 'CH' denotan a los algoritmos \ref{algo:limbladiano2} y \ref{algo:conmuH} respectivamente, subrutinas de 'F'. 'Md', denota entre las l\'ineas \ref{algo:limb2_linea_inicial_loop_mod} y \ref{algo:limb2_linea_final_loop_mod} del Algoritmo \ref{algo:limbladiano2} involucra s\'olo operadores modales como aratMod; tambi\'en 'At' denota desde la l\'inea \ref{algo:limb2_linea_inicial_loop_At} y \ref{algo:limb2_linea_final_loop_At}, \'estas l\'ineas s\'olo involucran operadores at\'omicos, como aratAt. Seguidamente, 'AM' denota entre las l\'ineas \ref{algo:conmuH_linea_inicial_loop_AM} y \ref{algo:conmuH_linea_final_loop_AM} del Algoritmo \ref{algo:conmuH}, dichas l\'ineas operadores at\'omicos y de modos; 'AtC' denota entre las l\'ineas \ref{algo:conmuH_linea_inicial_loop_At} y \ref{algo:conmuH_linea_final_loop_At}, \'estas l\'ineas involucran \'unicamente operadores At\'omicos. \'Esta tabla contiene los porcentajes de demora de cada una de las secciones mencionadas, comparados sus tiempos versus la demora total del Algoritmo que las contiene.\\

\qquad De la Tabla \ref{tabla:proporciones_tiempo_F_lim_conmu}, podemos concluir, los Algoritmos que involucran operadores de los modos, demoran casi 8 veces en su c\'omputo que los Algoritmos que involucran operadores at\'omicos.

\begin{table}[h]
 \centering
\begin{tabular}{|c|c|c|c|c|c|c|c|}
\hline \multicolumn{2}{|c|}{} &\multicolumn{6}{|c|}{$\dimA=\dimB$}\\
\hline \'Ambito& Algor. & 2&4&6&8&10&12\\
\hline \multirow{2}{*}{$R_f$} 	& Li & 69\% & 75\% & 71\% & 59\% & 55\% & 54\%\\
\cline{2-8} 			& CH & 31\% & 25\% & 29\% & 41\% & 45\% & 46\%\\
\hline \multirow{2}{*}{Li} 	& Md & 93\% & 97\% & 97\% & 96\% & 96\% & 95\%\\
\cline{2-8} 			& At & 7\% & 3\% & 3\% & 4\% & 4\% & 5\%\\
\hline \multirow{2}{*}{CH} 	& AM & 87\% & 89\% & 91\% & 93\% & 93\% & 94\%\\
\cline{2-8} 			& AtC & 13\% & 11\% & 9\% & 7\% & 7\% & 6\%\\
\hline
\end{tabular}\caption{Proporciones de demora sobre el total (del resp. \'ambito)\\ de Algoritmos \ref{algo:main_int_f}, \ref{algo:limbladiano2} y \ref{algo:conmuH}.}\label{tabla:proporciones_tiempo_F_lim_conmu}
\end{table}

\subsubsection{Resultados paralelizaci\'on}
\qquad Para el c\'omputo en paralelo, se utilizaron los resultados expuestos en la secci\'on anterior. Se observa de aquellos resultados ha de ser conveniente calcular el Limbladiano (Li en \label{tabla:proporciones_tiempo_F_lim_conmu}) y el Conmutador del Hamiltoniano (CH en \label{tabla:proporciones_tiempo_F_lim_conmu}) en paralelo, \'esta paralelizaci\'on ser\'a denotada por P1. Adem\'as, se observa que las rutinas que m\'as costo computacional tienen, de las expuestas en (\ref{sec:computo_eficiente_operadores}), son las asociadas a operadores del campo electromagn\'etico (\ref{sec:expre_modales_orden2}), por ende, una segunda paralelizaci\'on es el c\'alculo en paralelo del bucl\'es, o \emph{loops} que aparecen en \'estos algoritmos (por ejemplo, Algoritmo \ref{algo:aratMod} l\'inea \ref{algo:aratMod_bucle0}, \'esta segunda paralelizaci\'on ser\'a denotada por P2. Finalmente, una tercera paralelizaci\'on, P3, estar\'a dada por la combinaci\'on de ambas paralelizaciones, mientras que la rutina original (pero con el n\'umero de correcciones fijadas a tres) ser\'a denotada por P0.\\

\qquad Para medir el rendimiento (\emph{performance}) de las paralelizaciones, expondremos los tiempos totales de demora de cada versi\'on (para diferentes dimensiones de los espacios modales), el factor de mejora $\alpha$ y la eficiencia lograda. Para \'esto \'ultimo; si $T_s$ es el tiempo de demora del programa \emph{serial}, esto es, el programa original; $T_p$ es el tiempo de demora \emph{real} del programa paralelizado, esto es, el tiempo que el usuario mide; $T_t$ es el tiempo total de demora, esto es, la suma de los tiempos en cada hilo de procesamiento. Entonces, $\alpha=T_s/T_p$ y la eficiencia $E_A$ de un algoritmo $A$ es calculada seg\'un:
$$E_A=100\cdot\frac{T_s}{T_t}\%.$$
\qquad Notemos que si $n$ es el n\'umero de hilos de procesamiento, entonces para obtener una eficiencia del 100$\%$ es necesario que cada hilo tenga un tiempo de procesamiento igual a $T_s/n$ y que ser\'a equivalente en dicho caso a $T_p$. En la Tabla \ref{tabla:rendimientos_paralelizaciones} se muestran tanto los factores de mejora de los tiempos, como las eficiencias logradas, mientras que en la Tabla \ref{tabla:tiempos_paralelizaciones} se muestran los tiempos respectivos. En todos \'estos casos, el n\'umero de correciones del m\'etodo de integraci\'on fue fijado a tres, \'esto para tener una mejor medida del producto de las paralelizaciones en s\'i, adem\'as el par\'ametro relacionado con el ruido t\'ermico fue fijado en $\bar{n}=0\ldotp 001$.\\

\begin{table}[h]
 \centering
\begin{tabular}{|c|c|c|c|c|c|c|c|c|c|c|c|c|}
\hline $\dimA=\dimB$&\multicolumn{2}{|c|}{2}&\multicolumn{2}{|c|}{4}&\multicolumn{2}{|c|}{6}&\multicolumn{2}{|c|}{8}&\multicolumn{2}{|c|}{10}&\multicolumn{2}{|c|}{12}\\
\hline Rutina&$\alpha$&$E_A\%$&$\alpha$&$E_A\%$&$\alpha$&$E_A\%$&$\alpha$&$E_A\%$&$\alpha$&$E_A\%$&$\alpha$&$E_A\%$\\
\hline P1 & 1.16 &59 &1.11 &57 &1.04 & 54&1.29 &68 &1.30 &{\color{red}69} &1.27 &{\color{red}69} \\
\hline P2 & {\color{red}0.98} &50 &1.25 &63 &1.25 & 64&1.21 &64 &1.2 &67 &1.23 &{\color{red}69} \\
\hline P3 & 1.32 &45 &1.68 &57 &1.63 & 57&{\color{red}1.85} &64 &1.84 &65 &1.73 &61 \\
\hline
\end{tabular}\caption{Rendimiento de las paralelizaciones realizadas.}\label{tabla:rendimientos_paralelizaciones}
\end{table}
\begin{table}[h]
 \centering
\begin{tabular}{|c|c|c|c|c|c|c|c|}
\hline \multicolumn{2}{|c|}{} &\multicolumn{6}{|c|}{$\dimA=\dimB$}\\
\hline Rutina& 2&4&6&8&10&12\\
\hline P0 & 24s& 4m03s& 18m49s& 1h19m& 3h59m& 8h13m\\
\hline P1 & 20s& 3m38s& 18m09s& 1h01m& 3h04m& 6h29m\\
\hline P2 & 24s& 3m14s& 15m03s& 1h05m& 3h16m& 6h41m\\
\hline P3 & 18s& 2m24s& 11m31s& 42m58s& 2h10m& 4h45m\\
\hline
\end{tabular}\caption{Tiempos de demora de rutinas paralelizadas.}\label{tabla:tiempos_paralelizaciones}
\end{table}