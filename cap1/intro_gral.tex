\Pag{Introducci�n}
\sectionm{Introducci�n}
%\section{Introducci\'on}
\subsection{Introducci\'on}
\subsection{Objetivos Generales del Trabajo de T\'itulo}
%\subsubsection{Objetivos espec\'ificos}
\quad A continuaci\'on se presentan los objetivos deseados de alcanzar en el presente Trabajo de T\'itulo.

\begin{enumerate}
 \item Implementar c\'odigos din\'amicos que resuelvan el problema a presentar en (\ref{sec:problema_resolver}). Por \emph{dinamismo} entenderemos que dichos c\'odigos sean reutilizables en modelos m\'as generales que el presentado en (\ref{ionca}). Pudiendo adaptarse a otros modelos en los que se agreguen otros subespacios componentes.

\item Obtener eficiencia en los algoritmos utilizados para computar los c\'alculos en los que se involucran operadores de baja densidad (gran cantidad de ceros).

\item Identificar \emph{cuellos de botella} en el programa a obtener, esto es, subrutinas que tengan un mayor costo computacional frente a las dem\'as.

\item Incluir en el limbladiano t\'erminos que den cuenta de ruido t\'ermico en la cavidad y observar qu\'e ocurre con el sistema ante dicho cambio.

\item Observar el comportamiento de los programas obtenidos ante un mayor n\'umero de fotones aceptados en los modos del campo electromagn\'etico.

\item Estudiar posibilidades de paralelizaci\'on de los c\'odigos obtenidos.
\end{enumerate}

\subsubsection{Objetivos Espec\'ificos}
\begin{enumerate}
 \item Escoger el lenguaje y el compilador a utilizar en la plataforma donde sean ejecutados los programas.
\item Realizar pruebas de tiempo de ejecuci\'on que justifiquen la elecci\'on de las estrategias de c\'omputo a seguir.
\item Justificar que el m\'etodo de Heun siga siendo v\'alido de utilizar. Esto es, observar que la Ecuaci\'on Maestra continue siendo diferenciable y que cumpla la condici\'on de Lipschitz.
\end{enumerate}
