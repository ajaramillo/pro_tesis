%%%Creado por Miguel Navarro inicio de edici�n Agosto 2008

\documentclass[11pt]{article}
%\documentclass[11pt,letter]{article}
%\usepackage[spanish]{babel}
%\usepackage[latin9]{inputenc} % Caractres con acentos. 
\usepackage[spanish]{babel}
\usepackage[latin9]{inputenc}
% \usepackage{latexsym}
\usepackage{amsmath}
\usepackage{amssymb}
\usepackage[dvips]{graphicx}

\usepackage[lofdepth,lotdepth]{subfig}

\usepackage{arydshln} %dashed lines for arrays

\usepackage[algoruled]{algorithm2e}
\SetKwBlock{Begin}{Inicio}{Fin}
\SetKwFor{For}{Para}{Hacer}{Fin}
\SetKwFor{While}{Mientras}{Hacer}{Fin}
\SetKwInput{KwIn}{Entrada}
\SetKwInput{KwOut}{Salida}
\SetKwIF{If}{ElseIf}{Else}{Si}{Entonces}{Si no, Si}{Si no}{Fin}

\usepackage{tikz}

%\usepackage{algorithm}
%\usepackage{algorithmic}
%\input{spanishAlgorithmic}

%\usepackage{makeidx}
%\pagestyle{myheadings}
%\pagestyle{headings}
\usepackage{fancyhdr}%package
%\pagestyle{fancy}%lineas encabezado
%\fancyhf{}
%\fancyfoot[C]{\thepage} 
%\fancyhead[R]{\leftmark}
\pagestyle{empty}
\renewcommand{\headrulewidth}{0.6pt}
\usepackage{pstricks}
\usepackage{setspace}
%\usepackage{tocloft} 
\usepackage{sectsty}
% \usepackage{epsfig}
%\sectionfont{Cap�tulo \centering}
%\sectionfont{\sffamily\raggedright\underline}
%\usepackage{titlesec}
\usepackage{titletoc}

%\usepackage{epsfig}
% \usepackage{pst-grad} % For gradients
% \usepackage{pst-plot} % For axes

\newcommand{\autortesis}{Alfredo Jaramillo Palma}
\newcommand{\titulotesis}{MEJORA DE IMPLEMENTACI�N EN FORTRAN PARA C�LCULOS EN LA GENERACI�N DE DOS FOTONES POR ENCARGO CON RUIDO T�RMICO.}
\newcommand{\profguia}{Dr. Robert Guzm\'an Estrada}
\newcommand{\profexaminadoruno}{Examinador1}
\newcommand{\profexaminadordos}{Examinador2}

\usepackage{pstricks,pstricks-add,pst-math,pst-xkey}
%\renewcommand{\cftchapfont}{\bfseries}
%\renewcommand{\cftchappagefont}{\bfseries}
%\renewcommand{\cftchappresnum}{Cap\'{\i}tulo }
%\newcommand{\cftsecpresnum}{Cap�tulo }
%\renewcommand{\cftchapaftersnum}{.}
%\renewcommand{\cftchapnumwidth}{6em}

%\usepackage{titlesec}
%\usepackage{titletoc}
%\usepackage{tocloft}
%\newcommand{\cftsecpresnum}{}
%\renewcommand{\cftsecfont}{\bfseries}
%\renewcommand{\cftsecpagefont}{\bfseries}
%\renewcommand{\cftsecpresnum}{Cap\'{i}tulo }
%\renewcommand{\cftsecnumwidth}{6em}
\newcommand{\ket}[1]{\vert\, #1 \,\rangle}
\newcommand{\bra}[1]{\langle\, #1 \,\vert}
\newcommand{\braket}[2]{\langle\, #1 \,\vert\, #2 \,\rangle}

 %\hdashlinewidth=2pt \hdashlinegap=2pt

\textheight=21cm
\textwidth=17cm
\evensidemargin=2cm
\topmargin=0.5cm
\oddsidemargin=0cm
\parindent=0cm
\usepackage{multirow}
\usepackage{textcomp}
\usepackage{url}
\usepackage{hyperref}
 

\newcommand{\dimA}{N_a}
\newcommand{\dimB}{N_b}
\newcommand{\dimAt}{N_{at}}
\newcommand{\dimT}{dimt}
\newcommand{\nmod}{N_{mod}}

% \usepackage[usenames,dvipsnames]{pstricks}
% \usepackage{epsfig}
% \usepackage{pst-grad} % For gradients
% \usepackage{pst-plot} % For axes


%\hyphenation{si-gui-ente}
%\hyphenation{res-pues-tas}
\begin{document}

\parskip 1em
%Creado por Miguel Navarro septiembre 2007
\newcommand{\titlem}{``\titulotesis''} %no borrar comillas
\newcommand{\authorm}{\autortesis}
\begin{titlepage}
\hbox{
\parbox{0.18\textwidth}{\begin{flushleft}
\includegraphics[scale=0.1]{cap0/logo_ufro_azul_2}
\end{flushleft}}
\parbox{0.7\textwidth}{
\begin{center}
UNIVERSIDAD DE LA FRONTERA \\ FACULTAD DE INGENIER�A, CIENCIAS Y ADMINISTRACI�N \\ DEPARTAMENTO DE INGENIER�A MATEM�TICA
\end{center}}
\parbox{0.2\textwidth}{\includegraphics[scale=0.335]{cap0/logo_carrera_solo}}
}
\begin{center}
\vspace*{2.5cm}
%\Large{``T�tulo de la Tesis''}\\
\Large{\titlem}\\
\end{center}
\vspace*{2.5cm}
\begin{flushright}
\setlength{\arrayrulewidth}{1.4pt}%grosor \hline
\Large{\begin{tabular}{c}
\hline TRABAJO PARA OPTAR AL T\'{I}TULO \\
DE INGENIERO MATEM\'ATICO \\
\hline
\end{tabular}}
\end{flushright}
\vspace*{0.4cm}
\begin{flushright}
\Large{Profesor Gu�a: \profguia}
\end{flushright}
\vspace*{2cm}
\begin{center}
\Large{\authorm\\2012}
\end{center}
\end{titlepage}
\begin{center}
\large{\titlem}\\
\vspace*{0.25cm}
\Large{Alfredo Jaramillo Palma}\\
\vspace*{4.5cm}
\large{COMISI�N EXAMINADORA}
\end{center}
\vspace*{0.25cm}
\begin{center}
\Large{\profguia\\ Profesor Gu\'ia}
\end{center}
\vspace*{3.5cm}
\begin{minipage}{0.5 \textwidth} % 0.5 \textwidth = 30% ancho de p�gina
\begin{center}
\large{\profexaminadoruno\\ Profesor Examinador 1}
\end{center}
\end{minipage}
\begin{minipage}{0.5 \textwidth} % 0.5 \textwidth = 30% ancho de p�gina
\begin{center}
\large{\profexaminadordos \\Profesor Examinador 1}
\end{center}
\end{minipage}%no quitar
\vspace*{3.5cm}
\hspace*{9.5cm}
{\large{\begin{tabular}{lcc}
Nota trabajo escrito & : &  \\ 
Nota examen & : &  \\ 
Nota final & : &  \\ 
\end{tabular}}}
\begin{center}
\section*{RESUMEN}
\end{center}
\quad En el presente Trabajo de T\'itulo se mejoraron los programas realizados en \cite{gino}, dichos programas solucionan la Ecuaci\'on de Sch\"odinger para el \'atomo de Calcio atrapado en una cavidad, con ciertas condiciones (condici\'on inicial, acoplamiento a modos del campo electromagn\'etico, acomplamientos a l\'aseres) de manera tal que se logran emitir, de manera determin\'istica dos fotones en ciertos modos. La mejora se obtuvo en tanto se redujeron dr\'asticamente los tiempos de c\'omputo y se obtuvieron programas capaces de resolver problemas m\'as gen\'ericos, en particular con ruido t\'ermico en la cavidad. A continuaci\'on resumimos los contenidos de los cap\'itulos de este documento.\\

\begin{description}
 \item [\emph{Ca\'pitulo 1:} ] Se hace una introducci\'on al problema resuelto desde el punto de vista f\'isico y se describen los objetivos del presente Trabajo de T\'itulo.
\item [\emph{Ca\'pitulo 2:} ] Se hace una breve introducci\'on, desde el punto de vista de la F\'isica, de los temas relevantes involucrados en el presente trabajo. Se comienza desde la idea de de Broglie sobre las ondas de materia, se solucionan ciertos casos fundamentales de la Ecuaci\'on de Sch\"odinger desde el punto de vista conceptual; como lo es por ejemplo, el caso del oscilador arm\'onico. Luego, utilizando la notaci\'on de Dirac, se presenta la \emph{primera cuantizaci\'on del campo electromagn\'etico}, donde aparecer\'a el fot\'on, concepto fundamental que en la parte final del cap\'itulo, ser\'a ligado el modelo cu\'antico de un \'atomo a trav\'es de la emisi\'on y absorci\'on de materia. \'Esto \'ultimo es de vital importancia, puesto que es el fundamento mediante el cual podremos emitir fotones de manera determin\'istica.
\item [\emph{Ca\'pitulo 3:} ] En este cap\'itulo, se presenta el modelo matem\'atico utilizado para el estudio de la evoluci\'on del Calcio II, una vez que es llevado al primer estado excitado. Se definen los espacios de Hilbert que compondr\'an el espacio total $\mathcal{H}$, luego se presentan los operadores de $\mathcal{H}$ en $\mathcal{H}$ involucrados en el modelo utilizado y que tendr\'an importancia fundamental en la mejora de los tiempos de c\'omputo del programa. Tambi\'en se presenta formalmente la adici\'on de ruido t\'ermico en la cavidad, a trav\'es de la adici\'on de ciertos t\'erminos en el Limbladiano, operador que se modela las p\'erdidas de energ\'ia del sistema. Finalmente presentamos el m\'etodo num\'erico, que ser\'a utilizado para integrar la ecuaci\'on maestra (nombre de una forma particular de la Ec. de Sch\"odinger: el m\'etodo de Heun.
\item [\emph{Ca\'pitulo 4:} ] \'Este es el cap\'itulo m\'as extenso, y con raz\'on puesto que contien gran parte del trabajo realizado. Se comienza definiendo variables importantes desde el punto de vista computacional e identificando puntos importantes de los algoritmos generales, utilizados para computar la soluci\'on del problema. Luego, se estudian las estructuras de los operadores introducidos el Cap\'itulo 3, y se escriben los algoritmos de c\'omputo de la acci\'on, de cada operador, sobre un eventual elemento de $\mathcal{H}$. Tambi\'en se estudia la acci\'on de las composiciones de estos operadores, que sean importantes en el modelo. Aqu\'i es donde se obtuvo el primero de los logros, puesto que gracias a estos estudios, el n\'umero de operaciones de procesador necesarias se redujo en un orden, lo que signific\'o una gran ganancia en tiempo de c\'omputo. Finalmente, se escriben los principales algoritmos que componen el programa general, utilizando como bloques de construcci\'on los obtenidos anteriormente, esto se hizo de manera tal de lograr la generalidad necesaria, de manera tal que \'estos algoritmos puedan ser reutilizados en modelos m\'as complejos, inclusive del mismo problema f\'isico.
\item [\emph{Ca\'pitulo 5:} ] Presentamos en este cap\'itulo los resultados, cuantitativos y cualitativos de los algoritmos obtenidos en el Cap\'itulo  5. Tambi\'en se presentan los resultados del programa anterior con el fin de realizar una comparaci\'on de los tiempos de c\'alculo, una vez que se establece que las soluciones obtenidas son las mismas.
\item [\emph{Ca\'pitulo 6:} ] En \'este cap\'itulo se realiza el an\'alisis, de los resultados y de sus consecuencias en tanto logros obtenidos e impacto en trabajos futuros.
\item [\emph{Anexo:} ] Sobre el Software utilizado en el presente trabajo.
\end{description}

\quad Como validaci\'on de los resultados obtenidos, se utilizaron, tanto los resultados del trabajo anterior, conocimiento de la Teor\'ia y resultados de publicaciones estudiadas, como \cite{single-photon} y \cite{single-photon-walther}. Los logros m\'as importantes de este trabajo, se obtuvieron gracias a la identificaci\'on de las distintas estructuras de los operadores involucrados, y por ende de la acci\'on que \'estos realizan sobre alg\'un elemento del espacio de Hilbert $\mathcal{H}$, inclusive no siendo necesario el c\'omputo de \'estos operadores sino, obteniendo directamente el resultado de su operaci\'on y el de sus composiciones. %descripci�n del documento
\vspace*{16cm}
\begin{flushright}
\large{A Ida y Victor.}
\end{flushright}
\begin{center}
\section*{AGREDECIMIENTOS}
\end{center}\vspace{2cm}
\quad Agradezco en primer lugar a mis padres, por su apoyo y confianza incondicionales; a mis amigos, en especial Miguel Navarro, por sus consejos en el presente trabajo y quien tambi�n es el autor de la estructura de este documento en \LaTeX; a Robert Guzm�n, mi profesor gu�a por su apoyo y paciencia; a Ra�l Benavidez, Fernando Marchant, Francisco Pe\~na, Carlos Abarz�a y Julio L�pez, profesores que de una u otra forma, han participado positivamente en mi formaci�n como profesional; y a Luis S�nchez, compa\~nero de carrera con quien compartiera mis primeros a\~nos con fruct�feras discusiones. Finalmente agradezco al Centro de Excelencia en Modelaci\'on y Computaci\'on Cient\'ifica de la Universidad de La Frontera por el acceso prestado al servidor KUDI, plataforma donde se ejecutaron los programas obtenidos.
\renewcommand{\contentsname}{�ndice} 
\renewcommand{\appendixname}{Ap�ndice} 
\renewcommand{\figurename}{Figura}
\renewcommand{\listfigurename}{�ndice de Figuras} 
\renewcommand{\tablename}{Tabla} 
\renewcommand{\listtablename}{�ndice de tablas} 
\renewcommand{\refname}{}%Bibliograf�a} 
\renewcommand{\listalgorithmcfname}{Lista de Algoritmos}%
\renewcommand{\algorithmcfname}{Algoritmo}%

\onehalfspacing
\parindent 3em
\titlecontents{section}
  [0pt]
  {\addvspace{0.1pc}}%
  {\contentsmargin{0pt}%
  \bfseries
  \makebox[30pt][l]{Cap�tulo\ \thecontentslabel.}%
  \hspace*{2.5em}}
  {\contentsmargin{0pt}}
  {\titlerule*[100pc]{.}\contentspage}
  [\addvspace{.1pc}]
\tableofcontents %�ndice
\newpage
\listoftables %lista de tablas
\newpage
\listoffigures %lista de figuras
\newpage
\listofalgorithms %lista de algoritmos
\newpage
\pagestyle{fancy}%lineas encabezado
\renewcommand{\footrulewidth}{0.5pt}
\renewcommand{\headrulewidth}{0.5pt}
\newcommand{\sectionm}[1]{\section{#1}\fancyhead[R]{Cap�tulo \thesection.\quad #1 }\parindent=0em}%\addcontentsline{toc}{section}{Cap�tulo \thesection.\quad #1}}
\setcounter{page}{1}
\makeatletter

\def\@seccntformat#1{\@ifundefined{#1@cntformat}%
{\csname  the#1\endcsname\quad}%    default
{\csname  #1@cntformat\endcsname}%  individual  control
}
\def\section@cntformat{Cap�tulo \thesection.\quad}
\def\subsection@cntformat{\thesubsection.\quad}
\makeatother
\sectionfont{\centering\underline}

%\newcommand{\thepagem}{\hspace*{2cm}\thepage}

\newcommand{\Pag}[1]{
\stepcounter{capitulo}
\pagestyle{empty}%paginas en blanco
\vspace*{6.5cm}
\begin{center}
\LARGE{\textbf{CAP�TULO \thecapitulo}}\\
\vspace*{0.2cm}
\LARGE{\textbf{#1}}
\end{center}
\addtocounter{page}{-1}
\newpage

\pagestyle{fancy}%lineas encabezado
\fancyhf{}
\fancyfoot[C]{\titlem\hspace*{5cm}\thepage}
%\fancyfoot[R]{\thepagem} 
}

\newcounter{capitulo}
\clearpage
 \makeatletter
 \@addtoreset {equation}{section}
 \makeatother

\renewcommand{\theequation}{\arabic{section}.\arabic{equation}}
 
%%inicio del documento

%%PRIMERA PARTE : INTRO, F�sica-Matem�tica
\pagestyle{empty}%paginas en blanco
\begin{center}
\part*{Problema F\'isico-Matem\'atico}
\end{center}
\Pag{Introducci�n}
\sectionm{Introducci�n}
%\section{Introducci\'on}
\subsection{Introducci\'on}
\subsection{Objetivos Generales del Trabajo de T\'itulo}
%\subsubsection{Objetivos espec\'ificos}
\quad A continuaci\'on se presentan los objetivos deseados de alcanzar en el presente Trabajo de T\'itulo.

\begin{enumerate}
 \item Implementar c\'odigos din\'amicos que resuelvan el problema a presentar en (\ref{sec:problema_resolver}). Por \emph{dinamismo} entenderemos que dichos c\'odigos sean reutilizables en modelos m\'as generales que el presentado en (\ref{ionca}). Pudiendo adaptarse a otros modelos en los que se agreguen otros subespacios componentes.

\item Obtener eficiencia en los algoritmos utilizados para computar los c\'alculos en los que se involucran operadores de baja densidad (gran cantidad de ceros).

\item Identificar \emph{cuellos de botella} en el programa a obtener, esto es, subrutinas que tengan un mayor costo computacional frente a las dem\'as.

\item Incluir en el limbladiano t\'erminos que den cuenta de ruido t\'ermico en la cavidad y observar qu\'e ocurre con el sistema ante dicho cambio.

\item Observar el comportamiento de los programas obtenidos ante un mayor n\'umero de fotones aceptados en los modos del campo electromagn\'etico.

\item Estudiar posibilidades de paralelizaci\'on de los c\'odigos obtenidos.
\end{enumerate}

\subsubsection{Objetivos Espec\'ificos}
\begin{enumerate}
 \item Escoger el lenguaje y el compilador a utilizar en la plataforma donde sean ejecutados los programas.
\item Realizar pruebas de tiempo de ejecuci\'on que justifiquen la elecci\'on de las estrategias de c\'omputo a seguir.
\item Justificar que el m\'etodo de Heun siga siendo v\'alido de utilizar. Esto es, observar que la Ecuaci\'on Maestra continue siendo diferenciable y que cumpla la condici\'on de Lipschitz.
\end{enumerate}
 %descripci�n gral. del trabajo. Objetivos.
\Pag{Fundamentos F�sicos}
\sectionm{Fundamentos F�sicos}
\subsection{Ondas de Materia}
\quad Luis de Broglie, en 1924, propuso la existencia de ondas de materia en su tesis doctoral que present� en la Facultad de Ciencias de las Universidad de Par�s. En principio, a pesar de lo original y concienzudo de su trabajo, sus ideas fueron consideradas carentes de realidad f�sica por su aparente falta de evidencias experimentales. Fue Albert Einstein quien reconoci� su importancia y validez y atrajo la atenci�n de otros f�sicos hacia �l.

\quad La hip�tesis de de Broglie consist�a en que el comportamiento dual de la radiaci�n, onda-part�cula, deber�a ser igualemente aplicable a la materia. Por tanto, una part�cula de materia, como el electr�n, tendr�a asociada un funci�n de onda que gobierne su movimiento,  tal como el fot�n tiene asociada una funci�n de onda de luz. �l propuso que los efectos ondulatorios de la materia est�n relacionados con los aspectos corpusculares de la misma forma cuantitativa que en el caso de la radiaci�n. De acuerdo con de Broglie, tanto para la materia como para la radiaci�n, la energ�a total $E$ de un ente se relaciona con la frecuencia $v$ de la onda asociada a su movimiento por medio de la ecuaci�n
\begin{equation}
 E=h\,v,
\end{equation}

y el impulso $p$ del ente se relaciona con la longitud de onda $\lambda$ de la onda asociada por la ecuaci\'on
\begin{equation}
 p=\frac{h}{\lambda}.\label{eqphl}
\end{equation}

\quad En lo anterior, los conceptos corpusculares, energ�a $E$ e impulso $P$, se relacionan con los conceptos ondulatorios, frecuencia $v$ y longitud de onda $\lambda$ a trav\'es de la constante de Planck $h$. A la ecuaci\'on (\ref{eqphl}) escrita en la forma siguiente, se le denomina ``relaci�n de de Broglie''. 
\begin{equation}
 \lambda=\frac{h}{p}, \label{relaciondebroglie}
\end{equation}

que predice la ``longitud de onda de de Broglie'' de un ``onda de materia'' asociada con el movimiento de una part�cula de impulso $p$.

\quad Elsasser, en 1926, propuso que la naturaleza ondulatoria de la materia pod�a ser probada del mismo modo como inicialmente se prob� la naturaleza ondulatoria de los rayos X, a saber, haciendo incidir un haz de electrones, con la energ�a apropiada, sobre un s�lido cristalino. Los �tomos del cristal son utilizados como un arreglo tridimiensional de centros dispersores para la onda electr�nica, y por lo tanto, deber�n dispersar fuertemente a los electrones en direcciones caracter�sticas, de igual forma en que se produce la difracci�n de los rayos X. Davisson y Germer confirmaron esta idea en Estados Unidos y Thomson en Escocia.

\quad Davisson y Germer, produjeron un aparato que produc�a un haz de electrones mediante un filamento caliente, luego �stos eran acelerados a trav�s de un diferencia de potencial $V$ y emergen de un ``ca\~n�n' electr�nico'' con una energ�a cin�tica $eV$. Este haz de electrones incide sobre un monocristal de nickel y un detector era ubicado de manera adecuada para medir la dispersi�n en un �ngulo determinado. El resultado de este experimento fue la existencia de ``picos'' en los gr�ficos de detecci�n versus �ngulo del detector, lo que confirm� cualitativamente la validez del postulado de de Broglie, puesto que �ste fen�meno (an�logo a las ``reflexiones de Bragg'' en rayos X) s�lo puede ser explicado como una interferencia constructiva en las ondas dispersadas por el arreglo peri�dico de los �tomos hacia los planos del cristal. Las part�culas cl�sicas no pueden mostrar interferencia, mientras que las ondas s� lo hacen. Las interferencia que aqu� se implica, no ocurre entre las ondas asociadas con un electr�n y las ondas asociadas con otro, sino que se refiere a la interferencia que ocurre entre diferentes partes de la onda asociada con un electr�n y que han sido dispersadas por varias regiones del cristal. Esto se puede demostrar, si se utiliza un haz de electrones con una intesidad tan baja que los electrones pasan por un aparato uno por uno y mostrando que el patr�n de electrones despersados permanece invariante.

\quad G.P. Thomson, en 1927, demostr� la difracci�n de haces de electrones que pasan a trav�s de pel�culas delgadas, confirmando independientemente y en detalle, la relaci�n de de Broglie (\ref{relaciondebroglie}). As� como el experimento de Davisson-Germer es an�logo al de Laue en difracci�n de rayos X, el experimento de Thomson es an�logo al m�todo de Debye-Hull-Scherrer de difracci�n de rayos X por un polvo. Thomson utiliz� electrones de mayor energ�a, que son mucho m�s penetrantes, de modo que son varios cientos de planos at�micos los que contribuyen a la onda difractada.

\subsection{Teor\'ia de Sch\"odinger de la Mec�nica Cu�ntica}

\quad Los experimentos mencionados en la secci�n anterior, demostraron contundentemente, que las part�culas, en sistemas microsc�picos se mueven de acuerdo a las leyes del movimiento ondulatorio de alg�n tipo, y no de acuerdo con las leyes newtonianas que obedecen las part�culas en los sistemas macrosc�picos. As�, una part�cula microsc�pica, act�a como si ciertos aspectos de su comportamiento estuviesen gobernados por el comportamiento de una onda asociada de de Broglie o una funci�n de onda. Los experimentos considerados s�lo tratan con casos simples (tales como part�culas libres y osciladores arm�nicos simples, etc.) que se pueden analizar con procedimientos sencillos (que involucran la aplicaci�n directa del postulado de de Broglie, postulado de Planck, etc.). Pero ciertamente es deseable el estar preparado para tratar los casos m�s complicados que ocurren en la naturaleza. Para hacer esto se deber� tener un procedimiento m�s general que se pueda utilizar para tratar el comportamiento de las part�culas de cualquier sistema microsc�pico. La ``Teor�a de Sch\"odinger de la Mec�nica Cu�ntica'' proporciona tal procedimiento.

\quad Esta teor�a especifica las leyes del movimiento ondulatorio que obedecen las part�culas de cualquier sistema microsc�pico. Lo cual se hace especificando, para cada sistema, la ecuaci�n que describe el comportamiento de tal funci�n de onda y especificando tambi�n la conexi�n entre este comportamiento y el de la part�cula. As�, se tien una extensi�n del postulado de de Broglie. M�s a�n, existe una estrecha relaci�n entre ella y la teor�a de Newton del movimiento de part�culas en sistemas macrosc�picos. La teor�a de Sch\"odinger es una generalizaci�n que incluye a la teor�a de Newton como un caso especial (en el l�mite de macrosc�pico-microsc�pico), tal como la teor�a de Einstein de la relatividad es una generalizaci�n que incluye a la teor�a de Newton como un caso especial (en el l�mite de velocidades bajas-altas).

\quad Para presentar entonces esta teor�a haremos la siguiente formulaci�n, con respecto a un part�cula:

\begin{enumerate}
 \item En relaci�n al concepto cl�sico de trayectoria, subtituiremos el concepto de estado variante en el tiempo. El ``estado cu�ntico'' de una part�cula, como el electr�n, se caracteriza por una ``funci�n de onda'' $\Psi$ que depende de la posici�n y el tiempo $\Psi(\mathbf{r},t)$, que contiene toda la informaci�n acerca de la part�cula.

\item $\Psi(\mathbf{r},t)$ es interpretada como la amplitud de probabilidad que en el instante $t$, la part�cula se encuentre en la posici�n $\mathbf{r}$ (Interpretaci�n de Born), $d\mathbb{P} (\mathbf{r},t)=C\vert \Psi(\mathbf{r},t)\vert^2d^3\mathbf{r}$

donde $C$ es una constante de normalizaci�n.

\item \emph{El principio de descomposici�n espectral} aplica a la medici�n de cualquier cantidad f�sica:
\begin{enumerate}
 \item El resultado debe pertenecer a un conjunto de valores propios $\{a\}$.
\item Cada valor propio $a$ tiene asociado un autoestado, es decir, una autofunci�n $\Psi_a(\mathbf{r},t)$. Esta funci�n es tal que, si $\Psi(\mathbf{r},t_0)=\Psi_a(\mathbf{r})$ (donde $t_0$ es el tiempo en el cual la medici�n es realizada), la medici�n dar� como resultado siempre $a$.
\item Para cada $\Psi(\mathbf{r},t)$, la probabilidad $\mathbb{P}_a$ que al medir, sea retornado el valor propio $a$ en el tiempo $t_0$ es encontrado mediante la descomposici�n de $\Psi(\mathbf{r},t)$ en t�rminos de las autofunciones $\Psi_a(\mathbf{r})$:

 \begin{equation}
  \Psi(\mathbf{r},t_0)=\sum_a c_a \Psi_a(\mathbf{r}),
 \end{equation}

entonces:
\begin{equation}
 \mathbb{P}_a=\frac{\vert c_a\vert^2}{\sum_a \vert c_a\vert^2}.
\end{equation}

\item Si la medici�n resulta ser $a$, la funci�n de onda de la part�cula, inmediatamente despu�s de la medici�n es:
\begin{equation}
 \Psi'(\mathbf{r},t_0)=\Psi_a(\mathbf{r}).
\end{equation}

\end{enumerate}
\item Queda por escribir la ecuaci�n, que describe la evoluci�n de la funci�n de onda $\Psi(\mathbf{r},t)$. �sta puede ser introducida en forma natural utilizando las relaciones de de Broglie y de Planck. Sin embargo, tan s�lo asumiremos su validez. Si la part�cula de masa $m$ est� sujeta a la acci�n de un potencial $V(\mathbf{r},t)$, la ecuaci�n de Schr\"odinger toma la forma:

\begin{equation}
i\hbar \frac{\partial}{\partial t}\Psi(\mathbf{r},t)=-\frac{\hbar^2}{2\,m}\triangledown^2 \Psi(\mathbf{r},t)+V(\mathbf{r},t)\,\Psi(\mathbf{r},t).\label{eqscho}
\end{equation}

\quad Se observa que esta ecuaci�n es lineal y homog�nea en $\Psi$. Como consecuencia, para part�culas materiales, es v�lido el principio de superposici�n que, combinado con la interpretaci�n de $\Psi$ como amplitud de probabilidad, es la raz�n de los efectos tipo ondas. M�s a�n, la ecuaci�n diferencial (\ref{eqscho}) es de primer orden con respecto al tiempo. Esta condici�n es necesaria si se desea que el estado inicial al tiempo $t_0$, caracterizado por $\Psi(\mathbf{r},t_0)$ determine el estado posterior.

\quad Existe entonces una analog�a fundamental entre materia y radiaci�n: en ambos casos, una descripci�n correcta de los fen�menos requiere la introducci�n de conceptos cu�nticos, y en particular, de la idea de funci�n de onda.
\end{enumerate}

\subsection{Soluciones de la Ecuaci�n de Sch\"odinger independientes del tiempo}
\quad En esta secci�n mostraremos algunas soluciones a la Ecuaci�n de Sch\"odinger suponiendo que el potencial $V$ que act�a sobre la part�cula es independiente del tiempo. Adem\'as supondremos que la part�cula tiene s�lo un grado espacial de libertad, en particular, en $x$. Podremos apreciar, que al ser confinada la part�cula en una regi�n finita, las energ�as permitidas ser�n discretas, a lo contrario del caso cl�sico en el cual los valores de energ�a posible forman un continuo, tambi�n introduciremos los operadores de subida y bajada, que son de gran utilidad tambi�n en los casos m�s generales. Estos operadores ser�n llamados posteriormente, en el caso de radiaci�n, operadores de creaci�n y aniquilaci�n (de fotones).
\\
\subsubsection{Estados Estacionarios}\label{estadosestacionarios} \quad La Ecuaci�n de Sch\"odinger, en el caso unidimensional tiene la forma:
\begin{equation}
 i\hbar \frac{\partial \Psi}{\partial t}=-\frac{\hbar^2}{2\,m}\frac{\partial^2\Psi}{\partial x^2} +V\,\Psi,
\end{equation}

\quad Adem\'as supondremos que el potencial $V$ s�lo depende de la posici�n, es decir tiene la forma $V(x)$. As�, podemos utilizar la t\'ecnica de separaci�n de variables. Buscaremos entonces soluciones de la forma:
\begin{equation}
 \Psi(x,t)=\psi(x)\phi(t),
\end{equation}

\quad Se tiene entonces:

\begin{equation*}
 \frac{\partial \Psi}{\partial t}=\psi\frac{d \phi}{dt},\;\frac{\partial^2 \Psi}{\partial x^2}=\frac{d^2 \psi}{\partial x^2}\phi
\end{equation*}
 
luego, la Ecuaci�n de Sch\"odinger en este caso se puede escribir:
\begin{equation}
 i\hbar \frac{1} {\phi} \frac{d\phi}{dt}=-\frac{-\hbar^2}{2m}\frac{1}{\psi}\frac{d^2\psi}{dx^2}+V.
\end{equation}

\quad Introducimos entonces una constante de separaci�n, que por conveniencia llamaremos $E$, de manera que (tras un breve arreglo algebraico):
\begin{equation}
 \frac{d\phi}{dt}=-\frac{i\,E} {\hbar}\psi,\label{eqschphi}
\end{equation}

y
\begin{equation}
-\frac{\hbar^2}{2m}\frac{d^2\psi}{dx^2}+V\psi=E\psi. \label{eqschpsi}
\end{equation}

\quad Hemos obtenido entonces dos ecuaciones independientes para la parte espacial $\psi$ (\ref{eqschphi}) y la parte temporal $\phi$ (\ref{eqschpsi}) de $\Psi$. La soluci�n para (\ref{eqschphi}) es inmediata integrando y se puede escribir (dejando la constante de integraci�n en la parte temporal) como:
\begin{equation}
 \phi(t)=e^{-iEt/\hbar}.
\end{equation}

\quad La segunda ecuaci�n es conocida como la ``Ecuaci�n de Sch\"odinger independiente del tiempo''. En las pr�ximas secciones de este cap�tulo asumiremos formas especiales para $V(x)$. Antes de realizar esto, debemos destacar algunas caracter�sticas importantes de este tipo de soluciones:

\begin{enumerate}

 \item Son soluciones estacionarias, a pesar que $\Psi$ depende expl�citamente de $t$,

\begin{equation*}
 \Psi(x,t)=\psi(x)e^{-iEt/h},
\end{equation*}

pero el significado f�sico est� dado por el cuadrado de esta funci�n,
\begin{equation}
\vert \Psi(x,t)\vert^2=\Psi^*\Psi=\psi^*e^{iEt/h}\psi e^{-iEt/h}=\vert \psi(x)\vert^2,\label{cuad} 
\end{equation}

que no depende del tiempo. Lo mismo ocurrir� con cualquier variable aleatoria cuya funci�n de probabilidad asociada sea esta funci�n de onda, o m�s exactamente, su cuadrado (\ref{cuad}). En la secci�n (\ref{postulados}) veremos que esto �ltimo significa que el valor de expectaci�n de cualquier variable f�sica posible de ser medida, es constante en el tiempo cuando el sistema se encuentra en una estado con funci�n de onda de este tipo.

\item Son estados con \emph{energ�a total definida}. En Mec�nica Cl�sica, la energ�a total (cin�tica m�s potencial) est� dada por el Hamiltoniano:
\begin{equation}
 H(x,p)=\frac{p^2}{2m}+V(x)
\end{equation}

\quad El correspondiente Operador Hamiltoniano, obtenido mediante la sustituci�n can�nica $p\rightarrow (\hbar /i)(\partial/\partial x)$, es entonces\footnote{Utilizamos ( $\hat{ }$ ) para distinguir la variable f�sica de su operador correspondiente.}:
\begin{equation}
 \hat{H}=-\frac{\hbar^2 }{2m}\frac{\partial^2}{\partial x^2}+V(x).
\end{equation}

\quad As�, la Ecuaci�n de Sch\"odinger independiente del tiempo puede escribirse como:
\begin{equation}
 \hat{H}\psi=E\psi,\label{ecvalprop}
\end{equation}

que es una ecuaci�n de valores propios. Y el valor esperado de la energ�a es:
\begin{equation}
 \langle H \rangle = \int \psi^* \hat{H} \psi dx=E\int \vert \psi \vert^2dx=E\int \vert \Psi\vert^2dx=E.
\end{equation}

Adem�s,
$$\hat{H}^2\psi=\hat{H}(\hat{H}\psi)=\hat{H}(E\psi)=E(\hat{H}\psi)=E^2\psi$$

luego
$$\langle H^2 \rangle = \int \psi^* \hat{H}^2 \psi dx=E^2 \int \vert \psi \vert^2dx=E^2.$$

\quad As�, la varianza de $H$ es:
\begin{equation}
 \sigma_H=\langle H^2\rangle-\langle H\rangle^2=E^2-E^2=0.
\end{equation}

\quad Por tanto, en cualquier momento, si se mide la energ�a total del sistema, el valor ser� siempre el mismo, $E$. En resumen, una soluci�n separable tiene la propiedad que en toda medici�n de la energ�a total retornar� un un cierto valor caracter�stico de esa soluci�n. Que es la raz�n para elegir $E$ como constante de separaci�n.
\item Por tratarse de autofunciones de un operador hermitiano, de (\ref{ecvalprop}) resulta que toda soluci�n puede ser escrita como combinaci�n lineal de las autofunciones estacionarias, entonces, la soluci�n general est� dada por:

 \begin{equation}
  \Psi(x,t)=\sum_{n=1}^\infty c_n\psi_n(x) e^{-iE_nt/\hbar},
 \end{equation}

con
$$\hat{H}\psi_n=E_n\psi_n.$$
\end{enumerate}

\subsubsection{El pozo cuadrado de paredes infinitas}\label{pozoinfinito}

\quad Supongamos un potencial de la forma:
\begin{equation}
V(x)=\left\{
\begin{array}{ll}
 0, & \mbox{si }0\leq x\leq a,\\
\infty,& \mbox{otro caso}
\end{array}\right..
\end{equation}

\quad Una part�cula bajo estas restricciones es libre en la regi�n interior del pozo ($0 \leq x \leq a$, donde se encuentra inicialmente), excepto en los bordes de esta, donde una fuerza infinita no permite que ella escape. As�, fuera del pozo $\psi(x)=0$, y dentro del pozo, la Ecuaci�n de Sch\"odinger independiente del tiempo se escribe:
\begin{equation}
 -\frac{\hbar^2}{2m}\frac{d^2\psi}{dx^2}=E\psi,
\end{equation}

o
\begin{equation}
 \frac{d^2\psi}{dx^2}=-k^2\psi,\;\;\mbox{ con }k\equiv\frac{\sqrt{2mE}}{\hbar}.
\end{equation}

\quad Que es la ecuaci�n del oscilador arm�nico, cuya soluci�n general est\'a dada por
\begin{equation}
 \psi(x)=A\sin kx +B \cos kx,
\end{equation}

con $A,\,B$ constantes a ser determinadas dadas las condiciones de borde. Tras un breve an�lisis sobre las condiciones de borde debido a las ``paredes'' del pozo y sobre la continuidad de $\psi$, se demuestra r�pidamente que para el caso del pozo infinito se tien $B=0$ y que las soluciones posibles son de la forma
\begin{equation}
 \psi(x)=A\sin kx.
\end{equation}

\quad Y nuevamente, dada las condiciones de borde, se deben cumplir las condiciones de periocidad sobre los posibles valores de la constante $k$:
\begin{equation}
 k_n=\frac{n\pi}{a},\;\;\mbox{ con }n=1,2,3,\ldots
\end{equation}

\quad Adem\'as los posibles valores de la energ�a $E$ son:
\begin{equation}
 E_n=\frac{\hbar^2k_n^2}{2m}=\frac{n^2\pi^2\hbar^2}{2ma^2}.
\end{equation}

\quad Entonces, aqu� es donde se presenta una gran diferencia con el caso cl�sico, puesto que para esta part�cula confinada los valores de energ�a posibles est�n discretizados (puede demostrarse lo contrario para otros de no confinamiento), dicho de otra forma, existen valores de energ�a permitidos y otros prohibidos por la teor�a. Las autofunciones correspondientes, tras normalizar son:
\begin{equation}
 \psi_n(x)=\sqrt{\frac{2}{a}}\sin\left( \frac{n\pi}{ a}x \right)
\end{equation}

\quad Se tiene entonces un conjunto infinito numerable de posibles soluciones, cada una con un posible valor de energ�a para cada entero $n$. El caso $n=1$ es el de menor energ�a y es llamado \emph{estado fundamental}, los dem�s son llamados \emph{estados excitados}. Es de suma importancia notar que el menor valor posible de energ�a 
$$E_1=\frac{\pi^2\hbar^2}{2ma^2},$$

no es nulo, este es otro de los resultados fundamentales de la Mec�nica Cu�ntica, que por ejemplo, en el caso de radiaci�n, implica que a�n en el vac�o existe un nivel de energ�a m�nimo permitido no nulo; en el caso de niveles de energ�a del �tomo, este m�nimo es el que permite la estabilidad del �tomo (v�ase p�g. 287 de \cite{eisberg} y 176 de \cite{dutra}).\\

\quad Hemos revisado superficialmente los principales resultados del pozo de potencial de paredes infinitas, a continuaci�n mostraremos su aplicaci�n a otro importante caso.
\subsubsection{El oscilador arm�nico}\label{osciladorarmonico}
\quad Recordemos el caso cl�sico unidimensional de una masa $m$ unida a un resorte de constante restitutiva $k$. Se tiene entonces que la fuerza neta (sin otras ``fuentes'' de fuerza presentes) actuante sobre la masa obedece la Ley de Hooke:$$F=-kx,$$

cuya soluci�n es
$$x(t)=A\sin(\omega t)+B\cos(\omega t),$$

donde
$$\omega=\frac{k}{m}$$

es la frecuencia angular de oscilaci�n. La energ�a potencial asociada se puede escribir como:
\begin{equation}
 V(x)=\frac{1}{2}kx^2,\label{potosc}
\end{equation}

\quad A pesar que no exiten los osciladores perfectos (aparecen t�rminos discipativos), sabemos, por el teorema de Taylor, que todo potencial puede ser aproximado, a segundo orden al rededor del punto de oscilaci�n $x_0$ (siendo este punto un m�nimo local), por un potencial de la forma (\ref{potosc}). Esto lo podemos escribir as�:
$$V(x)=V(x_0)+V'(x_0)(x-x_0)+\frac{1}{2}V''(x_0)(x-x_0)^2+O\left( (x-x_0)^3 \right)$$

haciendo un poco de �lgebra, incluyendo la condici�n $V'(x_0)=0$ y eliminando los t�rminos de orden mayor se tiene:
$$V(x)\cong \frac{1}{2}V''(x_0)(x-x_0)^2,$$

que es de la misma forma que (\ref{potosc}) tomando $k=V''(x_0)$. Aqu� radica la importancia del oscilador arm�nico, en primera instancia, todo movimiento oscilante puede ser aproximado por un oscilador arm�nico con potencial de la forma (\ref{potosc}).\\

\quad Un problema cu�ntico en que esta aproximazci�n es de suma importancia, es el de estudiar el comportamiento de un electr�n ante la presencia de un potencial central, esto lo podemos hacer estudiando primero el caso unidimensional. Para esto supondremos un potencial de la forma:
\begin{equation}
 V(x)=\frac{1}{2}m\omega^2x^2
\end{equation}

donde hemos inclu�do la masa en la constante $k$. Podemos entonces aplicar lo visto en la secci�n (\ref{estadosestacionarios}), puesto que no existe dependencia temporal en el potencial. Resolvemos entonces la Ecuaci�n de Sch\"odinger independiente del tiempo:
\begin{equation}
 -\frac{\hbar^2}{2m}\frac{d^2\psi}{dx^2}+\frac{1}{2}m\omega^2x^2\psi=E\psi.\label{schopotar}
\end{equation}

\quad Para resolver la ecuaci�n, introduciremos el operador de momento (no discutiremos su origen) $p\equiv (\hbar/i)d/dx$ y los operadores:
\begin{equation}
 a_\pm\equiv \frac{1}{\sqrt{2\hbar m\omega}}\left(\mp ip+m\omega x\right).
\end{equation}

\quad Se puede demostrar sin mucha dificultad que dados estos operadores, se tiene:
\begin{equation}
 H=\hbar \omega\left(a_+a_-+\frac{1}{2}\right)
\end{equation}

 donde hemos utilizado adem\'as las importantes relaciones:

\begin{eqnarray}
[x,p]&=&i\hbar \\
\left[a_-,a_+\right]&=&1.
\end{eqnarray}

\quad La primera de estas relaciones aparece, por ejemplo, como causa fundamental del principio de incertidumbre de Heissenberg (v�ase p�g. 110 de \cite{griffiths}). Con esto, la ecuaci�n (\ref{schopotar}) toma la forma:
\begin{equation}
 \hbar \omega\left(a_+ a_- + \frac{1}{2}\right)=E\psi.\label{eq1}
\end{equation}

\quad As�, autofunciones de los operadores $a_+,a_-$ y $a_+a_-$ cobran vital importancia en la soluci�n de (\ref{eq1}). No es muy dif�cil probar (v�ase por ejemplo \cite{griffiths} secci�n 2.3.1 o \cite{dutra} secci�n 2.4.1) que existir� una autofunci�n fundamental $\psi_0$, en el sentido de estar asociado al m�nimo autovalor del operador $a_+a_-$ igual a $0$ y por ende, el m�nimo valor del Hamiltoniano. Adem\'as el n�mero de autofunciones es numerable y a cada entero natural podemos asociar una autofunci�n $\psi_n$ tal que se cumplen:

\begin{eqnarray}
  a_-\psi_n=\sqrt{n}\psi_{n-1}&,&\;a_+\psi_n=\sqrt{n}\psi_{n+1}\label{subida1}\\
a_+a_-\psi_n=n\psi_n&, &\; a_-\psi_0=0\label{subida2}\\
H(a_+\psi_n)=(E+\hbar\omega)(a_-\psi)&, &\; H(a_-\psi_n)=(E-\hbar\omega)(a_-\psi).  \label{subida3}
\end{eqnarray}

\quad Dadas las ecuaciones anteriores es que $a_-$ y $a_+$ reciben los nombres de operadores de bajada y subida respectivamente. Resolviendo la ecuaci�n (\ref{schopotar}) para el nivel fundamental, encontramos que
\begin{equation}
 \psi_0(x)=\left(\frac{m\omega}{\pi\hbar}\right)^{1/4}e^{-\frac{m\omega}{2\hbar}x^2},
\end{equation}


y luego aplicando las ecuaciones (\ref{subida1}) a (\ref{subida3}) se tiene que se pueden caracterizar todas las soluciones en forma resumida y en funci�n de la autofunci�n fundamental $\psi_0$:
\begin{equation}
 \psi_n=A_n(a_+)^n\psi_0(x),\mbox{  con }E_n=\left(n+\frac{1}{2}\right)\hbar\omega,
\end{equation}

donde $A_n$ es la constante de normalizaci�n. As�, aplicando el operador de subida, podemos recuperar todas las soluciones del oscilador arm�nico.

\subsection{Notaci�n de Dirac}\label{dirac}
\quad La Mec�nica Cu�ntica, cuyos postulados revisaremos en la siguiente secci�n, descansa en la hip�tesis que las funciones de onda asociadas a los entes f�sicos estudiados pertenecen a un espacio vectorial de caracter�sticas especiales, un espacio de Hilbert. Sin querer profundizar demasiado, recordemos que estos espacios vectoriales poseen producto interno y adem�s son completos. El formalismo de Dirac, nos provee de una notaci�n abstracta en lo que prima son las cualidades algebraicas y funcionales de los entes matem�ticos (en este caso vectores) con los que se trabaja. As�, en esta notaci�n, la suma, la multiplicaci�n por escalar, el producto interno, la conjugaci�n, las proyecciones sobre una base de vectores, etc. surgen de manera natural realizando operaciones sencillas sobre los grafemos utilizados.\\

\quad Sea $H$ un espacio vectorial con producto interno, un elemento de $H$, con cuerpo $\mathbb{K}$ ser� llamado \emph{ket}. Ser� representado con el s�mbolo $\ket{\cdot}$, reemplazando $\cdot$ por cualquier s�mbolo que nos sirva para identificar ket en particular, por ejemplo: $\ket{\psi}$.

\begin{enumerate}

\item Producto por escalar. Si $\ket{\psi}\in H$ y $\lambda \in \mathbb{K}$ entonces el producto del escalar por el vector ser� presentado en cualquiera de las siguientes formas:
$$\lambda\ket{\psi}=\ket{\lambda\psi}=\ket{\psi}\lambda.$$
\item Conjugaci�n. El conjugado del ket $\ket{psi}$ ser� denotado por: $$\bra{\psi}.$$

\item Producto interno. El producto interno de $\ket{\psi}$ por $\ket{\phi}$ ser� escrito como:
$$\braket{\psi}{\phi}.$$

\item Proyecci�n. Supongamos que deseamos proyectar $\ket{\psi}$ sobre $\ket{\phi}$, entonces con esta notaci�n esta proyecci�n se puede escribir de forma natural (sea $P_\phi(\psi)$ la proyecci�n):
$$P_\phi(\psi)=\left(\ket{\phi}\bra{\phi}\right)\ket{\psi}=\ket{\phi}\braket{\phi}{\psi}=\braket{\phi}{\psi}\ket{\phi}.$$
\end{enumerate}

\quad Para mayor profundidad en este formalismo, el lector puede revisar \cite{cohen} a partir de la p�g. 108.
\subsection{Postulados de la Mec�nica Cu�ntica}\label{postulados}

\quad En esta secci�n, $H$ nuevamente es un espacio de Hilbert, que en general, se trata de la composici�n de espacios, uno para la parte espacial, y otros, como por ejemplo, uno que describa los grados de libertad de spin de la part�cula (v�ase \cite{eisberg} secci�n 8.3).\\

\quad Los postulados son:

\begin{enumerate}
 \item {\it En un tiempo particular $t_0$, el estado del sistema f�sico est� definido por un ket $\ket{\psi(t_0)}$ perteneciente al espacio $H$.}\\

\quad Obervamos entonces que el principio de superposici�n est� impl�cito en este postulado dado que el estado del sistema pertenece a un espacio vectorial.
 \item {\it Toda cantidad f�sica $O$ posible de ser medida es describida mediante un operador $\hat{O}$ que act�a de $H$ en $H$; este operador es llamado un observable.}\\

\quad Entonces, cada vez que una medici�n es realizada, el estado del sistema se ve alterado, en contraposici�n con la visi�n cl�sica; el principio de incertidumbre est� impl�cito en este postulado.

\item{\it Los �nicos posibles resultados de una medici�n de la cantidad $O$, corresponden a alguno de los autovalores del observable $\hat{O}$}.\\

\quad Es de gran importancia entonces, el conocer la descomposici�n espectral del observable asociado a la cantidad deseada de medir.
\item {\it Si la cantidad f�sica $O$ es medida en un sistema en un estado normalizado $\ket{\psi}$, la probabilidad $\mathbb{P}(a_n)$ de obtener el autovalor no degenerado $o_n$ del correspondiente observable $\hat{O}$ es:
$$\mathbb{P}(o_n)=\vert\braket{u_n}{\psi}\vert^2,$$

donde $\ket{u_n}$ es un autovector normalizado de $\hat{O}$ asociado con el autovalor $o_n$.}

\quad No profundizaremos en los dem�s casos, como espectros continuos, degenerados, finitos, etc. El lector interesado puede encontrar m�s detalles en \cite{cohen} p�g. 217 en adelante.

\item {\it Si la medici�n de la cantidad f�sica $O$ en el sistema que se encuentra en el estado $\ket{\psi}$ retorna como resultado $o_n$, el estado del sistema inmediatamente despu�s de la medici�n es la proyecci�n normalizada, $$\frac{P_n\ket{\psi}}{\sqrt{\bra{\psi}P_n\ket{\psi}}},$$ de $\ket{\psi}$ sobre el subespacio propio asociado con $o_n$.}\\

\quad Este postulado caus� inquietudes entre los progenitores de esta teor�a, siendo Einstein uno de los principales detractores dado que da cabida a la llamada ``acci�n a distancia''. El lector interesado puede leer acerca de la paradoja EPR en \cite{griffiths}, p�g. 421-423.

\item {\it La evoluci�n temporal del estado $\ket{\psi(t)}$ est� dada por la ecuaci�n de Sch\"odinger:
\begin{equation}
 i\hbar \frac{d}{dt}\ket{\psi(t)}=H(t)\ket{\psi(t)},\label{ecschooperadores}
\end{equation}


donde $H(t)$ es el observable asociado al total de energ�a del sistema.}
\end{enumerate}
\quad Como comentario final, mencionamos que la constante de Planck $\hbar$ juega un importante rol en el l�mite entre la teor�a Cu�ntica y la Cl�sica, siendo su valor ``peque\~no'' el responsable que los efectos cu�nticos sean despreciables a escala macrosc�pica, para una visi�n no formal de este hecho recomendamos la lectura de \cite{maravillas}.

\subsection{Campo electromagn�tico cuantizado}
\quad En esta secci�n presentaremos la descripci�n can�nica del campo electromagn�tico\footnote{Diremos ``campo electromagn�tico a pesar que en las expresiones utilizaremos su descripci�n en los dos campos: $\mathbf{E}$, el campo el�ctrico y $\mathbf{B}$, el campo magn�tico.}, sin presentar el desarrollo t�pico de esta cuatizaci�n, el lector interesado puede revisar \cite{dutra} secci�n 2.4.1.\\

\quad Tras solucionar las ecuaciones de Maxwell para una cavidad conductora perfecta, se encuentra que el campo electromagn�tico puede ser descrito como combinaci�n lineal de soluciones bases, llamadas \emph{modos} del campo electromagn�tico. Estos modos pueden ser descritos por dos n�meros $n,\,s$, que representan la polarizaci�n de la onda correspondiente. As�, el campo electromagn�tico al interior de la cavidad puede ser escrito como\footnote{El texto entre negritas indica que la variable es un vector.}:

\begin{eqnarray}
 \mathbf{E}&=&\sum_{n,s}\mathbf{\chi_{n,s}}e_{n,s}\\
\mathbf{B}&=&\sum_{n,s}\frac{1}{k_n}\nabla\wedge\mathbf{\chi_{n,s}}b_{n,s},
\end{eqnarray}

donde la evoluci�n de los coeficientes $e_n,b_n$ est� acoplada por la constante $k_n$ que representa la frecuencia del modo, este est� dada por las ecuaciones:
\begin{eqnarray}
\dot{b}_{n,s}&=&-ck_ne_{n,s}\label{coefmod1}\\
\dot{e}_{n,s}&=&-ck_nb_{n,s}\label{coefmod2}
\end{eqnarray}
o bien:
\begin{eqnarray}
 \ddot{e}_{n,s}&+&c^2k_n^2e_{n,s}=0\\
\ddot{b}_{n,s}&+&c^2k_n^2b_{n,s}=0,
\end{eqnarray}

donde $c$ es la velocidad de la luz. As�, el campo electromagn�tico puede ser descrito por la evoluci�n de una serie de osciladores arm�nicos acoplados. Ahora, la energ�a total almacenada en el campo electromagn�tico puede ser calculada mediante (v�ase \cite{electro} secci�n 8.1.2):
\begin{equation}
 \varepsilon=\frac{1}{8\pi}\int d^3r \left(\mathbf{E}^2+\mathbf{B}^2\right),\label{energia1}
\end{equation}

el primer paso de la cuantizaci�n, es definir las nuevas variables:
\begin{eqnarray}
 q_{n,s}&=&\frac{\gamma}{ck_n}\,b_{n,s}\\
p_{n,s}&=&-\gamma \,e_{n,s},
\end{eqnarray}
donde $\gamma$ es un constante a determinar. De esta forma las ecuaciones (\ref{coefmod1}) a (\ref{coefmod2}) adquieren la forma de las ecuaciones de Hamilton para un conjunto de osciladores no acoplados:
\begin{eqnarray}
 \dot{q}_{n,s}&=&p_{n,s}\\
\dot{p}_{n,s}&=&-c^2\,k_n^2\,q_{n,s},
\end{eqnarray}

cuyo Hamiltoniano es:\begin{equation}
H=\frac{1}{2}\sum_{n,s}\left\{p^2_{n,s}+c^2\,k_n^2\,q_{n,s}^2\right\}.\label{hamcampo} 
\end{equation}

as�, (\ref{energia1}) puede ser escrita como:\begin{equation}
 \varepsilon=\frac{1}{8\pi}\sum_{n,s}\left\{\frac{p^2_{n,s}}{\gamma^2}+\frac{c^2k_n^2}{ \gamma^2}q^2_{n,s}\right\},\label{energia2}
\end{equation}

y escogemos $\gamma=1/\sqrt{4\pi}$ para que el Hamiltoniano (\ref{hamcampo}) sea la energ\'ia total (\ref{energia2}).\\

\quad El siguiente paso en la cuantizaci�n, consiste en reemplazar los corchetes de Poisson de las variables $q_{n,s}$ y $p_{n,s}$ por los conmutadores cu�nticos (v�ase \cite{dutra} secci�n 2.1). En este caso, los conmutadores de los operadores correspondientes son:
$$[\hat{q}_{n,s},\hat{p}_{n',s'}]=i\hbar\delta_{n,n'}\delta_{s,s'}$$

\quad Ahora, recordando lo visto en la secci�n (\ref{osciladorarmonico}) sobre el oscilador arm�nico, definimos los operadores de subida y bajada:
\begin{eqnarray}
\hat{a}_{n,s} &=&(ck_n\hat{q}_{n,s}+i\hat{p}_{n,s})\frac{1}{\sqrt{2\hbar ck_n}}\\
\hat{a}^\dag_{n,s} &=&(ck_n\hat{q}_{n,s}-i\hat{p}_{n,s})\frac{1}{\sqrt{2\hbar ck_n}},
\end{eqnarray}

as�, el campo electromagn�tico cuantizado puede ser escrito de la siguiente forma:
\begin{eqnarray}
 \mathbf{\hat{E}}=i\sum_{n,s}\sqrt{2\pi\hbar c k_n}(a_{n,s}-a_{n,s}^\dag)\mathbf{\chi_{n,s}}\\
\mathbf{\hat{B}}=\sum_{n,s}\sqrt{\frac{2\pi\hbar c} {k_n}}(a_{n,s}+a_{n,s}^\dag)\nabla\wedge\mathbf{\chi_{n,s}}\\
\end{eqnarray}

y el Hamiltoniano:\begin{equation}
 \hat{H}=\sum_{n,s}{\hbar c k_n}\left(a_{n,s}^\dag a_{n,s}+\frac{1}{2}\right).\label{hamilelectrocuanti}
\end{equation}

\quad Las mismas propiedades estudiadas en la secci�n (\ref{osciladorarmonico}) son ahora v�lidas, s�lo se agrega la complicaci�n de un mayor n�mero de osciladores arm�nicos involucrados, dependiendo de la frecuencia ($k_n$) y de la polarizaci�n del modo $n,s$. As�, la energ�a del campo electromagn�tico ha sido cuantizada y a cada una de estas \emph{unidades de energ�a} le llamaremos \emph{fot�n}.

 \subsection{Absorci�n y emisi�n de radiaci�n. Emisi�n espont�nea.}\label{absorcion}
 
 \quad Tal como hemos discretizado el campo electromagn�tico, resultando este descrito mediante la combinaci�n lineal de una serie de osciladores arm�nicos cu�nticos, el comportamiento de un electr�n en un �tomo, puede ser tambi�n descrita por operadores de subida y bajada correspondientes, junto con un Hamiltoniano asociado, que entrega la energ�a total del sistema dependiendo del estado particular. Ahora, ambas descripciones tienen un punto de uni�n. Ambos sistemas, ''�tomo`` y ''campo electromagn�tico`` pueden interactuar entre s� mediante el intercambio de energ�a. Digamos por ejemplo, que un �tomo se encuentra en el nivel fundamental de energ�a $\psi_a$, si este �tomo es puesto en interacci�n con un modo del campo electromagn�tico, cuya frecuencia es $\omega$, la probabilidad que el �tomo pase a un nuevo nivel de energ�a, m�s alto $\psi_b$, tal que la diferencia de energ�a entre ambos niveles es $E_b-E_a=\hbar \omega_0$, es proporcional al cuadrado de la amplitud del modo del campo interactuante e inversamente proporcional al cuadrado de la diferencia de ambas frecuencias, $(\omega_0-\omega)^2$. De ocurrir esta ''subida`` en el nivel energ�tico del �tomo, el modo del campo electromagn�tico pierde exactamente la energ�a ganada por el �tomo: decimos entonces que el �tomo a \emph{absorvido} un fot�n. El fen�meno contrario tambi�n es posible, el �tomo, al estar en un nivel de energ�a $\psi_b$ puede decaer a otro nivel $\psi_a$ al ser perturbado por un modo del campo electromagn�tico; la energ�a perdida por el �tomo en esta interacci�n, es ganada por un modo del campo electromagn�tico cuya \emph{unidad de energ�a} (fot�n) coincida con la energ�a perdida por el �tomo, a este fen�meno se le conoce como \emph{emisi�n estimulada}. Finalmente, como hemos visto, el campo electromagn�tico, al ser descrito por osciladores arm�nicos cu�nticos, posee un m�nimo de energ�a no nulo, esta concecuencia, puramente cu�ntica da cavida a las llamadas \emph{fluctuaciones del vac�o} que pueden interactuar con un nivel inestable del �tomo y lograr hacerlo decaer, a este fen�meno se le conoce como \emph{emisi�n espont�nea}. Para mayor informaci�n sobre estos fen�menos el lector puede revisar \cite{griffiths} secci�n 9.9.2 o \cite{dutra} cap�tulo 6.
\Pag{Modelo Estudiado}
\sectionm{Modelo Estudiado}
\subsection{El I\'on $Ca^{+}$}\label{ionca}
 \quad A grandes rasgos, uno de los principales resultados de la Mec\'anica Cu\'antica, aplicada a la modelaci\'on de \'atomos, es que los electrones se distribuyen al rededor del n\'ucleo formando nubes electr\'onicas organizadas por capas o casquetes, cada una m\'as alejada al n\'ucleo que la otra. En el caso del Calcio, tal como aparece en la tabla peri\'odica, estas tres primeras capaz se encuentran completas y dos electrones m\'as, se encuentran en la cuarta capa. Sin embargo, estos \'ultimos electrones est\'an ligados d\'ebilmente, lo que significa que la energ\'ia necesaria para ionizar este \'atomo es baja. Por esto \'ultimo, en la naturaleza, el calcio se encuentra ionizado, esto se denota $Ca^+$ significando que uno de los electrones de la \'ultima capa ha sido perdido, habiendo emitido un fot\'on. Por poseer s\'olo un electr\'on libre en la \'ultima capa, se dice que el $Ca^+$ tiene forma hidrogenoide, esto es, su comportamiento es ``similar'' al del Hidr\'ogeno, que tambi\'en posee s\'olo un electr\'on libre.\\

\quad En este trabajo, los niveles de energ\'ia relevantes del $Ca^+$ ser\'an cinco, denotando por $\ket{0}$ el nivel fundamental, o de m\'inima energ\'ia, $\ket{1}$ el siguiente nivel, hasta el nivel $\ket{4}$, el de mayor energ\'ia. Las correspondencias formales con los niveles at\'omicos est\'andares es la siguiente:
\begin{eqnarray*}
\ket{0}&\rightarrow& 4^2S_{1/2}\\
\ket{1}&\rightarrow& 3^2D_{3/2}\\
\ket{2}&\rightarrow& 3^2D_{5/2}\\
\ket{3}&\rightarrow& 4^2P_{1/2}\\
\ket{4}&\rightarrow& 4^2P_{1/2}
\end{eqnarray*}

\quad Para mayor informaci\'on acerca de la nomenclatura utilizada para describir los niveles at\'omicos el lector puede consultar \cite{eisberg} secci\'on 9.7.\\

\quad La idea de absorci\'on de radiaci\'on, presentada en (\ref{absorcion}) puede ser utilizada para llevar un \'atomo a un nivel de mayor energ\'ia si as\'i se desea, la fuente de luz utilizada para este fin es un l\'aser; se dice que el l\'aser es utilizado para \emph{bombear} el \'atomo del nivel energ\'etico menor al mayor. El l\'aser $1$ lleva al \'atomo del nivel $\ket{1}$ al $\ket{3}$ y el l\'aser $2$ del $\ket{0}$ al $\ket{4}$.\\
\begin{figure}[h]
\centering
\begin{minipage}{0.52 \linewidth}
\centering
  \includegraphics[scale=0.4]{cap3/nivelesmodelo_transiciones.eps}\caption{Esquematizaci\'on de los decaimientos de los niveles at\'omicos.}\label{fig:decaimientos}
\end{minipage}
\begin{minipage}{0.52 \linewidth}
\centering
  \includegraphics[scale=0.4]{cap3/nivelesmodelo_laseres.eps}\caption{Esquematizaci\'on de las transiciones producto de los l\'aseres.}\label{fig:laseres}
\end{minipage}
\end{figure}
\clearpage

\quad Adem\'as, en nuestro modelo, el \'atomo se encuentra acoplado a dos modos del campo electromagn\'etico al interior de la cavidad, uno de estos modos, el modo $a$, se acopla a las transiciones energ\'eticas del \'atomo entre $\ket{3}$ y $\ket{0}$, el otro, el modo $b$, se acopla a las transiciones entre $\ket{4}$ y $\ket{2}$. Las interacciones producto de los decaimientos espont\'aneos, esquematizadas en la Figura \ref{fig:decaimientos}, y las producto de los l\'aseres en la Figura \ref{fig:laseres}, las resumimos a continuaci\'on:

\begin{itemize}
 \item[$\ket{0}$: ] Desde este nivel, el \'atomo es llevado al nivel $\ket{4}$ por el l\'aser $2$. Por tratarse del nivel fundamental, no existen posibles decaimientos. Existe tambi\'en la probabilidad de reabsorver un fot\'on presente en el modo $a$ y que el \'atomo regrese al nivel $\ket{1}$, pero esta probabilidad es despreciable frente a la mencionada anteriormente.

\item[$\ket{1}$: ] Como ya dijimos, desde este nivel el \'atomo es llevado al nivel $\ket{3}$ por el l\'aser $1$. No es posible que el \'atomo decaiga al nivel fundamental, producto de las reglas de selecci\'on (v\'ease \cite{griffiths} secci\'on 9.3.3).

\item[$\ket{2}$: ] Este estado es absorvente, en el sentido que por reglas de selecci\'on, el \'atomo no puede caer a niveles inferiores.

\item[$\ket{3}$: ] Desde este estado, el \'atomo decaer\'a al nivel fundamental, emitiendo un fot\'on en el modo $a$, siendo casi nula la probabilidad que decaiga de regreso al nivel $\ket{1}$.

\item[$\ket{4}$: ] Desde este estado, el \'atomo decaer\'a al nivel $\ket{2}$, emitiendo un fot\'on en el modo $b$, no pudiendo decaer a otros estados por las reglas de selecci\'on, salvo al fundamental con probabilidad despreciable, pero de todas maneras retorna al nivel $\ket{4}$ (esto lleva a definir m\'as adelante la \emph{tasa de transici\'on}).
\end{itemize}

\subsection{Modelo Matem\'atico}\label{sec:modelo_matematico}
\subsubsection{$\mathcal{H}_{at}$, $\mathcal{H}_a$, $\mathcal{H}_b$ y el espacio compuesto $\mathcal{H}$}\label{HatHaHb}

\quad El espacio de Hilbert al cual pertenecen los posibles estados del \'atomo, es de dimensi\'on infinita, sin embargo, s\'olo cinco de estos estados se encuentran involucrados en la evoluci\'on del sistema a estudiar, luego podemos truncar la dimensi\'on del Hilbert asociado al \'atomo, que denotaremos por $\mathcal{H}_{at}$. A saber, s\'olo poseer\'a cinco elementos en su base, es decir $\mathcal{H}_{at}=\left(\mathbb{C}^5,+,\cdot\right)$, tomando como base del espacio, la base can\'onica $\{e_i\}:\;i=1\ldots 5$, con, por ejemplo:

$$
e_1=\left(
\begin{array}{cccc}
0&0&0&0
\end{array}
\right)^t,
$$

y haciendo la correspondencia para los niveles at\'omicos de la secci\'on anterior\footnote{El sub\'indice ``at'' indica que el elemento pertenece al espacio asociado al \'atomo.}:

$$\ket{i}_{at}\equiv e_i,\hspace{2cm}i=1\ldots 5.$$

\quad Al igual que para el caso at\'omico, los espacios de Hilbert asociados a los modos $a$ y $b$ del campo electromagn\'etico acoplados al \'atomo, que denotarmos por $\mathcal{H}_a$ y $\mathcal{H}_b$ resp. ser\'an truncados en su dimensi\'on. En el caso en el cu\'al no existe ruido t\'ermico, este truncamiento se puede tomar como el m\'inimo de fotones posibles de ser emitidos por el \'atomo, es decir, dos. Cuando incluyamos el ruido t\'ermico en la secci\'on (\ref{ruidotermico}), este nivel de truncamiento ser\'a aumentado dado que el n\'umero de fotonos posibles en cada modo aumentar\'a.\\

\quad Para el caso sin ruido t\'ermico seleccionamos entonces $\mathcal{H}_a=\mathcal{H}_b=\left(\mathbb{C}^3,+,\cdot\right)$, dado que se requiere que en la base del espacio est\'en los tres niveles posibles, desde $0$ a $2$ fotones. La base elegida en cada Hilbert ser\'a nuevamente la can\'onica (esta vez en $\mathbb{C}^3$) y la correspondencia para $\mathcal{H}_a$ con los kets del modo $a$ ser\'a:

$$\ket{n}_a\equiv e_n,\hspace{2cm} n=1\ldots 3.$$

la correspondencia para $\mathcal{H}_b$ es an\'aloga.\\

\quad Finalmente, el espacio de Hilbert $\mathcal{H}$ que representa el sistema total, es el resultado de la composici\'on, mediante el producto tensorial, de estos espacios menores: $$\mathcal{H}=\mathcal{H}_{at}\otimes \mathcal{H}_{a} \otimes \mathcal{H}_{b}.$$

\subsubsection{Variables y constantes involucradas}\label{sec:variables_involucradas}
\begin{tabular*}{\textwidth}{ll}
\hline\multirow{2}{*}{\large{Constante}}&\multirow{2}{*}{\large{Descripci\'on}}\\ &\\ \hline\\\vspace*{0.5cm}
$\kappa_a$, $\kappa_b$ & Factor de p\'erdida de energ\'ia en la cavidad asociado al modo $a$, $b$ respectivamente.\\
&V\'ease \cite{dutra}.\\\\

$g_a$, $g_b$ & Constantes de acoplamiento entre los modos $a$ y $b$ resp. y los niveles at\'omicos asociados.\\\\

$\delta_1$, $\delta_2$ & Par\'ametros de perturbaci\'on a segundo orden. V\'ease \cite{robert}.\\\\ 

$\Omega_1(t)$, $\Omega_2(t)$ & Intensidad de los l\'aseres que bombean electrones en los niveles at\'omicos descritos en la\\& secci\'on (\ref{ionca}).\\\\

$\vartriangle_1$, $\vartriangle_2$ & Desfase o \emph{detuning} entre las frecuencias de los niveles at\'omicos y las frecuencias efecti- \\&vamente alcanzadas por el electr\'on efecto del l\'aser producto del \emph{efecto Stark}. V\'ease \cite{single-photon}.\\\\

$v_a$, $v_b$ & Frecuencias efectiva de interacci\'on entre los niveles at\'omicos y los modos $a$ y $b$ resp.\\\\

$v_1$, $v_2$ & Frecuencias de los l\'aseres $1$ y $2$.\\\\
$\Gamma_{ij}$ & Tasa de transici\'on del nivel $j$ al nivel $i$.\\\\
$\nmod$ & N\'umero de modos del campo electromagn\'etico acoplados al \'atomo. $\nmod=2$ en este \\&trabajo.\\\\
$\dimAt$ & N\'umero de niveles at\'omicos relevantes en el modelo. $\dimAt=5$ en este trabajo.\\\\
$\dimA,\dimB$ & Dimensi\'on del espacio $\mathcal{H}_a$ y $\mathcal{H}_b$ asociados al los modos del campo electromagn\'etico\\ &  $a$, $b$ respectivamente. Esto corresponde al n\'umero de fotones permitidos m\'as uno.\\
$\dimT$ & Dimensi\'on del espacio mayor $\mathcal{H}$.
\end{tabular*} 

\subsubsection{Operadores involucrados}\label{sec:operadores_involucrados}

\quad A continuaci\'on describimos los operadores a utilizar (De aqu\'i en adelante, utilizaremos $\ket{i}$ para denotar exclusivamente el ket que representa el nivel $i$ del \'atomo, en otro caso se utilizar\'a e sub\'indice correspondiente):

\begin{tabular*}{\textwidth}{ll}
\hline\multirow{2}{*}{\large{Operador}}&\multirow{2}{*}{\large{Descripci\'on}}\\ &\\ \hline\\\vspace*{0.5cm}

$I_{at}$ & Operador identidad en el espacio $\mathcal{H}_{at}$.\vspace*{0.5cm} \\\hline\\\vspace*{0.5cm}

$I_a$  & Operador identidad en el espacio $\mathcal{H}_{a}$. An\'alogo para $I_{b}$.\vspace*{0.5cm}\\\hline\\\vspace*{0.5cm}

$\hat{a}$& \begin{minipage}{14cm}
      Act\'ua seg\'un $\mathcal{H}_a\rightarrow \mathcal{H}_a$. Operador de creaci\'on del modo $a$, llamado as\'i puesto que opera \emph{generando} un fot\'on en el modo $a$. Este es equivalente al operador de subida descrito en la secci\'on (\ref{osciladorarmonico}). Sin embargo, en secciones siguientes utilizaremos esta misma notaci\'on (sin ambig\"uedad pues el significado ser\'a le\'ido del contexto) para denotar el operador que act\'ua seg\'un $\mathcal{H}\rightarrow\mathcal{H}$ y definido por:

\begin{equation}
 \hat{a}\equiv I_{at}\otimes \hat{a} \otimes I_b,\label{ec:operador_creacion_a}
\end{equation}

El lector no debe confundirse, al lado izquierdo hemos definido el nuevo operador en funci\'on del operador descrito en el p\'arrafo anterior. An\'alogo para $\hat{b}$.
     \end{minipage}\vspace*{0.5cm}\\\hline\\\vspace*{0.5cm}

$\hat{a}_{ij}$ & \begin{minipage}{14cm} Act\'ua seg\'un $\mathcal{H}\rightarrow\mathcal{H}$. Operador de transiciones at\'omicas definido por: 

\begin{equation}\hat{a}_{ij}\equiv \ket{i}\bra{j}\otimes I_a \otimes I_b.\label{ec:operador_transicion_H}
\end{equation} \end{minipage}\vspace*{0.5cm}\\\hline\\\vspace*{0.5cm}
\end{tabular*} 

En la siguiente secci\'on describiremos otros operadores a utilizar pero de mayor importancia conceptual.

\subsubsection{Ecuaci\'on maestra}\label{ecuacionmaestra}
\quad Por \emph{Ecuaci\'on maestra} (o \emph{Ecuaci\'on de evoluci\'on}) entenderemos la ecuaci\'on diferencial que describe el comportamiento en el tiempo del estado del sistema o alg\'un equivalente. Siendo $\psi(t)\in \mathcal{H}$ el estado del sistema en un instante $t$, definimos el operador de densidad $\rho$ mediante\footnote{Por simplicidad, escribiremos simplemente $\rho$, omitiendo el s\'imbolo \textasciicircum }:
\begin{equation}
 \rho\equiv\ket{\psi}\bra{\psi},\label{operadordensidad}
\end{equation}

donde hemos omitido la dependencia temporal por simplicidad de notaci\'on. De esta forma la ecuaci\'on (\ref{ecschooperadores}) puede ser escrita como (v\'ease \cite{cohen} complemento $E_{III}$): 
\begin{equation}
\frac{d}{dt}\rho(t)= \frac{1}{i\hbar}\left[H(t),\rho(t)\right].\label{ecschorho}
\end{equation}

\quad Ade\'mas se tiene que, si $\hat{O}$ es un observable, asociado a la cantidad f\'isica $o$, entonces el valor esperado $\bar{o}$ de $o$ en un tiempo $t$ est\'a dado por:

\begin{equation}
 \bar{o}=tr\left(\hat{O}\rho(t)\right).\label{ec:traza_opRho}
\end{equation}


\quad Ahora bien, en nuestro modelo, consideraremos la visi\'on m\'as realista en la cual existen p\'erdidas de energ\'ia en el sistema, que corresponden a p\'erdidas de energ\'ia producto de imperfecciones en los espejos de la cavidad y en emisiones espont\'aneas (v\'ease (\ref{absorcion})). Estas p\'erdidas, se describen mediante el operador \emph{Limbladiano} que debe ser agregado a la ecuaci\'on (\ref{ecschorho}), as\'i, la forma de la ecuaci\'on de evoluci\'on en nuestro trabajo ser\'a:
\begin{equation}
\frac{d}{dt}\rho(t)= \frac{1}{i\hbar}\left[H(t),\rho(t)\right]+\mathcal{L}(\rho).\label{ecschorholimb}
\end{equation}

\quad En un problema en particular, debemos entonces determinar primero, cu\'al es el espacio $\mathcal{H}$ con el cual modelaremos el problema f\'isico. Para luego determinar los operadores $H$ y $\mathcal{L}$, Hamiltoniano y Limbladiano respectivamente. En la secci\'on (\ref{HatHaHb}), ya hemos resuelto la primera pregunta, aunque un breve ajuste deber\'a ser hecho cuando agreguemos ruido t\'ermico en la siguiente secci\'on.\\

\quad En cuanto al Hamiltoniano del sistema, este puede ser descrito como la suma de los Hamiltonianos de los sistemas componentes por separado (v\'ease (\ref{HatHaHb})), suma que denotaremos por $H_0$ y el Hamiltoniano que describe la interacci\'on entre los sistemas, en particular, la interacci\'on entre cada modo y el \'atomo, adem\'as de la interacci\'on entre los l\'aseres y los niveles at\'omicos relacionados. Este \'ultimo Hamiltoniano ser\'a denotado por $H_I$:
\begin{eqnarray}
 H=H_0+H_I.
\end{eqnarray}

\quad Ahora, la energ\'ia presente (o \emph{almacenada}) en el modo $a$ del campo electromagn\'etico corresponde al \emph{n\'umero promedio de fotones presente en el modo $a$}, $n_a$, ponderado por la frecuencia particular del modo $a$, $\omega_a$. El operador asociado a la cantidad $n_a$, denotado por $\hat{n}_a$ corresponde a la composici\'on de su operador creaci\'n con su operador de aniquilaci\'on asociados, es decir $\hat{n}_a=\hat{a}^\dag \hat{a}$. As\'i, la energ\'ia presente en el modo $a$ es calculada utilizando el operador\footnote{En este expresi\'on, y en otras durante el presente trabajo, omitimos $\hbar$ (en este caso en la expresi\'on $\hbar\omega_a\hat{a}^\dag \hat{a}$) ya que consideraremos esta constante igual a la unidad.}:

$$\omega_a\,\hat{a}^\dag \hat{a}.$$

\quad An\'alogamente, el operador: $$\omega_b\,\hat{b}^\dag \hat{b},$$

es utilizado en el c\'alculo de la energ\'ia presente en el modo $b$. Finalmente, la energ\'ia presente en el \'atomo (considerando s\'olo la energ\'ia presente en los niveles considerados por el modelo) corresponde a la suma de los proyectores de cada nivel $i$, ponderado por su frecuencia natural $\omega_i$ de oscilaci\'on:
\begin{equation}
 H_0=\omega_a \hat{a}^\dag \hat{a} + \omega_b \hat{b}^\dag \hat{b}+\sum_{i=0}^4 \omega_i \ket{i}\bra{i}.\label{ec:H0}
\end{equation}

\quad Ahora, la el Hamiltoniano de interacci\'on $H_I$ es escrito como\footnote{La notaci\'on ``h.c'' indica que deben sumarse los conjugados de los sumandos escritos previamente.}:

\begin{equation}
 H_I=\Omega_1(t)\,\ket{3}\bra{1}\,e^{-iv_a t}+\Omega_2(t)\,\ket{4}\bra{0}\,e^{-iv_bt}+g_a\ket{3}\bra{0}\hat{a}+g_b\ket{4}\bra{2}\hat{b}+\mbox{h.c}.\label{ec:HI}
\end{equation}

\quad Los primeros dos t\'erminos indican la interacci\'on entre los l\'aseres y los niveles at\'omicos asociados, los dos \'ultimos indican el acoplamiento entre los modos del campo y los niveles asociados. Finalmente, el Limbladiano para el modelo sin ruido t\'ermico corresponde a:
\begin{equation}
\begin{array}{lcl}
\mathcal{L}(\rho)&=&\frac{\Gamma_{04}}{2}(2\,a_{04}\rho a^\dag_{04}-a^\dag_{04} a_{04}\rho-\rho\,a^\dag_{04}a_{04})+\\
&+&\frac{\Gamma_{03}}{2}(2\,a_{04}\rho a^\dag_{03}-a^\dag_{03} a_{03}\rho-\rho\,a^\dag_{03}a_{03})+\\
&+&\frac{\Gamma_{13}}{2}(2\,a_{13}\rho a^\dag_{13}-a^\dag_{13} a_{13}\rho-\rho\,a^\dag_{13}a_{13})+\\
&+&\frac{\Gamma_{24}}{2}(2\,a_{24}\rho a^\dag_{24}-a^\dag_{24} a_{24}\rho-\rho\,a^\dag_{24}a_{24})+\\
&+&\frac{\Gamma_{14}}{2}(2\,a_{14}\rho a^\dag_{14}-a^\dag_{14} a_{14}\rho-\rho\,a^\dag_{14}a_{14})+\\

&+&\kappa_a(2\,\hat{a}\rho \hat{a}^\dag-\hat{a}^\dag\hat{a}\rho - \rho\hat{a}^\dag\hat{a} )+\\
&+&\kappa_b(2\,\hat{b}\rho \hat{b}^\dag-\hat{b}^\dag\hat{b}\rho - \rho\hat{b}^\dag\hat{b} ).

\end{array}\label{limbladiano}
\end{equation}

\quad Donde $\kappa_a$ y $\kappa_b$ son las tasas de p\'erdida de la cavidad para cada modo, que supondremos iguales en este trabajo. Adem\'as los $\Gamma_{ij}$ son las tasas de transici\'on entre $\ket{i}$ y $\ket{j}$, hemos despreciado las transiciones entre el nivel $\ket{1}$ y el $\ket{0}$ y entre $\ket{2}$ y $\ket{0}$ puesto que las tasas asociadas son de ocho \'ordenes de magnitud diferentes a las dem\'as tasas y su efecto queda fuera de la escala de tiempo considerada en este trabajo (v\'ease \cite{single-photon-walther}). Finalmente, el estado inicial del sistema se encuentra dado por las siguientes condiciones:

\begin{itemize}
 \item En cada modo, inicialmente no hay fotones presentes. Por tanto el modo $a$ empieza en el estado $\ket{0}_a$, al igual que el modo $b$ empieza en el estado $\ket{0}_b$.
\item El \'atomo empieza en el nivel de energ\'ia descrito por $\ket{1}$.
\end{itemize}

Por tanto, el operador de densidad inicial $\rho(0)=\ket{\psi(0)}\bra{\psi(0)}$ se puede considerando:

\begin{equation}
 \ket{\psi(0)}=\ket{1}\otimes \ket{0}_a \otimes \ket{0}_b.\label{rho0}
\end{equation}

\quad As\'i, el conjunto dado por (\ref{ecschorholimb}), (\ref{ec:H0}), (\ref{ec:HI}), (\ref{limbladiano}) y (\ref{rho0}) describe el problema matem\'atico a resolver para analizar la evoluci\'on del sistema f\'isico descrito en (\ref{ionca}).

\subsubsection{Adici\'on de Ruido t\'ermico}\label{ruidotermico}
\quad Un caso f\'isico m\'as realista que el presentado en las secciones anteriores, es el que se obtiene al considerar que la cavidad se encuentra en un \emph{ba\~no t\'ermico}, esto es, no existe un vac\'io perfecto en la cavidad. El efecto de esta hip\'otesis, es estudiado en \cite{carmichael} secci\'on 1.3 para el caso de equilibro t\'ermico y su efecto en nuestro modelo ser\'a a\~nadir el t\'ermino:
\begin{equation}
 2\kappa\bar{n}(\hat{a}\rho\hat{a}^\dag+\hat{a}^\dag\rho \hat{a}-\hat{a}^\dag \hat{a}\rho-\rho \hat{a} \hat{a}^\dag),\label{ruido}
\end{equation}

al limbladiano (\ref{limbladiano}) por cada modo, $a$ y $b$, reemplazando $\hat{a}$ ($\hat{a}^\dag$) por el respectivo operador de aniquilaci\'on (creaci\'on). La constante $\bar{n}$ es el n\'umero de fotones promedio a temperatura $T$ para el modo involucrado.

% \Pag{Integraci\'on Ecuaci\'on Maestra}
% \sectionm{Integraci\'on Ecuaci\'on Maestra}
\subsection{Problema a resolver}\label{sec:problema_resolver}
\quad Recordemos que el problema a resolver, visto en la secci\'on (\ref{ecuacionmaestra}), es el encontrar $\rho\in \mathcal{C}^2_0\left([0,t_f],\mathbb{C}^n\right)$, y definido en (\ref{operadordensidad}), donde $n$ es la dimensi\'on total resultado del producto de las dimensiones de los espacios componentes vistos en (\ref{HatHaHb}) y $t_f=\frac{100}{\kappa}$, con $\kappa$ igual a la tasa de p\'erdida de energ\'ia de las cavidades, v\'ease \cite{single-photon-walther}. Siendo la ecuaci\'on a satisfacer:
\begin{eqnarray}
 \frac{d}{dt}\rho(t)&=&f\left(t,\rho(t)\right)\\
&=&-i\hbar\left[H(t),\rho(t)\right]+\mathcal{L}(\rho(t)),\label{ec_maestra_integracion}
\end{eqnarray}
con $H$ y $\mathcal{L}$ dados por (\ref{ec:H0}) + (\ref{ec:HI}) y (\ref{limbladiano}) respectivamente. Adem\'as $t_f$ es el tiempo final que define el intervalo sobre el cual se desea resolver. Finalmente se tiene la condici\'on inicial:
\begin{equation}
 \rho(0)=\ket{\psi(0)}\bra{\psi(0)},\hspace*{1.3cm}\ket{\psi(0)}=\ket{1}\otimes\ket{0}_a\otimes \ket{0}_b.
\end{equation}

\quad En este trabajo, se resolver\'a este problema mediante integraci\'on num\'erica utilizando el m\'etodo de Heun, sobre una malla temporal equiespaciada $t_i,$ $i=1\ldots N$, $N=20000$, $t_1=0\ldotp0s$ y $t_N=t_f$. El an\'alisis de la conveniencia de este m\'etodo por sobre otros para este problema, es realizado en \cite{gino}. Debemos destacar que la Continuidad y Diferenciabilidad de la Ecuaci\'on Maestra siguen siendo v\'alidas tras agregar el ruido t\'ermico, puesto que esto s\'olo a\~nade t\'erminos del tipo $\Vert C_m\Vert$ en la constante de acotaci\'on $M$ en el Cap\'itulo 6 de \cite{gino}, lo mismo ocurre en el an\'alisis del cumplimiento de la condici\'on de Lipschitz en la segunda variable de $f$.

\subsubsection{M\'etodo de integraci\'on de Heun}
\quad Este m\'etodo num\'erico de integraci\'on es del tipo \emph{predictor-corrector} de segundo orden. Como hip\'otesis para la utilizaci\'on del m\'etodo, supondremos que $f\in \mathcal{C}_0^2([0,t_f]\times\mathcal{H})$ de $\ref{ec_maestra_integracion}$ cumple la condici\'on de Lipschitz, esto es, $\exists L\in \mathbb{R}^+_0$ tal que:

$$\vert f(t,\rho)-f(t,\rho')\vert\leq L\Vert \rho-\rho'\Vert_\mathcal{H}$$

$\forall\, t\in[0,t_f],\; \forall\,\rho,\rho'\in\mathcal{H}.$

\quad Definiremos adem\'as la partici\'on $t_0=0<t_1<\ldots<t_N=t_f$ del intervalo $[0,t_f]$ que ser\'a utilizado por el m\'etodo. El m\'etodo de Huen, consiste en utilizar como predictor, el m\'etodo Runge-Kutta de segundo orden, y como corrector el m\'etodo del Trapecio. As\'i, el algoritmo de integraci\'on para nuestro problema resulta ser:

\begin{algorithm}[H]
\caption{M\'etodo de Heun}\label{algo:heun}
% \SetAlgoLined
\LinesNumbered
\KwIn{$f$, $Y_0$, $h$, $n$, $\{t_i\}_{i=0}^{n-1}$, $\epsilon$}
\KwOut{$\{Y_i\}_{i=0}^{n-1}$}
\Begin{\For{$i=0$ hasta $i=n-1$}
  {
  $Y_{i+1}^{(0)}=Y_i+hf(t_i,Y_i)$\;
  $Y_{i+1}^{(1)}=Y_i+\frac{h}{2}\left\{f(t_i,Y_i+f\left(t_{i+1},Y_{i+1}^{(0)})\right)\right\}$\;
  $error=errRel\left(Y_{i+1}^{(1)},Y_{i+1}^{(0)}\right)$\;
  $k=2$\;
  \While{$error>\epsilon$}{
    $Y_{i+1}^{(k)}=Y_i+\frac{h}{2}\left\{f\left(t_i,Y_i+f(t_{i+1},Y_{i+1}^{(k-1)})\right)\right\}$\;
    $error=errRel\left(Y_{i+1}^{(k)},Y_{i+1}^{(k-1)}\right)$\;
    $k=k+1$\;
  }
  $Y_{i+1}=Y_{i+1}^{(k)}$\;
}
}
\end{algorithm}

\quad Donde $Y_0$ es el punto inicial $\rho(0)$, $h$ es el paso de integraci\'on (v\'ease \cite{gino} Cap. 7), $n$ es el tama\~no de la malla temporal y $\epsilon$ es la tolerancia del m\'etodo corrector, en este trabajo $\epsilon=1E-6$.

\subsubsection{Truncamiento de dimensi\'on para $\mathcal{H}_a$ y $\mathcal{H}_b$} %modelo

%%SEGUNDA PARTE : MATEM�TICA-COMPUTACIONAL
\pagestyle{empty}%paginas en blanco
\begin{center}
  \part*{Problema Matem\'atico-Computacional}
\end{center}
%\include{intro3}
\Pag{Estudio computacional}
\sectionm{Estudio computacional}
\subsection{Identificaci\'on subproblemas y elementos importantes}

\subsubsection{Algoritmos generales}\label{sec:codigos_generales}
\quad Para comenzar el estudio computacional, escribiremos los algoritmos del programa realizado en FORTRAN y nos referiremos a sus secciones y subsecciones con la profundidad que sea necesaria.

\quad El primer Algoritmo, y por ende el m\'as gen\'erico del programa (al que llamaremos \emph{Main} o \emph{Principal}) es:

\begin{algorithm}[H]
 \caption{Main}\label{algo:main_completo}
\SetAlgoLined
\LinesNumbered
\KwIn{Par\'ametros}
\KwOut{Datos de C\'alculos}
\Begin{Carga de par\'ametros y definici\'on de variables\;
Inicializaci\'on de $\rho(0)$\;\label{linea:main_ini_rho}
Integraci\'on y registro\;\label{linea:main_int_regdatos}
}
\end{algorithm}

\quad Dada esta descripci\'on, la l\'inea o secci\'on de mayor importancia del programa \emph{main} ser\'a la \ref{linea:main_int_regdatos}, esto dado que los tiempos de c\'omputo de primera secci\'on es despreciable.\\

\quad El algoritmo para la integraci\'on y registro de datos es el mismo que el expresado en Algoritmo \ref{algo:heun}, pero esta vez a\~nadiendo el c\'alculo de las cantidades f\'isicas deseadas de obtener a partir de $\rho$ y el registro de \'estas. Esto lo describimos a continuaci\'on:

\begin{algorithm}[H]
\caption{Main: Integraci\'on y registro}\label{algo:main_int}
\LinesNumbered
\KwIn{$f$, $\rho_0$, $h$, $n$, $\{t_i\}_{i=0}^{n-1}$, $\epsilon$}
\KwOut{$\{Calculos(\rho_i)\}_{i=1}^{n-1}$}
\Begin{\For{$i=0$ hasta $i=n-1$}
  {
  $Calculos(\rho_i)$ y registro\;\label{linea:main_int_calculos}
  $\rho_{i+1}^{(0)}=\rho_i+hf(t_i,\rho_i)$\;
  $\rho_{i+1}^{(1)}=\rho_i+\frac{h}{2}\left\{f(t_i,\rho_i+f\left(t_{i+1},\rho_{i+1}^{(0)}\right)\right\}$\;\label{linea:main_int_f}
  $error=errRel\left(\rho_{i+1}^{(1)},\rho_{i+1}^{(0)}\right)$\; \label{linea:main_int_err}
  $k=2$\;
  \While{$error>\epsilon$}{
    $\rho_{i+1}^{(k)}=\rho_i+\frac{h}{2}\left\{f\left(t_i,\rho_i+f(t_{i+1},\rho_{i+1}^{(k-1)})\right)\right\}$\;
    $error=errRel\left(\rho_{i+1}^{(k)},\rho_{i+1}^{(k-1)}\right)$\;
    $k=k+1$\;
  }
  $\rho_{i+1}=\rho_{i+1}^{(k)}$\;
}
}
\end{algorithm}

\quad El m\'etodo escogido para la resoluci\'on de nuestro problema, se encuentra dentro de un tipo general de m\'etodos num\'ericos tales que, para resolver la tarea dada, calculan la derivada de una funci\'on de manera expl\'icita (que corresponde a evaluar $f$ en nuestro caso). Esta caracter\'istica ser\'a la de mayor coste computacional en nuestro trabajo y la gran mayor\'ia de las estrategias que sean estudiadas ser\'an desarrolladas en torno a aquella caracter\'istica. Por ende, del Algoritmo \ref{algo:main_int}, describiremos el algoritmo para las subrutinas $Calculos(\rho_i)$, $error$ (error relativo) y $f$ en los Algoritmos \ref{algo:main_int_calculos}, \ref{algo:main_int_err} y \ref{algo:main_int_f} respectivamente\footnote{Recordar que estamos considerando $\hbar=1$}.

\begin{algorithm}
 \caption{Integraci\'on y registro: $Calculos(\rho)$}\label{algo:main_int_calculos}
\LinesNumbered
\KwIn{$\rho$}
\KwOut{$\{\eta_i\}_{i=1}^{\dimAt}$}
\Begin{\For{$i=0$ hasta $\dimAt-1$}{
  $\eta_{i+1}=tr\left( \ket{i}\bra{i}\rho \right) $\;
  }
  \For{$i=1$ hasta $\nmod$}{
  $\eta_{i+\dimAt}=tr\left(a^\dag_i a_i\rho\right)$\;
  }
}
\end{algorithm}

\begin{algorithm}
 \caption{Integraci\'on y registro: $errRel$}\label{algo:main_int_err}
\LinesNumbered
\KwIn{$\rho,\;\rho'$}
\KwOut{$error$}
\Begin{$error=\frac{\Vert\rho-\rho'\Vert}{\Vert\rho\Vert}$\;\label{algo:main_int_err:linea:error}
}
\end{algorithm}

\begin{algorithm}
 \caption{Integraci\'on y registro: $f$}\label{algo:main_int_f}
\LinesNumbered
\KwIn{$t,\;\rho$}
\KwOut{$z=f(t,\rho)$}
\Begin{$h=H(t)$\;
$c=-i\left[h,\rho\right]$\;
$l=limbladiano(\rho)$\;
$z=c+l$\;
}
\end{algorithm}

\quad Estos algoritmos, que han sido escritos superficialmente, nos servir\'an de base en la siguiente secci\'on para el an\'alisis de las estrategias a seguir para una mayor eficiencia del programa Main.
\subsection{Estrategias de c\'omputo} %cálculo en paralelo de calcs(rho) al final, etc.
\quad Deseamos identificar, los algoritmos que m\'as trabajo computacional pueden significar, sin embargo, antes de realizar estudios al respecto, debemos tener cierta seguridad, de haber escrito algoritmos que ya poseen eficiencia de manera independiente para cada subproblema presente. Tras haber hecho esto, podemos trabajar en el subproblema que mayor costo tenga, luego el siguiente, etc.\\

\quad En vista de lo anterior, analizaremos primero los algoritmos generales escritos en la secci\'on anterior. Recordemos que en aquella secci\'on, se\~nalamos que es a partir del Algoritmo \ref{algo:main_int} que debemos concentrar nuestra atenci\'on.

\begin{description}

 \item[\emph{Algoritmo \ref{algo:main_int}} :] Una caracter\'istica importante que salta a la vista, tras compararlo con el Algoritmo \ref{algo:heun}, algoritmo desde el cual el primero fue derivado. En el segundo, donde se describe la integraci\'on a realizar en forma general, la salida del programa es $\{Y_i\}_{i=0}^{n-1}$, que en el contexto de nuestro trabajo, es equivalente a escribir $\{\rho_i\}_{i=0}^{n-1}$. Sin embargo, en el Algoritmo \ref{algo:main_int} la salida es $\{Calculos(\rho_i)\}_{i=0}^{n-1}$, dado que no son, escencialmente las matrices $\rho_i$ las que nos interesan, si no las cantidades f\'isicas que son posibles de calcular seg\'un la ecuaci\'on (\ref{ec:traza_opRho}). Esto lleva a dos estrategias inmediatas de c\'omputo:

\begin{enumerate}
 \item Calcular en cada paso $k$ la matriz de densidad $\rho_{k+1}$, luego realizar y registrar $Calculos(\rho_{k+1})$, finalmente hacer $\rho_k=\rho_{k+1}$, perdiendo los datos de la matriz anterior y yendo al siguiente paso. De esta forma, no es necesario almacenar todas las matrices de densidad, ahorrando cinco \'ordenes en cantidad de memoria, puesto que el n\'umero de iteraciones a realizar es $2E5$.
\item Calcular en cada paso $k$ la matriz de densidad $\rho_{k+1}$ en una nueva variable. Tras finalizar la integraci\'on. Realizar en paralelo cada $Calculo(\rho_k)$, de esta forma aprovechamos todo el potencial del n\'umero de hilos de procesamiento que se tengan a disposici\'on, pero a costo de un gran uso en la memoria disponible. 
\end{enumerate}

\item[\emph{Algoritmos \ref{algo:main_int_calculos} y \ref{algo:main_int_f}} :] Lo importante a destacar en esto casos, es que \'estos hacen uso de los operadores descritos en la secci\'on (\ref{sec:operadores_involucrados}). La forma de \'estos operadores, y el m\'etodo a utilizar para calcular expresiones en los que \'estos se encuentren involucrados, tedr\'a gran relevancia en vistas de lograr algoritmos eficiente. Estudiaremos esto en la secci\'on (\ref{sec:cal_operadores_involucrados}).

\item[\emph{Algoritmo \ref{algo:main_int_err}} :] Se destaca que la norma utilizada es la de Frobenius, pero m\'as a\'un, los operadores involucrados en este c\'alculo son siempre hermitianos (diferencia de dos matrices de densidad, hermitianas). Ahora bien, sea $\rho$ la matriz de coeficientes $\rho_{ij}\in \mathbb{C},\,i,j=1\ldots n$ que representa (en la base can\'onica) al operador $\hat{\rho}:\,\mathbb{C}^n\rightarrow \mathbb{C}^n$ hermitiano arbitrario. La norma de Frobenius de este operador (al cuadrado) se calcula seg\'un:
\begin{eqnarray}
 \Vert \hat{\rho} \Vert_F^2 &\equiv& tr\left( \rho^\dag\rho \right)\label{ec:frob_inicial}\\
&=&tr\left( \rho^2 \right)\\
&=& tr\left( \sum_{k=1}^n\rho_{ik}\rho_{kj} \right)\\
&=&\sum_{i,j=1}^n\left\{ \delta_{ij} \sum_{k=1}^n\rho_{ik}\rho_{kj} \right\} \\
&=&\sum_{i,j=1}^n\rho_{ij}\rho_{ji}\\
&=&\sum_{i,j=1}^n \vert \rho_{ij} \vert^2.\label{ec:frob_final}
\end{eqnarray}
\quad Esta propiedad reduce el n\'umero de operaciones necesarias\footnote{\label{pie:numero_peraciones}Por \emph{n\'umero de operaciones} entenderemos el n\'umero de multiplicaciones m\'as el n\'umero de sumas.} para el c\'alculo de la norma de $\mathcal{O}(n^3)$ a $\mathcal{O}(n^2)$.
\end{description}

\subsection{C\'omputo eficiente de operadores involucrados}\label{sec:cal_operadores_involucrados}

\quad De la ecuaci\'on (\ref{limbladiano}), (\ref{ec:H0}) y (\ref{ec:HI}), vemos que los operadores mencionados en la secci\'on (\ref{sec:operadores_involucrados}), son de gran importancia en los c\'omputos de la funci\'on $f$ a integrar. A continuaci\'on estudiaremos la estructura de \'estos operadores con el fin de reducir el n\'umero de operaciones necesarias para el c\'omputo del Hamiltoniano y el Limbladiano. Comenzaremos\footnote{Las variables utilizadas en \'estos estudios han sido descritas en la secci\'on (\ref{sec:variables_involucradas}).} esto estudiando el operador $\hat{a}$ definido en (\ref{ec:operador_creacion_a}). Antes, recordemos que si $C=A\otimes B$ es el producto tensorial de dos matrices $A\in\mathcal{M}_{n\times m}(\mathbb{C})$, $B\in\mathcal{M}_{l\times k}(\mathbb{C})$, entonces los coeficientes $c_{ij}$ de $C$ pueden ser calculados seg\'un:\begin{equation}\label{ec:kron}
 \begin{array}{lll}
  c_{ij}=a_{i_1j_1}\cdot b_{i_2j_2}, & i=1\ldots n\cdot l,j=1\ldots m\cdot k\\
i_1=\left\lceil \frac{i}{l}\right\rceil, & j_1=\left\lceil \frac{j}{k}\right\rceil\\
i_2=(i-1)\bmod l+1, & j_2=(j-1)\bmod k+1,
 \end{array}
\end{equation}
sin embargo, tras una breve inspecci\'on, nos damos cuenta que \'este producto tiene la \'util representaci\'on:
\clearpage\begin{equation}\label{ec:kron_esquema}C=\left(
\begin{array}{c:c:c:c}
a_{11}B & a_{12}B & \cdots & a_{1m}B \\ \hdashline
a_{21}B & a_{22}B & \vdots & a_{2m}B \\ \hdashline
\vdots & \vdots & \ddots & \vdots \\ \hdashline
a_{n1}B & a_{n2}B & \cdots & a_{nm}B
\end{array}\right).
\end{equation}
\subsubsection{Operador $\hat{a}$, de creaci\'on del modo $a$.}
\quad El operador de creaci\'on del modo $a$ que act\'ua de $\mathcal{H}_a\rightarrow\mathcal{H}_a$, es representado en la base can\'onica por la matriz de coeficientes $a_{ij},\,i,j=1\ldots \dimA$ seg\'un:\begin{equation}\label{ec:coef_creacion_a}
a_{ij}=\left\{
\begin{array}{cl}
 \sqrt{i} & ,j=i+1,i=1 \ldots (\dimA -1)\\
0, & , i>\dimA-1
\end{array}
\right..
\end{equation}
\quad Ahora bien, recordemos se define un operador $\hat{a}$, que act\'ua de $\mathcal{H}\rightarrow \mathcal{H}$ seg\'un lo visto en (\ref{sec:operadores_involucrados}). Para estudiar su estructura, utilizamos (\ref{ec:kron_esquema}) siendo en este caso $A$ una matriz identidad de orden $n=\dimAt$ y $B$ la matriz representante del operador de creaci\'on del modo $a$. As\'i, es inmediato que:\begin{equation}\label{ec:IkronA}
 I^n \otimes B=
\left(\begin{array}{cccc}
B & \mathbb{O}& \cdots &\mathbb{O}\\
\mathbb{O} & B &\ddots &\vdots\\
\vdots & \ddots & \ddots & \mathbb{O}\\
\mathbb{O} &\cdots &\mathbb{O} & B
\end{array}\right).
\end{equation}
\quad Donde $\mathbb{O}$ son matrices nulas del mismo tama\~no que $B$, siendo el n\'umero de bloques de la diagonal igual a $\dimAt$.\\

\quad Ahora, utilizando nuevamente (\ref{ec:kron_esquema}), siendo esta vez $B$ la matriz identidad, de orden $n$, y $A$ una matriz cuadrada cualquiera de orden $m$, se tiene que:\begin{equation}\label{ec:AkronI}
 A \otimes I^n=
\left(\begin{array}{ccc:ccc:ccc:ccc}
a_{11}& & & a_{12} & & &  & & &a_{1m}& &  \\ 
&\ddots& & & \ddots & &  & \cdots & &&\ddots &  \\ 
& &a_{11} & & & a_{12} &  & & & & & a_{1m} \\ \hdashline

a_{21}& & & a_{22} & & &  & & &a_{2m}& &  \\ 
&\ddots& & & \ddots & &  & \cdots & &&\ddots &  \\ 
& &a_{21} & & & a_{22} &  & & & & & a_{2m} \\ \hdashline

& & & & & & & & & & &  \\ 
&\vdots& & & \vdots & &  & \vdots & &&\vdots &  \\ 
& & & & & & & & & & & \\ \hdashline

a_{m1}& & & a_{m2} & & &  & & &a_{mm}& &  \\ 
&\ddots& & & \ddots & &  & \cdots & &&\ddots &  \\ 
& &a_{m1} & & & a_{m2} &  & & & & & a_{mm}
\end{array}\right).
\end{equation}
\quad Donde cada bloque es una matriz diagonal de orden $n$. Es claro que el n\'umero de bloques es $m^2$. Aplicando primero (\ref{ec:IkronA}) con la identidad de orden $n=\dimAt$ y $A$ el operador de creaci\'on del modo $a$, y luego aplicando a este resultado la ecuaci\'on (\ref{ec:AkronI}) con $A=I\otimes \hat{a}$ y la identidad de orden $\dimA$, se tiene que el operador definido en (\ref{ec:operador_creacion_a}) puede ser representado, seg\'un sus coeficientes, como\footnote{la sigla ``c.o.c.'' significa \emph{cualquier otro caso}.}:\begin{equation}\label{ec:coef_operador_creacion_generalizado}
(I^n\otimes \hat{a} \otimes I^l)_{i,j}=\left\{
\begin{array}{cl}
 \sqrt{\lceil \frac{i}{l} \rceil \mod m} & ,j=i+l,i=1 \ldots (n\cdot m -1)\cdot l\\
0 & ,\mbox{c.o.c.}
\end{array}
\right.,
\end{equation}
con $n\equiv\dimAt$, $m\equiv\dimA$ y $l\equiv\dimB$.
\subsubsection{Operador de creaci\'on generalizado}\label{sec:operador_creacion_generalizado}
\quad Siguiendo las mismas ideas de la secci\'on anterior, es sencillo encontrar la expresi\'on para el operador definido por:\begin{equation}
 \hat{a}^{k,m}_g:=I^{n_1}\otimes I^{n_2} \otimes \ldots \otimes I^{n_{g-1}} \otimes \hat{a}^k \otimes I^{n_{g+1}}\otimes \ldots \otimes I^{n_m},
\end{equation}
con $k$ siendo la dimensi\'on del subespacio en el cual act\'ua el operador de creaci\'on $\hat{a}$. Entonces:\begin{equation}
\left(\hat{a}^{k,m}_g\right)_{ij}=\left\{
\begin{array}{cl}
\sqrt{\lceil \frac{i}{\Pi_2} \rceil \mod k} & ,j=i+\Pi_2,i=1 \ldots (\Pi_1 -1)\cdot \Pi_2\\
0 & , i>(\Pi_1 -1)\cdot \Pi_2
\end{array}
\right.,
\end{equation}
con $\Pi_2=\prod^{n_m}_{i=g+1}n_i,\,\,\Pi_1=k\cdot\prod^{g-1}_{i=1}n_i$. Donde esta vez el espacio de Hilbert considerado en el cual act\'ua este operador es de dimensi\'on $\Pi_1 \cdot \Pi_2$. Esta expresi\'on ser\'a importante en vistas de generar un algoritmo que pueda ser utilizado en modelos m\'as generales que el considerado en este trabajo.
\subsubsection{Operador $\hat{a}_{ij}$, de transiciones at\'omicas}
\quad Recordemos que el operador de transici\'on del nivel at\'omico $\ket{j}$ a $\ket{i}$ est\'a definido por la proyecci\'on:
$$\ket{i}\bra{j},$$
donde cada $\ket{i}$ es un vector perteneciente a la base can\'onica, como fue visto en (\ref{sec:modelo_matematico}). Luego, esta proyecci\'on puede ser representada seg\'un la composici\'on:
$$e_i\cdot e_j^t,$$
que corresponde a una matriz de cuadrada de orden $\dimAt$ tal que todos sus elementos son nulos salvo el ubicado en la $i$-\'esima fila y la $j$-\'esima columna. Es decir los coeficientes $a_{kl}$ de la matriz que representa al operador $\hat{a}_{ij}$ pueden ser descritos por:\begin{equation}
 a_{kl}=\delta_{kj}\delta_{li},\qquad k,l=1\ldots \dimAt.\label{ec:coef_operador_transicion_Hat}
\end{equation}
\quad Ahora, repitiendo el an\'alisis de la secci\'on anterior para productos del tipo $A\otimes I^n$, esta vez con $A$ descrito seg\'un (\ref{ec:coef_operador_transicion_Hat}), se tendr\'a que la matriz representante de (\ref{ec:operador_transicion_H}) puede ser descrita por:\begin{equation}\hat{a}_{ij}\equiv \ket{i}\bra{j}\otimes I_a \otimes I_b=\label{ec:matriz_operador_transicion_H}
\left(\begin{array}{c:c:c:c}
\mathbb{O}^m_{11} & \mathbb{O}^m_{12} & \cdots & \mathbb{O}^m_{1n} \\ \hdashline
\mathbb{O}^m_{21} & \mathbb{O}^m_{22} & \cdots & \mathbb{O}^m_{2n} \\ \hdashline
\vdots & \vdots & I^m_{ji} & \vdots \\ \hdashline 
\mathbb{O}^m_{n1} & \mathbb{O}^m_{n2} & \cdots & \mathbb{O}^m_{nn}
\end{array}\right),
\end{equation}
con $m=\dimA\cdot \dimB$ y $n=\dimAt$. Es decir, \'estas matrices est\'an compuestas por $n^2$ bloques cuadrados de orden $m$, de los cuales todos son nulos salvo el bloque de la fila $j$, columna $i$ que corresponde a una matriz identidad de orden $m$. Luego, los coeficientes $a_{kl}:\,k,l=1\ldots n\cdot m=\dimT$ de esta matriz, pueden ser descritos por:\begin{equation}\label{ec:coef_operador_transicion_H}
 c_{kl}=\left\{\begin{array}{ll}
\delta_{kl}, & k=(j-1)\cdot n+1\ldots j\cdot n,\,l=(i-1)\cdot m+1\ldots i\cdot m \\
0, & \mbox{c.o.c.}
\end{array}\right..
\end{equation}
\subsubsection{Operador de transiciones at\'omicas generalizado}\label{sec:operador_transiciones_generalizado} \quad Deseamos ahora encontrar expresiones para los elementos no nulos de la matriz que represente al nuevo operador, definido por:\begin{equation}\label{ec:operador_transiciones_generalizado}
 \hat{a}_{ij}:=\ket{i}\bra{j} \otimes I^{n_1}\otimes I^{n_2} \otimes \ldots \otimes I^{n_d}.
\end{equation}
\quad En este caso, el esquema de la matriz ser\'a semejante al dado por (\ref{ec:matriz_operador_transicion_H}), la diferencia radicar\'a en que el orden $m$ del esquema anterior, esta vez corresponder\'a al producto total de las dimensiones de los dem\'as subespacios no asociados al \'atomo en cuesti\'on (modos del campo electromagn\'etico, otros \'atomos, etc.), es decir, en este caso $m=\prod_{i=1}^{d}n_i$. Esta vez, los coeficientes de las matrices representantes est\'an dados por:\begin{equation}\label{ec:coef_operador_transicion_generalizado}
 c_{kl}=\left\{\begin{array}{ll}
\delta_{kl}, & k=(j-1)\cdot \Pi+1\ldots j\cdot \Pi,\,l=(i-1)\cdot \Pi+1\ldots i\cdot \Pi \\
0, & \mbox{c.o.c.}
\end{array}\right..
\end{equation}
con $k,l=1\ldots \Pi$ y $\Pi=\prod_{i=1}^dn_i$.
\subsection{C\'omputo eficiente de operadores compuestos}\label{sec:computo_eficiente_operadores}
\quad De las expresiones (\ref{ec:coef_operador_creacion_generalizado}) y (\ref{ec:coef_operador_transicion_generalizado}) podemos apreciar que \'estas matrices cumplen con la propiedad de ser matrices de baja \emph{densidad} (o \emph{sparse}), en el sentido que la proporci\'on de elementos no nulos, sobre el total de elementos, es baja. De hecho, siendo $\dimT$ la dimensi\'on del espacio mayor y por ende el orden de los operadores generalizados definidos en la secci\'on anterior, se tiene que para el caso del operador de craci\'on $\hat{a}$ definido en (\ref{sec:operador_creacion_generalizado}), entonces su densidad $\rho_{\hat{a}}$ cumple:
$$\rho_{\hat{a}}=\mathcal{O}\left(\frac{1}{\dimT}\right),$$
mientras que para el operador $\hat{a}_{ij}$ definido en (\ref{sec:operador_transiciones_generalizado}) su densidad cumple:
$$\rho_{\hat{a}_{ij}}=\frac{1}{\dimAt\cdot\dimT}.$$

\quad Por ejemplo, para el caso m\'as sencillo estudiado, en el cual $\dimT=45$, se tiene $\hat{a}$ tiene una densidad del 2.1\% y la densidad de $\hat{a}_{ij}$ es de 0.4\%.\\

\quad La baja densidad de \'estos operadores y el poseer una estructura bien definida, nos permitir\'a escribir algoritmos que calculen expresiones algebraicas (en las que \'estos operadores participen) con mayor eficiencia, al no ser necesario realizar todas las operaciones que se realizar\'ian de no haber estructura y no poseer baja densidad. Esto lo analizamos con mayor profundidad a continuaci\'on, primero definiendo las siguientes variables de importancia, indicando al final entre par\'entesis los valores particulares para este trabajo.
\begin{description}

\item [$N$ :] \qquad N\'umero de espacios componentes del espacio mayor. ($N=3$).
\item[$n_i$ :] \qquad $i=1\ldots N$. Dimensi\'on del i-\'esimo espacio. ($n_1=5$).
\item[$\Pi_1^g$ :] \qquad Equivalente a $\prod_{i=1}^{g-1}n_i$ con $g=2\ldots N$.
\item[$\Pi_2^g$ :] \qquad Equivalente a $\prod_{i=g+1}^{N}n_i$.
\item[$T^g$ :] \qquad Equivalente a $(n_g-1)\Pi_2^g$.
\item[$\Pi_3^g$ :] \qquad Equivalente a $T+\Pi_2$.
\item[$\Pi$ :] \qquad  Equivalente a $\prod_{i=2}^N{n_i}$.
\item[$ord$ :] \qquad Arreglo de enteros que especifica la dimensi\'on de cada subespacio.
\end{description}
\quad A continuaci\'on, describiremos esquemas de c\'omputo para diferentes expresiones de inter\'es. Donde $\rho$ ser\'a una matriz compleja de orden $\dimT$ con coeficientes $\rho_{ij},\,i,j=1\ldots \dimT$. En todos los casos a presentar se tiene que el n\'umero original de operaciones a realizar (esto es sin aprovechar las estructuras de los operadores) es $$\mathcal{O}(\dimT^3).$$ 
\subsubsection{Expresiones tipo $\hat{\rho}\rightarrow \hat{a}\hat{\rho}$}\label{sec:arho}
\quad Con $\hat{a}\equiv\hat{a}^{n_g,N}_g$, definido en (\ref{sec:operador_creacion_generalizado}). A continuaci\'on presentamos brevemente esquematizaciones de c\'omo operan estas expresiones sobre $\rho$, junto con el algoritmo a utilizar dado el esquema, esto lo haremos a partir de lo expuesto en (\ref{ec:operador_transiciones_generalizado}) y (\ref{ec:coef_operador_transicion_generalizado}).
\subsubsection*{Expresiones $\hat{a}\hat{\rho}$ y $\hat{a}^{\dag}\hat{\rho}$: } \quad En el primer caso $\hat{a}\hat{\rho}$, $\rho$ es dividida en \emph{sub-matrices fila}, esquem\'aticamente:\begin{equation}\left(
 \begin{array}{c}
\vdots\\
B_{1} \\\hdashline
\vdots\\
B_{2}\\\hdashline
\multirow{2}{*}{\Huge$\vdots$}\\
\\\hdashline
\vdots\\
B_{n}
 \end{array}\right).\label{ec:esquema_rho_arho}
\end{equation}
\quad Donde los $n=\Pi_1$ bloques esta vez poseen $T^g$ filas y $\dimT$ columnas, es decir abarcan todos las columnas de $\rho$. Los puntos suspensivos justo debajo de los bloques indican elementos que no ser\'an considerados y ser\'an nulos en el resultado, cuyo esquema es:\begin{equation}\left(
 \begin{array}{c}
\mathcal{B}_{1} \\
\vdots\\\hdashline
\mathcal{B}_{2}\\
\vdots\\\hdashline
\multirow{2}{*}{\Huge$\vdots$}\\
\\\hdashline
\mathcal{B}_{n}\\
\vdots
 \end{array}\right).
\end{equation}
 \quad Es decir, se realiza un movimiento hacia arriba de los bloques seleccionados. Adem\'as si $M=\mathcal{B}_{i}$ y $S=B_{i}$ entonces las relaciones entre sus coeficientes son:\begin{equation}
M_{kl}=\gamma_{k}S_{kl},\,\qquad k=1\ldots T^g,\,l=1\ldots \dimT, \label{ec:bloque_arM}
\end{equation}
con\begin{equation}
%bii = alpha*sqrt(real(mod(ceiling(RDIV(ei,pi2)),k)))
 \gamma_{k}=\sqrt{\left\lceil \frac{k}{\Pi_2^g} \right\rceil\bmod n_g}.\label{ec:coef_gammas_arMod}
\end{equation}
\quad Escribimos a continuaci\'on el algoritmo que realiza el c\'omputo de esta expresi\'on:

\begin{algorithm}[H]
\caption{arMod}\label{algo:arMod}
\LinesNumbered
\KwIn{$\rho$,\, $id$: Matriz densidad, n\textdegree{} subespacio asociado al modo.}
\KwOut{$r=\hat{a}\rho$}
\Begin{
      $k=ord[id]$\;
      $pi1= producto(ord[1:id-1])$\;
      $pi2= producto(ord[id+1:\dimT])$\;
      $T=(k-1)*pi2$\;
      $pi3=T+pi2$\;
      $r=ceros(\dimT)$\;
      
      \For{$bi=0$ hasta $(pi1-1)\cdot pi3$; $bi+=pi3$}{
    \For{$ei=1$ hasta $T$}{
      \For{$ej=1$ hasta $\dimT$}{
	$\gamma=\sqrt{\left\lceil\frac{ei}{pi2}\right\rceil\bmod k }$\;
	$r[bi+ei,\,ej]=\gamma\cdot\rho[pi2+bi+ei,ej]$\;
      }
     }
  }
}
\end{algorithm}
\quad El n\'umero de operaciones realizado por el algoritmo es $$\mathcal{O}\left(\dimT^2\left(\frac{n_g-1}{n_g}\right)\right)=\mathcal{O}\left(\dimT^2\right).$$ El caso $\hat{a}^{\dag}{\rho}$ es an\'alogo al anterior, salvo que esta vez el movimiento de los bloques es hacia abajo.
\subsubsection*{Expresiones $\hat{\rho}\hat{a}$ y $\hat{\rho}\hat{a}^{\dag}$: } \quad En este caso, $\rho$ es dividida en \emph{sub-matrices columna}, esquem\'aticamente:\begin{equation}\rho=\left(
\begin{array}{cc:cc:cc:cc}
 B_1&\cdots & B_2 & \cdots& \multicolumn{2}{:c:}{\mbox{\Huge$\cdots$}} &B_n&\cdots
\end{array}\right),\label{ec:esquema_rhoa}
\end{equation}
donde los $B_i$ son sub-matrices de $\dimT$ filas y $T^g$ columnas, $n=\Pi_1$, y los puntos suspensivos a la derecha de cada bloque denotan columnas que no son consideradas. As\'i el resultado se puede describir por:\begin{equation}\left(
\begin{array}{cc:cc:cc:cc}
 \cdots&\mathcal{B}_1 & \cdots & \mathcal{B}_2& \multicolumn{2}{:c:}{\mbox{\Huge$\cdots$}} &\cdots&\mathcal{B}_n
\end{array}\right),
\end{equation}
\quad Es decir, se mueven los bloques hacia la derecha. Adem\'as si $M=\mathcal{B}_{i}$ y $S=B_{i}$ entonces las relaciones entre sus coeficientes son:\begin{equation}
M_{kl}=\gamma_{l}S_{kl},\,\qquad k=1\ldots \dimT,\,l=1\ldots T^g, \label{ec:bloque_raM}
\end{equation}
con\begin{equation}
 \gamma_{l}=\sqrt{\left\lceil \frac{l}{\Pi_2^g} \right\rceil\bmod n_g}.\label{ec:coef_gammas_raMod}
\end{equation}
\quad Escribimos a continuaci\'on el algoritmo que realiza el c\'omputo de esta expresi\'on:

\begin{algorithm}[H]
\caption{raMod}\label{algo:raMod}
\LinesNumbered
\KwIn{$\rho$, $id$: Matriz densidad, n\textdegree{} subespacio asociado al modo.}
\KwOut{$r=\rho\hat{a}$}
\Begin{
      $k=ord[id]$\;
      $pi1= producto(ord[1:id-1])$\;
      $pi2= producto(ord[id+1:\dimT])$\;
      $T=(k-1)*pi2$\;
      $pi3=T+pi2$\;
      $r=ceros(\dimT)$\;
      
      \For{$bj=0$ hasta $(pi1-1)\cdot pi3$; $bi+=pi3$}{
    \For{$ei=1$ hasta $\dimT$}{
      \For{$ej=1$ hasta $T$}{
	$\gamma=\sqrt{\left\lceil\frac{ej}{pi2}\right\rceil\bmod k }$\;
	$r[ei,\,pi2+bj+ej]=\gamma\cdot\rho[ei,bj+ej]$\;
      }
     }
  }
}
\end{algorithm}
\quad El n\'umero de operaciones realizado por el algoritmo es $$\mathcal{O}\left(\dimT^2\left(\frac{n_g-1}{n_g}\right)\right)=\mathcal{O}\left(\dimT^2\right).$$ El caso ${\rho}\hat{a}^{\dag}$ es an\'alogo al anterior, salvo que esta vez el movimiento de los bloques es hacia la izquierda. 
\subsubsection{Expresiones tipo $\hat{\rho}\rightarrow \hat{a}_{ij}\hat{\rho}$}\label{sec:aijrho} \quad Con $\hat{a}_{ij}$ definido en (\ref{sec:operador_transiciones_generalizado}). Esta vez, a partir de (\ref{ec:coef_operador_creacion_generalizado}) podemos deducir esquemas simples que caracterizen la acci\'on del operador de transiciones generalizado en las distintas expresiones consideradas en el Limbladiano.
\subsubsection*{Expresi\'on $\hat{a}_{ij}\hat{\rho}$: }\quad  Esta vez, $\rho$ ser\'a dividida en $\dimAt$ bloques fila, de $\Pi$ filas y $\dimT$ columnas, es decir:\begin{equation}
 \rho=\left(\begin{array}{c}              
B_1\\
B_2\\
\vdots\\
B_{\Pi}
\end{array}\right).\label{ec:esquema_aijrho}
\end{equation}
Y resultado es esquematizado por:
$$\mbox{$i$-\'esimo}\{\left(
\begin{array}{c}
 \cdots\\
B_j\\
\cdots
\end{array}
\right).
$$
Por ejemplo si $\dimA=3$ y escribimos:
$$\rho=\left(\begin{array}{ccc}              
B_1\\
B_2\\
B_3
\end{array}\right),
$$
entonces:
$$\hat{a}_{13}\rho=\left(\begin{array}{ccc}              
B_{3}\\
\mathbb{O}^\Pi\\
\mathbb{O}^\Pi
             \end{array}\right).
$$
\quad Escribimos a continuaci\'on el algoritmo que realiza el c\'omputo de esta expresi\'on:

\begin{algorithm}[H]
\caption{arAt}\label{algo:arAt}
\LinesNumbered
\KwIn{$\rho$, $i$, $j$.}
\KwOut{$r=\hat{a}_{ij}\rho\hat{a}_{ij}^\dag$}
\Begin{
      $pi = producto(ord[2:\dimT])$\;
      $i1 = i\cdot pi$\;
      $j1 = j\cdot pi$\;
      $r=ceros(pi)$\;
     \For{$ei=1$ hasta $pi$}{
     \For{$ej=1$ hasta $\dimT$}{
      $r[i1+ei,ej]=rho[j1+ei,j1+ej]$\;
      }
  }
}
\end{algorithm}
\quad El n\'umero de \emph{asignaciones} realizado por el algoritmo es $$\frac{\dimT^2}{\dimAt}=\mathcal{O}(\dimT^2).$$ El caso $\hat{a}_{ij}^{\dag}\hat{\rho}$ es igual al anterior, pero intercambiando $i$ po $j$, esto es inmediato dado que $\left(\ket{i}\bra{j}\right)^{\dag}=\ket{j}\bra{i}$.
\subsubsection*{Expresi\'on $\hat{\rho}\hat{a}_{ij}$: } Esta vez la esquematizaci\'on de la forma de escoger los elementos de $\rho$ viene dada por los $\dimAt$ bloques columna de $\dimT$ filas y $\Pi$ columnas:\begin{equation}
\rho=\left(\begin{array}{cccc}              
B_{1}&B_2&\cdots&B_{\Pi}
             \end{array}\right).\label{ec:esquema_rhoaij}
\end{equation}
Y resultado es esquematizado por:
$$\overset{ \overbrace{}^{\mbox{$j$-\'esimo}}}{\left(\begin{array}{ccc}
\cdots&B_{i}&\cdots
        \end{array}\right)}.
$$
Por ejemplo si $\dimA=3$ y escribimos:
$$\rho=\left(\begin{array}{ccc}              
B_{1}&B_2&B_3
             \end{array}\right),
$$
entonces:
$$\rho\hat{a}_{13}=\left(\begin{array}{ccc}              
\mathbb{O}^\Pi&\mathbb{O}^\Pi&B_{1}
             \end{array}\right).
$$
\quad Escribimos a continuaci\'on el algoritmo que realiza el c\'omputo de esta expresi\'on:

\begin{algorithm}[H]
\caption{raAt}\label{algo:raAt}
\LinesNumbered
\KwIn{$\rho$, $i$, $j$.}
\KwOut{$r=\hat{a}_{ij}\rho\hat{a}_{ij}^\dag$}
\Begin{
      $pi = producto(ord[2:\dimT])$\;
      $i1 = i\cdot pi$\;
      $j1 = j\cdot pi$\;
      $r=ceros(pi)$\;
     \For{$ei=1$ hasta $\dimT$}{
     \For{$ej=1$ hasta $pi$}{
      $r[ei,j1+ej]=rho[ei,i1+ej]$\;
      }
  }
}
\end{algorithm}
\quad El n\'umero de \emph{asignaciones} realizado por el algoritmo es $$\frac{\dimT^2}{\dimAt}=\mathcal{O}(\dimT^2).$$ Tambi\'en, el caso $\hat{\rho}\hat{a}_{ij}^{\dag}$ equivalente al primero pero intercambiando $i$ por $j$.
\subsubsection{Expresiones tipo $\hat{\rho}\rightarrow \hat{a}\hat{\rho}\hat{a}^{\dag}$}\label{sec:expre_modales_orden2}
\quad Estos casos son una continuaci\'on de lo realizado en (\ref{sec:arho}).
\subsubsection*{Expresi\'on $\hat{a}\hat{\rho}\hat{a}^{\dag}$: } \quad Seg\'un lo visto en (\ref{sec:arho}) esto es seleccionar los bloques fila y moverlos hacia arriba y del resultado escoger los bloques columna y moverlos hacia la izquierda. El efecto neto es eleccionar $(\Pi_1^g)^2$ bloques de $\rho$ seg\'un el esquema:\\\begin{equation}\rho=\left(
 \begin{array}{cc:cc:cc:cc}
\ddots&\vdots& \ddots&\vdots& \multicolumn{2}{:c:}{\multirow{2}{*}{$\mathbf{\cdots}$} } &\ddots &\vdots \\
\cdots&B_{11}& \cdots & B_{12}& & &\cdots& B_{1n}\\\hdashline

\ddots&\vdots& \ddots&\vdots& \multicolumn{2}{:c:}{\multirow{2}{*}{$\mathbf{\cdots}$} } & \ddots&\vdots \\
\cdots&B_{21}& \cdots & B_{22}& & &\cdots& B_{2n}\\\hdashline
\multicolumn{2}{c:}{\multirow{2}{*}{\Huge$\vdots$} } & \multicolumn{2}{:c:}{\multirow{2}{*}{\Huge$\vdots$} } & \multicolumn{2}{:c:}{\multirow{2}{*}{\Huge$\vdots$} }&\multicolumn{2}{:c}{\multirow{2}{*}{\Huge$\vdots$} }\\
&&&&&&& \\\hdashline
\ddots&\vdots& \ddots&\vdots& \multicolumn{2}{:c:}{\multirow{2}{*}{$\mathbf{\cdots}$} } & \ddots&\vdots \\
\cdots&B_{n1}& \cdots & B_{n2}& & &\cdots& B_{nn}
\end{array}\right),
\end{equation}
con $n=\Pi_1^g$. Los puntos suspensivos a los lados de los bloques $B_{ij}$ indican elementos de $\rho$ que no son seleccionados. De hecho la distancia (hacia arriba o hacia abajo) entre estos bloques es igual a $\Pi_2^g$. As\'i, el resultado de la expresi\'on $\hat{a}\hat{\rho}\hat{a}^{\dag}$ se describe seg\'un:\\\begin{equation}\left(
 \begin{array}{cc:cc:cc:cc}
\mathcal{B}_{11} & \cdots & \mathcal{B}_{12} & \cdots & \multicolumn{2}{:c:}{\multirow{2}{*}{$\cdots$} } & \mathcal{B}_{1n}&\cdots \\
\vdots & \ddots & \vdots & \ddots & & & \vdots & \ddots \\\hdashline
\mathcal{B}_{21} & \cdots & \mathcal{B}_{21} & \cdots & \multicolumn{2}{:c:}{\multirow{2}{*}{$\cdots$} } & \mathcal{B}_{2n}&\cdots \\
\vdots & \ddots & \vdots & \ddots & & & \vdots & \ddots \\\hdashline
\multicolumn{2}{c:}{\multirow{2}{*}{\Huge$\vdots$} } & \multicolumn{2}{:c:}{\multirow{2}{*}{\Huge$\vdots$} } & \multicolumn{2}{:c:}{\multirow{2}{*}{\Huge$\vdots$} }&\multicolumn{2}{:c}{\multirow{2}{*}{\Huge$\vdots$} }\\
&&&&&&& \\\hdashline

\mathcal{B}_{n1} & \cdots & \mathcal{B}_{n2} & \cdots & \multicolumn{2}{:c:}{\multirow{2}{*}{$\cdots$} } & \mathcal{B}_{nn}&\cdots \\
\vdots & \ddots & \vdots & \ddots & & & \vdots & \ddots
\end{array}\right),
\end{equation}
donde los puntos suspensivos a los lados de cada bloque indican elementos nulos, adem\'as cada bloque $M=\mathcal{B}_{ij}$ corresponde al bloque $S=B_{ij}$ transformado seg\'un las relaciones de sus coeficientes:
$$M_{kl}=\gamma_{kl}S_{kl},\,\qquad k,l=1\ldots T^g,$$
con:\begin{equation}
 \gamma_{kl}=\sqrt{ \left(\left\lceil \frac{k}{\Pi_2^g} \right\rceil\bmod n_g\right)\left(\left\lceil \frac{l}{\Pi_2^g} \right\rceil\bmod n_g \right) }.\label{ec:coef_gammas}
\end{equation}
\quad El efecto entonces de esta transformaci\'on es mover ciertos bloques de $\rho$ hacia arriba y la izquierda, mutiplic\'andolos por los $\gamma_{kl}$ y anulando todos los dem\'as elementos no considerados. Podemos ahora escribir el algoritmo que realice los respectivos c\'omputos de esta expresi\'on:

\begin{algorithm}[H]
\caption{aratMod}\label{algo:aratMod}
\LinesNumbered
\KwIn{$\rho$, $id$: Matriz densidad, n\textdegree{} subespacio asociado al modo.}
\KwOut{$r=\hat{a}\rho\hat{a}^\dag$}
\Begin{
      $k=ord[id]$\;
      $pi1= producto(ord[1:id-1])$\;
      $pi2= producto(ord[id+1:\dimT])$\;
      $T=(k-1)*pi2$\;
      $pi3=T+pi2$\;
      $r=ceros(\dimT)$\;
      
      \For{$\{bi,\,bj\}=0$ hasta $(pi1-1)\cdot pi3$; $\{bi,\,bj\}+=pi3$\label{algo:aratMod_bucle0}}{
    \For{$\{ei,\,ej\}=1$ hasta $T$}{
     $\gamma=\sqrt{ \left(\left\lceil\frac{ei}{pi2}\right\rceil\bmod k\right)\left(\left\lceil\frac{ej}{pi2}\right\rceil\bmod k \right) }$\;
     $r[bi+ei,\,bi+ej]=\gamma\cdot\rho[pi2+bi+ej,\,pi2+bj+ej]$\;
     }
  }
}
\end{algorithm}
\quad As\'i, el n\'umero de operaciones baja ahora a: $$\mathcal{O}\left(\dimT^2\left(\frac{n_g-1}{n_g}\right)^2\right)=\mathcal{O}\left(\dimT^2\right).$$
\subsubsection*{Expresi\'on $\hat{a}^{\dag}\hat{\rho}\hat{a}$: } Esta expresi\'on tiene las mismas caracter\'isticas que la anterior, con la salvedad que esta vez los bloques son definidos desde arriba a la izquierda y desplazados hacia abajo a la derecha. Esquem\'aticamente la selecci\'on es:\begin{equation}\rho=\left(
 \begin{array}{cc:cc:cc:cc}
B_{11} & \cdots & B_{12} & \cdots & \multicolumn{2}{:c:}{\multirow{2}{*}{$\cdots$} } & B_{1n}&\cdots \\
\vdots & \ddots & \vdots & \ddots & & & \vdots & \ddots \\\hdashline
B_{21} & \cdots & B_{21} & \cdots & \multicolumn{2}{:c:}{\multirow{2}{*}{$\cdots$} } & B_{2n}&\cdots \\
\vdots & \ddots & \vdots & \ddots & & & \vdots & \ddots \\\hdashline
\multicolumn{2}{c:}{\multirow{2}{*}{\Huge$\vdots$} } & \multicolumn{2}{:c:}{\multirow{2}{*}{\Huge$\vdots$} } & \multicolumn{2}{:c:}{\multirow{2}{*}{\Huge$\vdots$} }&\multicolumn{2}{:c}{\multirow{2}{*}{\Huge$\vdots$} }\\
&&&&&&& \\\hdashline

B_{n1} & \cdots & B_{n2} & \cdots & \multicolumn{2}{:c:}{\multirow{2}{*}{$\cdots$} } & B_{nn}&\cdots \\
\vdots & \ddots & \vdots & \ddots & & & \vdots & \ddots
\end{array}\right),
\end{equation}
mientras que el resultado de la expresi\'on es:\begin{equation}\left(
 \begin{array}{cc:cc:cc:cc}
\ddots&\vdots& \ddots&\vdots& \multicolumn{2}{:c:}{\multirow{2}{*}{$\mathbf{\cdots}$} } &\ddots &\vdots \\
\cdots&\mathcal{B}_{11}& \cdots & \mathcal{B}_{12}& & &\cdots& \mathcal{B}_{1n}\\\hdashline

\ddots&\vdots& \ddots&\vdots& \multicolumn{2}{:c:}{\multirow{2}{*}{$\mathbf{\cdots}$} } & \ddots&\vdots \\
\cdots&\mathcal{B}_{21}& \cdots & \mathcal{B}_{22}& & &\cdots& \mathcal{B}_{2n}\\\hdashline
\multicolumn{2}{c:}{\multirow{2}{*}{\Huge$\vdots$} } & \multicolumn{2}{:c:}{\multirow{2}{*}{\Huge$\vdots$} } & \multicolumn{2}{:c:}{\multirow{2}{*}{\Huge$\vdots$} }&\multicolumn{2}{:c}{\multirow{2}{*}{\Huge$\vdots$} }\\
&&&&&&& \\\hdashline
\ddots&\vdots& \ddots&\vdots& \multicolumn{2}{:c:}{\multirow{2}{*}{$\mathbf{\cdots}$} } & \ddots&\vdots \\
\cdots&\mathcal{B}_{n1}& \cdots & \mathcal{B}_{n2}& & &\cdots& \mathcal{B}_{nn}
\end{array}\right).
\end{equation}
\quad Siendo la relaci\'on entre los bloques $B_{ij}$ y $\mathcal{B}_{ij}$ la misma que la definida en la secci\'on anterior. El algoritmo de c\'omputo es entonces:

\begin{algorithm}[H]
\caption{atraMod}\label{algo:atraMod}
\LinesNumbered
\KwIn{$\rho$, $id$: Matriz densidad, n\textdegree{} subespacio asociado al modo.}
\KwOut{$r=\hat{a}^\dag\rho\hat{a}$}
\Begin{
      $k=ord[id]$\;
      $pi1= producto(ord[1:id-1])$\;
      $pi2= producto(ord[id+1:\dimT])$\;
      $T=(k-1)*pi2$\;
      $pi3=T+pi2$\;
      $r=ceros(\dimT)$\;
      
      \For{$\{bi,\,bj\}=0$ hasta $(pi1-1)\cdot pi3$; $\{bi,\,bj\}+=pi3$}{
    \For{$\{ei,\,ej\}=1$ hasta $T$}{
     $\gamma=\sqrt{ \left(\left\lceil\frac{ei}{pi2}\right\rceil\bmod k\right)\left(\left\lceil\frac{ej}{pi2}\right\rceil\bmod k \right) }$\;
     $r[pi2+bi+ei,\,pi2+bi+ej]=\gamma\cdot\rho[bi+ej,\,bj+ej]$\;
     }
  }
}
\end{algorithm}
\quad Siendo, tambi\'en en este caso, el n\'umero de operaciones igual a: $$\mathcal{O}\left(\dimT^2\left(\frac{n_g-1}{n_g}\right)^2\right)=\mathcal{O}\left(\dimT^2\right).$$
\subsubsection*{Expresi\'on $\hat{a}^{\dag}\hat{a}\hat{\rho}$: }En este caso, la definici\'on de los bloques la misma que la utilizada en la secci\'on (\ref{sec:arho}). Y podemos observar de aquella secci\'on que esta vez no habr\'a movimiento de los bloques, la \'unica diferencia estar\'a entonces en los coeficientes que definen la transformaci\'on de los bloques. As\'i el esquema para $\rho$ es el mismo que (\ref{ec:esquema_rho_arho}), pero el resultado es:\begin{equation}\left(
 \begin{array}{c}
\vdots\\
\mathcal{B}_{1} \\\hdashline
\vdots\\
\mathcal{B}_{2}\\\hdashline
\multirow{2}{*}{\Huge$\vdots$}\\
\\\hdashline
\vdots\\
\mathcal{B}_{n}
 \end{array}\right).
\end{equation}
Definiendo nuevamente $M=\mathcal{B}_{i}$ y $S=B_{i}$, entonces las nuevas relaciones entre sus coeficientes son:\begin{equation}
M_{kl}=\gamma_{k}S_{kl},\,\qquad k=1\ldots T^g,\,l=1\ldots \dimT, \label{ec:subbloque_atarM}
\end{equation}
y:\begin{equation}
 \gamma_{k}=\left\lceil \frac{k}{\Pi_2^g} \right\rceil\bmod n_g.\label{ec:coef_gammas_atarMod}
\end{equation}
\quad Esta vez, el algoritmo para el c\'omputo de la expresi\'on es:

\begin{algorithm}[H]
\caption{atarMod}\label{algo:atarMod}
\LinesNumbered
\KwIn{$\rho$, $id$: Matriz densidad, n\textdegree{} subespacio asociado al modo.}
\KwOut{$r=\hat{a}^\dag\hat{a}\rho$}
\Begin{
      $k=ord[id]$\;
      $pi1= producto(ord[1:id-1])$\;
      $pi2= producto(ord[id+1:\dimT])$\;
      $T=(k-1)*pi2$\;
      $pi3=T+pi2$\;
      $r=ceros(\dimT)$\;
      
      \For{$bi=0$ hasta $(pi1-1)\cdot pi3$; $bi+=pi3$}{
    \For{$ei=1$ hasta $T$}{
      \For{$ej=1$ hasta $\dimT$}{
	$\gamma=\left\lceil\frac{ei}{pi2}\right\rceil\bmod k$\;
	$r[pi2+bi+ei,\,ej]=\gamma\cdot\rho[pi2+bi+ei,\,ej]$\;
      }
     }
  }
}
\end{algorithm}
\quad Siendo ahora el n\'umero de operaciones: $$\mathcal{O}\left(\dimT^2\left(\frac{n_g-1}{n_g}\right)\right)=\mathcal{O}\left(\dimT^2\right).$$
\subsubsection*{Expresi\'on $\hat{\rho}\hat{a}^{\dag}\hat{a}$: } En este caso, los bloques se definen como \emph{sub-matrices columna}, desde el lado derecho de $\rho$. Esquem\'aticamente la selecci\'on desde los elementos de $\rho$ es:\begin{equation}\rho=\left(
\begin{array}{cc:cc:cc:cc}
 \cdots&B_1 & \cdots & B_2& \multicolumn{2}{:c:}{\mbox{\Huge$\cdots$}} &\cdots&B_n
\end{array}\right),
\end{equation}
donde los $B_i$ son sub-matrices de $\dimT$ filas y $T$ columnas, $n=\Pi_1$, y los puntos suspensivos a la derecha de cada bloque denotan columnas que no son consideradas. As\'i el resultado se puede describir por:\begin{equation}\left(
\begin{array}{cc:cc:cc:cc}
 \cdots&\mathcal{B}_1 & \cdots & \mathcal{B}_2& \multicolumn{2}{:c:}{\mbox{\Huge$\cdots$}} &\cdots&\mathcal{B}_n
\end{array}\right),
\end{equation}
\quad Adem\'as si $M=\mathcal{B}_{i}$ y $S=B_{i}$ entonces las relaciones entre sus coeficientes son:\begin{equation}
M_{kl}=\gamma_{l}S_{kl},\,\qquad k=1\ldots \dimT,\,l=1\ldots T^g, \label{ec:subbloque_rataM}
\end{equation}
y:\begin{equation}
 \gamma_{l}=\left\lceil \frac{l}{\Pi_2^g} \right\rceil\bmod n_g.\label{ec:coef_gammas_rataMod}
\end{equation}
\quad Finalmente el algoritmo de c\'omputo, esta vez es:

\begin{algorithm}[H]
\caption{rataMod}\label{algo:rataMod}
\LinesNumbered
\KwIn{$\rho$, $id$: Matriz densidad, n\textdegree{} subespacio asociado al modo.}
\KwOut{$r=\rho\hat{a}^\dag\hat{a}$}
\Begin{
      $k=ord[id]$\;
      $pi1= producto(ord[1:id-1])$\;
      $pi2= producto(ord[id+1:\dimT])$\;
      $T=(k-1)*pi2$\;
      $pi3=T+pi2$\;
      $r=ceros(\dimT)$\;
      
     \For{$bj=0$ hasta $(pi1-1)\cdot pi3$; $bj+=pi3$}{
      \For{$ei=1$ hasta $\dimT$}{
	\For{$ej=1$ hasta $T$}{
	  $\gamma=\left\lceil\frac{ej}{pi2}\right\rceil\bmod k$\;
	  $r[ei,\,pi2+bj+ej]=\gamma\cdot\rho[ei,\,pi2+bj+ej]$\;
	}
      }
  }
}
\end{algorithm}
Y el n\'umero de operaciones necesarias es: $$\mathcal{O}\left(\dimT^2\left(\frac{n_g-1}{n_g}\right)\right)=\mathcal{O}\left(\dimT^2\right).$$
\subsubsection*{Expresi\'on $\hat{\rho}\hat{a}\hat{a}^{\dag}$: } Utilizando el mismo esquema que en (\ref{ec:esquema_rhoa}), esta vez el resultado se puede describir por:\begin{equation}\left(
\begin{array}{cc:cc:cc:cc}
 \mathcal{B}_1 & \cdots & \mathcal{B}_2 & \cdots & \multicolumn{2}{:c:}{\mbox{\Huge$\cdots$}} &\mathcal{B}_n&\cdots
\end{array}\right),
\end{equation}
Adem\'as si $M=\mathcal{B}_{i}$ y $S=B_{i}$ entonces las relaciones entre sus coeficientes son:\begin{equation}
M_{kl}=\gamma_{l}S_{kl},\,\qquad k=1\ldots \dimT,\,l=1\ldots T^g, \label{ec:subbloque_raatM}
\end{equation}
y:\begin{equation}
 \gamma_{k}=\left\lceil \frac{k}{\Pi_2^g} \right\rceil\bmod n_g.\label{ec:coef_gammas_raatMod}
\end{equation}
\quad Finalmente el algoritmo de c\'omputo, esta vez es:

\begin{algorithm}[H]
\caption{raatMod}\label{algo:raatMod}
\LinesNumbered
\KwIn{$\rho$, $id$: Matriz densidad, n\textdegree{} subespacio asociado al modo.}
\KwOut{$r=\rho\hat{a}\hat{a}^\dag$}
\Begin{
      $k=ord[id]$\;
      $pi1= producto(ord[1:id-1])$\;
      $pi2= producto(ord[id+1:\dimT])$\;
      $T=(k-1)*pi2$\;
      $pi3=T+pi2$\;
      $r=ceros(\dimT)$\;
      
     \For{$bj=0$ hasta $(pi1-1)\cdot pi3$; $bj+=pi3$}{
      \For{$ei=1$ hasta $\dimT$}{
	\For{$ej=1$ hasta $T$}{
	  $\gamma=\left\lceil\frac{ej}{pi2}\right\rceil\bmod k$\;
	  $r[ei,\,bj+ej]=\gamma\cdot\rho[ei,\,bj+ej]$\;
	}
      }
  }
}
\end{algorithm}
\quad Y el n\'umero de operaciones necesarias es: $$\mathcal{O}\left(\dimT^2\left(\frac{n_g-1}{n_g}\right)\right)=\mathcal{O}\left(\dimT^2\right).$$
\subsubsection{Expresiones tipo $\hat{\rho}\rightarrow \hat{a}_{ij}\hat{\rho}\hat{a}_{ij}^{\dag}$}\label{sec:expre_atomicas_orden2} \quad Esta vez se continuar\'a lo realizado en (\ref{sec:aijrho}). Comenzaremos con:
\subsubsection*{Expresi\'on $\hat{a}_{ij}\hat{\rho}\hat{a}_{ij}^{\dag}$: } \quad Esta vez, $\rho$ ser\'a dividido en $\dimAt^2$ bloques cuadrados del mismo orden $\Pi$. As\'i, $\rho$ es esquematizado seg\'un:\begin{equation}\rho=\left(
 \begin{array}{c:c:c:c}
B_{11}&B_{12}&\cdots&B_{1\,\Pi}\\\hdashline
B_{21}&B_{22}&\cdots&B_{2\,\Pi}\\\hdashline
\vdots &\vdots &\ddots &\vdots \\\hdashline
B_{\Pi\,1}&B_{\Pi\,2}&\cdots&B_{\Pi\,\Pi}
 \end{array}\right).
\end{equation}
Y el resultado de $\hat{a}_{ij}\hat{\rho}\hat{a}_{ij}^{\dag}$ es esquematizado seg\'un:\begin{equation}
 \mbox{$i$-\'esimo}\{
\overset{\overbrace{}^{ \mbox{$i$-\'esimo} }}{\left(\begin{array}{ccc}
&\vdots&\\
\cdots&B_{jj}&\cdots\\
&\vdots&
\end{array}
\right)}.
\end{equation}
Es decir, el bloque $B_{jj}$ es trasladado a la posici\'on del bloque $B_{ii}$ y el resto de la matriz resultado son elementos nulos. Por ejemplo, si $\dimAt=3$, podemos dividir $\rho$ en los nueve bloques cuadrados de orden $\Pi$:
$$\rho=\left(\begin{array}{ccc}
B_{11}&B_{12}&B_{13}\\
B_{21}&B_{22}&B_{23}\\
B_{31}&B_{32}&B_{33}
\end{array}\right),$$
entonces:
$$\hat{a}_{13}\rho\hat{a}^{\dag}_{13}=\left(
\begin{array}{ccc}
 B_{33} & \mathbb{O}^\Pi& \mathbb{O}^\Pi\\
\mathbb{O}^\Pi&\mathbb{O}^\Pi&\mathbb{O}^\Pi\\ 
\mathbb{O}^\Pi&\mathbb{O}^\Pi&\mathbb{O}^\Pi
\end{array}\right).$$
\quad El algoritmo que realiza estos c\'omputos es descrito a continuaci\'on:

\begin{algorithm}[H]
\caption{aratAt}\label{algo:aratAt}
\LinesNumbered
\KwIn{$\rho$, $i$, $j$.}
\KwOut{$r=\hat{a}_{ij}\rho\hat{a}_{ij}^\dag$}
\Begin{
      $pi = producto(ord[2:\dimT])$\;
      $i1 = i\cdot pi$\;
      $j1 = j\cdot pi$\;
      $r=ceros(pi)$\;
     \For{$\{ei,\,ej\}=1$ hasta $pi$}{
      $r[i1+ei,i1+ej]=rho[j1+ei,j1+ej]$\;
      }
}
\end{algorithm}
\subsubsection*{Expresi\'on $\hat{a}_{ij}^{\dag}\hat{a}_{ij}\hat{\rho}$} Utilizando nuevamente el esquema (\ref{ec:esquema_aijrho}), esta vez el resultado es esquematizado por:
$$\mbox{$j$-\'esimo}\{\left(
\begin{array}{c}
 \cdots\\
B_j\\
\cdots
\end{array}
\right).
$$
Por ejemplo si $\dimA=3$ y escribimos:
$$\rho=\left(\begin{array}{ccc}              
B_1\\
B_2\\
B_3
\end{array}\right),
$$
entonces:
$$\rho\hat{a}^{\dag}_{13}\hat{a}_{13}=\left(\begin{array}{ccc}              
\mathbb{O}^\Pi\\
\mathbb{O}^\Pi\\
B_{3}
             \end{array}\right).
$$
\quad Describimos ahora el algoritmo que realiza \'estos c\'omputos:

\begin{algorithm}[H]
\caption{atarAt}\label{algo:atarAt}
\LinesNumbered
\KwIn{$\rho$, $i$, $j$.}
\KwOut{$r=\hat{a}_{ij}^\dag\hat{a}_{ij}\rho$}
\Begin{
      $pi = producto(ord[2:\dimT])$\;
      $j1 = j\cdot pi$\;
      $r=ceros(pi)$\;
     \For{$ei=1$ hasta $pi$}{
     \For{$ej=1$ hasta $\dimT$}{
      $r[j1+ei,j1+ej]=rho[j1+ei,j1+ej]$\;
      }
  }
}
\end{algorithm}
\subsubsection*{Expresi\'on $\hat{\rho}\hat{a}_{ij}^{\dag}\hat{a}_{ij}$} Utilizando nuevamente el esquema (\ref{ec:esquema_rhoaij}), el resultado estar\'a dado por:
$$\overset{ \overbrace{}^{\mbox{$j$-\'esimo}}}{\left(\begin{array}{ccc}
\cdots&B_{j}&\cdots
        \end{array}\right)}.
$$
\quad Es decir, en este caso no hay movimiento del bloque, s\'olo anulaci\'on de todo el resto de los elementos. Por ejemplo si $\dimA=3$ y escribimos:
$$\rho=\left(\begin{array}{ccc}              
B_{1}&B_2&B_3
             \end{array}\right),
$$
entonces:
$$\hat{a}^{\dag}_{13}\hat{a}_{13}\rho=\left(\begin{array}{ccc}              
\mathbb{O}^\Pi&\mathbb{O}^\Pi&B_{3}
             \end{array}\right).
$$
\quad A continuaci\'on se presenta el algoritmo que genera \'estos c\'omputos:

\begin{algorithm}[H]
\caption{rataAt}\label{algo:rataAt}
\LinesNumbered
\KwIn{$\rho$, $i$, $j$.}
\KwOut{$r=\rho\hat{a}_{ij}^\dag\hat{a}_{ij}$}
\Begin{
      $pi = producto(ord[2:\dimT])$\;
      $j1 = j\cdot pi$\;
      $r=ceros(pi)$\;
     \For{$ei=1$ hasta $\dimT$}{
     \For{$ej=1$ hasta $pi$}{
      $r[j1+ei,j1+ej]=rho[j1+ei,j1+ej]$\;
      }
  }
}
\end{algorithm}
\subsubsection*{N\'umero de operaciones} Los resultados mostrados en esta secci\'on son de relevante importancia, puesto que hemos reemplazado las $\mathcal{O}(\dimT^3)$ operaciones por: 
$$\frac{\dimT^2}{\dimAt^2}=\mathcal{O}(\dimT^2)$$
\emph{asignaciones} en el primer caso, y:
$$\frac{\dimT^2}{\dimAt}=\mathcal{O}(\dimT^2),$$
\emph{asignaciones} en el los dos \'ultimos.

\subsection{Algoritmos espec\'ificos}
\quad Ahora que conocemos la forma de computar los operadores involucrados en nuestro modelo, y las tambi\'en las composiciones de dichos operadores, utilizamos \'estos algoritmos para escribir los que compondr\'an a los algoritmos generales escritos en (\ref{sec:codigos_generales}). Antes, definimos las variables globales a ser utilizadas:

\begin{tabular*}{\textwidth}{ll}
\hline\multirow{2}{*}{\large{Variable}}&\multirow{2}{*}{\large{Descripci\'on}}\\ &\\ \hline\\\vspace*{0.5cm}
$nlaser$ & N\'umero de l\'aseres acoplados al \'atomo.\\
$omega(i)$ & Amplitud fija del l\'aser $i$.\\\\
$sigma(i)$ & Varianza de gaussiana de la amplitud del l\'aser $i$ en el tiempo.\\\\
$nmod\equiv \nmod$ & Equivalente al n\'umero de modos acoplados al \'atomo.\\\\
$mod$ & Arreglo de enteros. mod(i) indica la posici\'on del modo i en la composici\'on de espacios.\\\\
$g$ & Arreglo de reales. $g(i)$ es la constante de acoplamiento del modo $i$ al \'atomo.\\\\
$v$ & Arreglo de reales. $v(i)$ es el cambio de fase del l\'aser $i$ determinadas a segundo orden,\\
    & \'vease \cite{robert}.\\\\
$exprAt$ & Arreglo de cadenas de caract\'eres. Especifica el tipo de expresiones at\'omicas involucra-\\
	&das en (\ref{limbladiano}).\\\\
$exprMod$ & Arreglo de cadenas de caract\'eres. Especifica el tipo de expresiones de los modos involucra-\\
	&das en (\ref{limbladiano}).\\\\
$gammaAt$ & Arreglo de reales. $gammaAt(i)$ es el coeficiente que pondera a $exprAt(i)$ \\
	  & en (\ref{limbladiano}).
\end{tabular*} 
\subsubsection{Error Relativo}
\quad Empezaremos fijando la atenci\'on en la L\'inea \ref{algo:main_int_err:linea:error} del Algoritmo \ref{algo:main_int_err}. Vemos que lo hace falta un algoritmo que calcule la norma de Frobenius de una matriz utilizando el resultado (\ref{ec:frob_final}). As\'i, escribimos:

\begin{algorithm}[H]
\caption{normFrob}\label{algo:norma_frob}
\LinesNumbered
\KwIn{$\rho$}
\KwOut{$error$}
$error=0$\;
\Begin{\For{$i=1$ hasta $\dimT$}{
    \For{$j=i+1$ hasta $\dimT$}{
      $error=error+2\cdot rho[i,\,j]\cdot rho[i,j]^*$\;
    }
    $error=error+rho[i,\,i]\cdot rho[i,i]^*$;
  }
}
\end{algorithm}
\quad Donde hemos usado el hecho que la matriz es Herm\'itica, por tanto basta sumar sobre la diagonal y \emph{dos veces} la matriz triangular superior.
\subsubsection{Conmutador de Hamiltoniano y $\rho$}
\quad En este trabajo se har\'a uso del Punto de Vista de Interacci\'on (o \emph{interaction picture}), marco te\'orico en el cual s\'olo es considerado el Hamiltoniano de Interacci\'on (\ref{ec:HI}), para mayor informaci\'on acerca de esta transformaci\'on de las ecuaciones el lector puede referirse a \cite{carmichael} secci\'on 1.2. Observando entonces (\ref{ec:HI}), es inmediato que dos posibles estrategias aparecen. La primera, calcular el hamiltoniano en el momento $t$ y luego hacer $(H\,\rho-\rho\,H)$. La segunda, y que result\'o ser, como se esperaba la de mayor eficiencia, es hacer uso de los algoritmos, que computan la acci\'on de cada uno de los operadores, que componen (\ref{ec:HI}), sobre $\rho$, tanto por la izquierda como por la derecha. Realizamos a continuaci\'on el algoritmo de esta segunda forma de c\'omputo:

\begin{algorithm}[H]
\caption{conmuH}\label{algo:conmuH}
\LinesNumbered
\KwIn{$\rho,\,t,\,z$}
\KwOut{$z=z+i\cdot[H(t),\,\rho]$}
\Begin{
$c=ceros(\dimT)$\;
\CommentSty{ // parte acoplamiento l\'aser-\'atomo, suponemos un pulso gaussiano}\;
\For{$i=1$ hasta $nlaser$\label{algo:conmuH_linea_inicial_loop_At}}{ 
  $Aomega=omega[i]*\exp(-(t-tm)^2/sigma[i]^2)$\;
\CommentSty{ // multiplicaci\'on por la derecha}\;
  $c=c+Aomega\cdot \exp\left(-i\cdot v(i)t\right)\cdot \mbox{arAt}(\rho,\,intAt(i),\,\mbox{'n'})$\;
\CommentSty{ // multiplicaci\'on por la derecha conjugada}\;
  $c=c+Aomega\cdot \exp\left(i\cdot v(i)t\right)\cdot \mbox{arAt}(\rho,\,intAt(i),\,\mbox{'c'})$\;
\CommentSty{ // multiplicaci\'on por la izquierda}\;
  $c=c-Aomega\cdot \exp\left(-i\cdot v(i)t\right)\cdot \mbox{raAt}(\rho,\,intAt(i),\,\mbox{'n'})$\;
\CommentSty{ // multiplicaci\'on por la izquierda conjugada}\;
  $c=c-Aomega\cdot \exp\left(i\cdot v(i)t\right)\cdot \mbox{raAt}(\rho,\,intAt(i),\,\mbox{'c'})$\;
  }\label{algo:conmuH_linea_final_loop_At}
\CommentSty{ // parte acoplamiento modo-\'atomo}\;
\For{$i=1$ hasta $nmod$\label{algo:conmuH_linea_inicial_loop_AM}}{
\CommentSty{ // multiplicaci\'on por la derecha}\;
  $aux=\mbox{arMod}(\rho,\,mod(i),\,\mbox{'n'})$\;
  $c=c+g(i)\cdot\mbox{arAt}(aux,\,acop(i),\,\mbox{'n'})$\;
\CommentSty{ // multiplicaci\'on por la derecha conjugada}\;
  $aux=g(i)\cdot\mbox{arAt}(\rho,\,acop(i),\,\mbox{'c'})$\;
  $c=c+\mbox{arMod}(aux,\,mod(i),\,\mbox{'c'})$\;
\CommentSty{ // multiplicaci\'on por la izquierda}\;
  $aux=g(i)\cdot\mbox{raAt}(\rho,\,acop(i),\,\mbox{'n'})$\;
  $c=c-\mbox{raMod}(aux,\,mod(i),\,\mbox{'n'})$\;
\CommentSty{ // multiplicaci\'on por la izquierda conjugada}\;
  $aux=\mbox{raMod}(\rho,\,mod(i),\,\mbox{'c'})$\;
  $c=c+g(i)\cdot\mbox{raAt}(aux,\,acop(i),\,\mbox{'c'})$\;
  }\label{algo:conmuH_linea_final_loop_AM}
  $z=z+i\cdot c$\;
}
\end{algorithm}

\quad En el Algoritmo \ref{algo:conmuH} hemos inclu\'ido una leve variante a los algoritmos arMod arAt y sus variantes, \'esta consiste en indicar mediante un par\'ametro, 'n' o 'c', si el operador involucrado debe ser conjugado o no, de esta forma ahorramos unas l\'ines de c\'odigo, siendo f\'acil implementar dichas modificaciones en los Algoritmos respectivos.

\subsubsection{Limbladiano} \quad Escribiremos una forma general de c\'omputo del Limbladiano a partir de (\ref{limbladiano}).

\begin{algorithm}[H]
\caption{limbladiano}\label{algo:limbladiano2}
\LinesNumbered
\KwIn{$\rho,\,l$}
\KwOut{$l=l+\mbox{Limbladiano}(\rho)$}
\Begin{
\CommentSty{ // En este trabajo}\;
\CommentSty{ // $exprAt=$'arat', 'atar', 'rata'}\;
\CommentSty{ // $exprMod=$'arat', 'atar', 'rata', 'atra', 'raat'}\;
  \For{$i=1$ hasta $niniterAt$\label{algo:limb2_linea_inicial_loop_At}}{
    $l=l+\mbox{araAt2}(\rho,\,intAt(i),\,gammaAt(i),\,exprAt)$\;
  }\label{algo:limb2_linea_final_loop_At}
  \For{$i=1$ hasta $nmod$\label{algo:limb2_linea_inicial_loop_mod}}{
    $l=l+\mbox{araMod2}(\rho,\,mod(i),\,gammaMod(i),\,exprMod)$\;
  }\label{algo:limb2_linea_final_loop_mod}
}
\end{algorithm}

\quad En el Algoritmo \ref{algo:limbladiano2} hemos utilizado las dos funciones auxiliares araAt2 y araMod2, s\'olo es necesario especificas que \'estas funciones seleccionan la rutina a ejecutar, de las vistas en (\ref{sec:computo_eficiente_operadores}) a partir de los arreglos de expresiones $exprAt$ y $exprMod$, multiplicando cada resultado por los coeficientes asignados a los arreglos $gammaAt$ y $gammaMod$ respectivamente. %repr.operadores (sparse), c�lculo paralelo de rho's

%%TERCERA PARTE : RESULTADOS Y CONCLUSIONES
\pagestyle{empty}%paginas en blanco
\begin{center}
 \part*{Resultados y conclusiones}
\end{center}

\Pag{Resultados}
\sectionm{Resultados}
\subsection{Plataforma Ejecuci\'on}
\quad Describiremos a continuaci\'on, las principales caracter\'isticas del equipo utilizado para la ejecuci\'on del programa.

\begin{description}
 \item[Nombre Servidor y ubicaci\'on : ] Kudi. C.E.M.C.C. Universidad de La Frontera. Temuco.
\item[Procesador : ] Intel XEON CPU, E5506. Velocidad reloj 2.13GHZ.
\item[N\textdegree{} N\'ucleos :] 4 n\'ucleos, 8 Hilos de procesamiento.
\item[Memoria RAM : ] 16GB.
\item[Cach\'e L1 : ] 32KB por N\'ucleo.
\item[Cach\'e L2 : ] 256KB por N\'ucleo.
\item[Cach\'e L3 : ] 4MB compartida.
\item[S.O. : ] GNU/Linux, distribuci\'on Fedora Core 13, Kernel 2.6.34.7-66.fc13.x86\_64.
\end{description}

\subsection{Resultados}

\subsubsection{Resultados anteriores}
En las figuras \ref{fig:niv_at_012}, \ref{fig:niv_at_34} y \ref{fig:modos_ab} se presentan la evoluci\'on del sistema obtenida en (\cite{gino} para el caso $\dimA=3$, $\dimB=3$ sin ruido t\'ermico.

%%%%%%%%%%%%%%%%%%%%%%%%%%%%%%%%%%%%%%%% RESULTADOS GINO %%%%%%%%%%%%%%%%%%%%%%%%%%%%%%%%%%%%%%%%%
\begin{figure}[!ht]
\centering
 \includegraphics[scale=0.75]{cap5/imagen_gino_p123.eps}\caption{Evoluci\'on de niveles at\'omicos 0, 1 y 2 en trabajo anterior.}\label{fig:niv_at_012}
 \includegraphics[scale=0.75]{cap5/imagen_gino_p45.eps}\caption{Evoluci\'on de niveles at\'omicos 3 y 4 en trabajo anterior.}\label{fig:niv_at_34}
\end{figure}
\begin{figure}[!ht]
\centering
 \includegraphics[scale=0.75]{cap5/imagen_gino_pAB.eps}\caption{Evoluci\'on poblacional modos $a$ y $b$ en trabajo anterior.}\label{fig:modos_ab}
\end{figure}

\subsubsection{Resultados niveles at\'omicos y poblaciones de los modos}
\quad Ahora, procedemos a presentar los resultados, en forma gr\'afica, de las curvas resultantes de la evoluci\'on del sistema para diferentes niveles de ruido, esto es, alrededor de $\bar{n}=0\ldotp 01$ que corresponde al ruido t\'ermico encontrado en la literatura, para temperaturas del orden $[\mu K]$.
\clearpage
%%%%%%%%%%%%%%%%%%%%%%%%%%%%%%%%%%%%%%%% RESULTADOS TRABAJO Niv 0, 1 y 2 %%%%%%%%%%%%%%%%%%%%%%%%%%%%%%%%%%%%%%%%%
\begin{figure}[ht]
\centering
\includegraphics[scale=0.75]{cap5/imagen_4_0.001_p123.eps}\caption{Evoluci\'on niveles at\'omicos 0, 1 y 2, $\bar{n}=0\ldotp 001$.}\label{fig:niveles_123_4_0001}
\includegraphics[scale=0.75]{cap5/imagen_5_0.005_p123.eps}\caption{Evoluci\'on niveles at\'omicos 0, 1 y 2, $\bar{n}=0\ldotp 005$.}\label{fig:niveles_123_5_0005}
\end{figure}
\begin{figure}[ht]
\centering
\includegraphics[scale=0.75]{cap5/imagen_6_0.01_p123.eps}\caption{Evoluci\'on niveles at\'omicos 0, 1 y 2, $\bar{n}=0\ldotp 01$.}\label{fig:niveles_123_6_001}
\includegraphics[scale=0.75]{cap5/imagen_7_0.05_p123.eps}\caption{Evoluci\'on niveles at\'omicos 0, 1 y 2, $\bar{n}=0\ldotp 05$.}\label{fig:niveles_123_7_005}
\end{figure}
\begin{figure}[ht]
\centering
\includegraphics[scale=0.75]{cap5/imagen_10_0.1_p123.eps}\caption{Evoluci\'on niveles at\'omicos 0, 1 y 2, $\bar{n}=0\ldotp 1$.}\label{fig:niveles_123_10_01}
\includegraphics[scale=0.75]{cap5/imagen_13_0.5_p123.eps}\caption{Evoluci\'on niveles at\'omicos 0, 1 y 2, $\bar{n}=0\ldotp 5$.}\label{fig:niveles_123_13_05}
\end{figure}
%%%%%%%%%%%%%%%%%%%%%%%%%%%%%%%%%%%%%%%% RESULTADOS TRABAJO Niv 0, 1 y 2 %%%%%%%%%%%%%%%%%%%%%%%%%%%%%%%%%%%%%%%%%


%%%%%%%%%%%%%%%%%%%%%%%%%%%%%%%%%%%%%%%% RESULTADOS TRABAJO Niv 3 y 4 %%%%%%%%%%%%%%%%%%%%%%%%%%%%%%%%%%%%%%%%%
\begin{figure}[ht]
\centering
\includegraphics[scale=0.75]{cap5/imagen_4_0.001_p45.eps}\caption{Evoluci\'on niveles at\'omicos 3 y 4, $\bar{n}=0\ldotp 001$.}\label{fig:niveles_45_4_0001}
\end{figure}
\begin{figure}[ht]
\centering
\includegraphics[scale=0.75]{cap5/imagen_5_0.005_p45.eps}\caption{Evoluci\'on niveles at\'omicos 3 y 4, $\bar{n}=0\ldotp 005$.}\label{fig:niveles_45_5_0005}
\end{figure}
\begin{figure}[ht]
\centering
\includegraphics[scale=0.75]{cap5/imagen_6_0.01_p45.eps}\caption{Evoluci\'on niveles at\'omicos 3 y 4, $\bar{n}=0\ldotp 01$.}\label{fig:niveles_45_6_001}
\includegraphics[scale=0.75]{cap5/imagen_7_0.05_p45.eps}\caption{Evoluci\'on niveles at\'omicos 3 y 4, $\bar{n}=0\ldotp 05$.}\label{fig:niveles_45_7_005}
\end{figure}
\begin{figure}[ht]
\centering
\includegraphics[scale=0.75]{cap5/imagen_10_0.1_p45.eps}\caption{Evoluci\'on niveles at\'omicos 3 y 4, $\bar{n}=0\ldotp 1$.}\label{fig:niveles_45_10_01}
\includegraphics[scale=0.75]{cap5/imagen_13_0.5_p45.eps}\caption{Evoluci\'on niveles at\'omicos 3 y 4, $\bar{n}=0\ldotp 5$.}\label{fig:niveles_45_13_05}
\end{figure}
%%%%%%%%%%%%%%%%%%%%%%%%%%%%%%%%%%%%%%%% RESULTADOS TRABAJO Niv 3 y 4 %%%%%%%%%%%%%%%%%%%%%%%%%%%%%%%%%%%%%%%%%


%%%%%%%%%%%%%%%%%%%%%%%%%%%%%%%%%%%%%%%% RESULTADOS TRABAJO <nA> <nB> %%%%%%%%%%%%%%%%%%%%%%%%%%%%%%%%%%%%%%%%%
\begin{figure}[ht]
\centering
 \includegraphics[scale=0.75]{cap5/imagen_4_0.001_pAB.eps}\caption{Evoluci\'on poblacional modos $a$ y $b$, $\bar{n}=0\ldotp 001$.}\label{fig:modos_ab_4_0001}
 \includegraphics[scale=0.75]{cap5/imagen_5_0.005_pAB.eps}\caption{Evoluci\'on poblacional modos $a$ y $b$, $\bar{n}=0\ldotp 005$.}\label{fig:modos_ab_5_0005}
\end{figure}
\begin{figure}[ht]
\centering
 \includegraphics[scale=0.75]{cap5/imagen_6_0.01_pAB.eps}\caption{Evoluci\'on poblacional modos $a$ y $b$, $\bar{n}=0\ldotp 01$.}\label{fig:modos_ab_6_001}
 \includegraphics[scale=0.75]{cap5/imagen_7_0.05_pAB.eps}\caption{Evoluci\'on poblacional modos $a$ y $b$, $\bar{n}=0\ldotp 05$.}\label{fig:modos_ab_7_005}
\end{figure}
\begin{figure}[ht]
\centering
 \includegraphics[scale=0.75]{cap5/imagen_10_0.1_pAB.eps}\caption{Evoluci\'on poblacional modos $a$ y $b$, $\bar{n}=0\ldotp 1$.}\label{fig:modos_ab_10_01}
 \includegraphics[scale=0.75]{cap5/imagen_13_0.5_pAB.eps}\caption{Evoluci\'on poblacional modos $a$ y $b$, $\bar{n}=0\ldotp 5$.}\label{fig:modos_ab_13_05}
\end{figure}
%%%%%%%%%%%%%%%%%%%%%%%%%%%%%%%%%%%%%%%% RESULTADOS TRABAJO <nA> <nB> %%%%%%%%%%%%%%%%%%%%%%%%%%%%%%%%%%%%%%%%%
\clearpage
\subsubsection{Espectros obtenidos}
\qquad Ahora presentamos los espectros (normalizados) para las curvas de poblaci\'on de ambos modos, $a$ y $b$ del campo. Se aprecia c\'omo a medida que el ruido aumenta es m\'as dif\'icil poder discriminar si ocurri\'o la emisi\'on de un fot\'on o s\'olo hay ruido en la cavidad, para esto comp\'arese la respectiva curva con la presentada en la Figura \ref{fig:espectro_ruidopuro}, que corresponde al espectro de un ruido puro. En las figuras, $DFT$ denota que la curva presentada corresponde a la Transformada Discreta de Fourier (Discrete Fourier Transform).
%%%%%%%%%%%%%%%%%%%%%%%%%%%%%%%%%%%%%%% RESULTADOS TRABAJO espectros nbar=0 (gino) %%%%%%%%%%%%%%%%%%%%%%%%
\begin{figure}[ht]
\hspace*{-1.6cm}
\begin{minipage}{0.52 \linewidth}
\centering
 \includegraphics[scale=0.6]{cap5/imagen_2_0.0_fft_nA.eps}
\caption{Espectro modo $a$, trabajo anterior sin riudo t\'ermico.}\label{fig:espectro_2_0}
\end{minipage}
\hspace*{1.5cm}
\begin{minipage}{0.52 \linewidth}
\centering
 \includegraphics[scale=0.6]{cap5/imagen_2_0.0_fft_nB.eps}
\caption{Espectro modo $b$, trabajo anterior sin riudo t\'ermico.}\label{fig:espectro_2_0}
\end{minipage}
\end{figure}
%%%%%%%%%%%%%%%%%%%%%%%%%%%%%%%%%%%%%%%% RESULTADOS TRABAJO espectros nbar=0 (gino) %%%%%%%%%%%%%%%%%%%%%%%%
\begin{figure}[ht]
\centering
\begin{minipage}{0.52 \linewidth}
\centering
 \includegraphics[scale=0.6]{cap5/imagen_espectro_ruidopuro.eps}
\caption{Espectro ruido puro normalizado.}\label{fig:espectro_ruidopuro}
\end{minipage}
\end{figure}
%%%%%%%%%%%%%%%%%%%%%%%%%%%%%%%%%%%%%%%% RESULTADOS TRABAJO espectro nbar=0.001 modos a y b%%%%%%%%%%%%%%%%%%%%%%%%
\begin{figure}[ht]
\hspace*{-1.6cm}
\begin{minipage}{0.52 \linewidth}
\centering
 \includegraphics[scale=0.6]{cap5/imagen_4_0.001_fft_nA.eps}
\caption{Espectro modo $a$, $\bar{n}=0\ldotp 001$.}\label{fig:espectro_4_0001}
\end{minipage}
\hspace*{1.5cm}
\begin{minipage}{0.52 \linewidth}
\centering
 \includegraphics[scale=0.6]{cap5/imagen_4_0.001_fft_nB.eps}
\caption{Espectro modo $b$, $\bar{n}=0\ldotp 001$.}\label{fig:espectro_4_0001}
\end{minipage}
\end{figure}
%%%%%%%%%%%%%%%%%%%%%%%%%%%%%%%%%%%%%%%% RESULTADOS TRABAJO espectro nbar=0.001 modos a y b%%%%%%%%%%%%%%%%%%%%%%%%
%%%%%%%%%%%%%%%%%%%%%%%%%%%%%%%%%%%%%%%% RESULTADOS TRABAJO espectro nbar=0.005 modos a y b%%%%%%%%%%%%%%%%%%%%%%%%
\begin{figure}[ht]
\hspace*{-1.6cm}
\begin{minipage}{0.52 \linewidth}
\centering
 \includegraphics[scale=0.6]{cap5/imagen_5_0.005_fft_nA.eps}
\caption{Espectro modo $a$, $\bar{n}=0\ldotp 005$.}\label{fig:espectro_5_0005}
\end{minipage}
\hspace*{1.5cm}
\begin{minipage}{0.52 \linewidth}
\centering
 \includegraphics[scale=0.6]{cap5/imagen_5_0.005_fft_nB.eps}
\caption{Espectro modo $b$, $\bar{n}=0\ldotp 005$.}\label{fig:espectro_5_0005}
\end{minipage}
\end{figure}
%%%%%%%%%%%%%%%%%%%%%%%%%%%%%%%%%%%%%%%% RESULTADOS TRABAJO espectro nbar=0.005 modos a y b%%%%%%%%%%%%%%%%%%%%%%%%
%%%%%%%%%%%%%%%%%%%%%%%%%%%%%%%%%%%%%%%% RESULTADOS TRABAJO espectro nbar=0.01 modos a y b%%%%%%%%%%%%%%%%%%%%%%%%
\begin{figure}[ht]
\hspace*{-1.6cm}
\begin{minipage}{0.52 \linewidth}
\centering
 \includegraphics[scale=0.6]{cap5/imagen_6_0.01_fft_nA.eps}
\caption{Espectro modo $a$, $\bar{n}=0\ldotp 01$.}\label{fig:espectro_6_001}
\end{minipage}
\hspace*{1.5cm}
\begin{minipage}{0.52 \linewidth}
\centering
 \includegraphics[scale=0.6]{cap5/imagen_6_0.01_fft_nB.eps}
\caption{Espectro modo $b$, $\bar{n}=0\ldotp 01$.}\label{fig:espectro_6_001}
\end{minipage}
\end{figure}
% %%%%%%%%%%%%%%%%%%%%%%%%%%%%%%%%%%%%%%%% RESULTADOS TRABAJO espectro nbar=0.01 modos a y b%%%%%%%%%%%%%%%%%%%%%%%%
%%%%%%%%%%%%%%%%%%%%%%%%%%%%%%%%%%%%%%%% RESULTADOS TRABAJO espectro nbar=0.05 modos a y b%%%%%%%%%%%%%%%%%%%%%%%%
\begin{figure}[ht]
\hspace*{-1.6cm}
\begin{minipage}{0.52 \linewidth}
\centering
 \includegraphics[scale=0.6]{cap5/imagen_7_0.05_fft_nA.eps}
\caption{Espectro modo $a$, $\bar{n}=0\ldotp 05$.}\label{fig:espectro_7_005}
\end{minipage}
\hspace*{1.5cm}
\begin{minipage}{0.52 \linewidth}
\centering
 \includegraphics[scale=0.6]{cap5/imagen_7_0.05_fft_nB.eps}
\caption{Espectro modo $b$, $\bar{n}=0\ldotp 05$.}\label{fig:espectro_7_005}
\end{minipage}
\end{figure}
%%%%%%%%%%%%%%%%%%%%%%%%%%%%%%%%%%%%%%%% RESULTADOS TRABAJO espectro nbar=0.05 modos a y b%%%%%%%%%%%%%%%%%%%%%%%%
%%%%%%%%%%%%%%%%%%%%%%%%%%%%%%%%%%%%%%%% RESULTADOS TRABAJO espectro nbar=0.1 modos a y b%%%%%%%%%%%%%%%%%%%%%%%%
\begin{figure}[ht]
\hspace*{-1.6cm}
\begin{minipage}{0.52 \linewidth}
\centering
 \includegraphics[scale=0.6]{cap5/imagen_10_0.1_fft_nA.eps}
\caption{Espectro modo $a$, $\bar{n}=0\ldotp 1$.}\label{fig:espectro_10_01}
\end{minipage}
\hspace*{1.5cm}
\begin{minipage}{0.52 \linewidth}
\centering
 \includegraphics[scale=0.6]{cap5/imagen_10_0.1_fft_nB.eps}
\caption{Espectro modo $b$, $\bar{n}=0\ldotp 1$.}\label{fig:espectro_10_01}
\end{minipage}
\end{figure}
%%%%%%%%%%%%%%%%%%%%%%%%%%%%%%%%%%%%%%% RESULTADOS TRABAJO espectro nbar=0.1 modos a y b%%%%%%%%%%%%%%%%%%%%%%%%
% %%%%%%%%%%%%%%%%%%%%%%%%%%%%%%%%%%%%%%% RESULTADOS TRABAJO espectro nbar=0.5 modos a y b%%%%%%%%%%%%%%%%%%%%%%%%
\clearpage
\begin{figure}[h]
\hspace*{-1.6cm}
\begin{minipage}{0.52 \linewidth}
\centering
 \includegraphics[scale=0.6]{cap5/imagen_13_0.5_fft_nA.eps}
\caption{Espectro modo $a$, $\bar{n}=0\ldotp 5$.}\label{fig:espectro_13_05}
\end{minipage}
\hspace*{1.5cm}
\begin{minipage}{0.52 \linewidth}
\centering
 \includegraphics[scale=0.6]{cap5/imagen_13_0.5_fft_nB.eps}
\caption{Espectro modo $b$, $\bar{n}=0\ldotp 5$.}\label{fig:espectro_13_05}
\end{minipage}
\end{figure}
%%%%%%%%%%%%%%%%%%%%%%%%%%%%%%%%%%%%%%% RESULTADOS TRABAJO espectro nbar=0.5 modos a y b%%%%%%%%%%%%%%%%%%%%%%%%

\subsubsection{Tiempos de ejecuci\'on} \qquad En la Tabla \ref{tabla:tiempos_modos} presentamos los tiempos\footnote{En esta secci\'on, los intervalos de tiempo ser\'an denotados seg\'un ``DDdHHhMMmSSs``, donde DD es la cantidad de d\'ias, HH horas, MM minutos y SS segundos.} de ejecuci\'on resultantes para diferentes truncamientos de los espacios asociados a los modos del campo, tomando siempre ($\dimA=\dimB=2\ldots 10)$.\\
% \begin{table}[h]
% \centering
% \begin{tabular}{|c|c|c|c|c|c|c|c|c|c|}
% \hline &\multicolumn{9}{|c|}{$\dimA=\dimB$}\\
% \hline $\bar{n}$& 2&3&4&5&6&7&8&9&10\\
% \hline $0\ldotp001$& $18$ &$69$ & $229$ & $562$ & $1135$ & $2532$ & $5792$ & $19090$ & $28813$\\
% \hline $0\ldotp 005$& $18$ &$67$ & $236$ & $594$ & $1328$ & $2260$ & $10178$ & $10296$ & $27449$\\
% \hline $0\ldotp 01$& $20$ &$75$ & $228$ & $581$ & $1148$ & $1958$ & $7800$ & $14641$ & $19991$\\
% \hline $0\ldotp 05$& $22$ &$79$ & $261$ & $593$ & $1272$ & $2556$ & $17517$ & $20969$ & $32860$\\
% \hline
% \end{tabular}\caption{Tiempo (seg) de ejecuci\'on para diferentes dimensiones de los espacios modales.}\label{tabla:tiempos_modos}
% \end{table}
\quad En las siguientes tables, denotaremos por $R_f$ el Algoritmo \ref{algo:main_int_f} asociado al c\'alculo de la funci\'on objetivo a integrar; $R_e$ denotar\'a al Algoritmo \ref{algo:main_int_err}, asociado al c\'alculo del error relativo y $R_c$ que denotar\'a al Algoritmo \ref{algo:main_int_calculos} asociado al c\'alculo de las cantidades f\'isicas a medir. En la Tabla \ref{tabla:tiempos_gino}, mostramos los tiempos de ejecuci\'on, del Programa original en FORTRAN. Este programa corresponde a la \emph{traducci\'on literal} del programa utilizado en \cite{gino} y escrito en MATLAB, para la resoluci\'on del problema del presente trabajo, con $\dimA=\dimB=2$ y sin ruido t\'ermico. Adem\'as se muestran tambi\'en las proporciones (en porcentaje) de los tiempos de demora de cada una de las rutinas principales definidas anteriormente. Tambi\'en, en la Tabla \ref{tabla:cantidad_memoria_rho} se muestra la cantidad de memoria utilizada para almacenar $\rho$, el color azul, verde y rojo, indican que para el tama\~no de los espacios modales asociado, la matriz $\rho$, el programa entero, se pueden almacenar totalmente en la cach\'e L1, L2 o L3 respectivamente.\\

\quad Adem\'as, en la Tabla \ref{tabla:numero_correcciones}, fijando $\dimA=\dimB=7$ y haciendo variar $\bar{n}$, se observa c\'omo var\'ia el n\'umero de iteraciones que realiza el m\'etodo corrector en la integraci\'on.

\begin{table}[h]
\centering
\begin{tabular}{|c|c|c|c|c|c|c|c|c|c|c|c|c|c|}
\hline $\dimA=\dimB$ & 2 &3 & 4& 5& 6&7 & 8&9 &10&11&12\\
\hline$\rho$ &\colorbox{green}{16}&\colorbox{blue}{50}&\colorbox{blue}{122}&\colorbox{blue}{253}&\colorbox{red}{469}&\colorbox{red}{800}&    \colorbox{red}{1281}&\colorbox{red}{1953}&\colorbox{red}{2860}&\colorbox{red}{4050}&    5578\\
\hline Total &\colorbox{red}{1012}&    \colorbox{red}{1276}&    \colorbox{red}{1796}&    \colorbox{red}{2596}&    \colorbox{red}{3868}&    5784&    8640&    12288&    17408&    24576&    33792\\
\hline
\end{tabular} \caption{Memoria utilizada por la matriz $\rho$ en y el programa completo en KB. }\label{tabla:cantidad_memoria_rho}
\end{table}

\begin{table}[h]
\centering
\begin{tabular}{|c|c|c|c|c|c|c|c|c|c|c|}
\hline &\multicolumn{9}{|c|}{$\dimA=\dimB$}&\\
\hline $\bar{n}$& 2&3&4&5&6&7&8&9&10&Total\\
\hline $0\ldotp001$& 17s &1m02s & \colorbox{red}{3m01s} & 6m54s & 15m40s & 35m53s& 1h31m & 2h03m & 4h16m &8h53m\\
\hline $0\ldotp 005$&18s &1m03s & 3m04s & \colorbox{red}{6m56s} & 15m26s & 36m16s & 1h38m & 2h51m & 4h34m & 10h06m\\
\hline $0\ldotp 01$& 19s &1m10s & 3m21s & 7m45s & \colorbox{red}{15m31s} & 38m00s & 1h35m & 2h54m & 5h02m&10h37m\\
\hline $0\ldotp 05$& 20s &1m11s & 3m25s & 7m41s & 15m36s & \colorbox{red}{45m40s} & 1h59m & 3h09m & 4h44m&11h06m\\
 \hline $0\ldotp 1$& 20s &1m14s & 3m32s & 8m02s & 16m58s & 46m09s & 1h41m & 3h58m & \colorbox{red}{5h55m}&12h51m\\
 \hline $0\ldotp 5$& 22s &1m20s & 3m51s & 8m55s & 18m42s & 50m27s & 2h10m & 4h06m & 6h16m&13h55m\\
\hline
\end{tabular}\caption{Tiempos de ejecuci\'on para diferentes dimensiones de los espacios modales.}\label{tabla:tiempos_modos}
\end{table}

\begin{table}[h]
 \centering
\begin{tabular}{|c|c|c|c|c|c|c|c|c|}
\hline $\dimA=\dimB$ & 2 & 3 & 4& 5& 6&7 & 8&9 \\
\hline Tiempo & 24m & 2h18m & 8h & 1d & 3d9h&14d19h & 21d & 38d \\
\hline $R_f$ &$90\ldotp 6\%$ & $92\ldotp 4\%$ &$91\ldotp 1\%$ &$91\ldotp 3\%$ &$94\ldotp 0\%$ &$93\ldotp 8\%$ &$94\ldotp 2\%$ & $88\ldotp 9\%$ \\
\hline $R_e$ &$3\ldotp 9\%$ &$3\ldotp 7\%$ &$3\ldotp 7\%$ &$3\ldotp 8\%$ &$2\ldotp 2\%$ &$2\ldotp 6\%$ &$2\ldotp 1\%$ & $3\ldotp 6\%$\\
\hline $R_c$ &$4\ldotp 7\%$ & $4\ldotp 6\%$&$4\ldotp 6\%$ &$4\ldotp 8\%$ &$3\ldotp 4\%$ &$3\ldotp 6\%$ &$3\ldotp 2\%$ &$7\ldotp 3\%$ \\
\hline
\end{tabular}\caption{Tiempos de ejecuci\'on y porcentajes por rutina del trabajo anterior \cite{gino}.}\label{tabla:tiempos_gino}
\end{table}
\begin{table}[h]
 \centering
\begin{tabular}{|c|c|c|c|c|c|c|c|c|}
\hline $\dimA=\dimB$ & 2 & 3 & 4& 5& 6&7 & 8&9 \\
\hline Tiempo & 12s & 37s & 1m48s & 4m03s & 10m8s& 21m26s & 52m44s & 1h42m \\
\hline $R_f$ &$86\ldotp 6\%$ & $89\ldotp 5\%$ &$90\ldotp 2\%$ &$89\ldotp 8\%$ &$85\ldotp 5\%$ &$84\ldotp 8\%$ &$85\ldotp 8\%$ & $87\ldotp 7\%$ \\
\hline $R_e$ &$4\ldotp 0\%$ &$3\ldotp 0\%$ &$3\ldotp 2\%$ &$3\ldotp 2\%$ &$4\ldotp 8\%$ &$4\ldotp 8\%$ &$3\ldotp 8\%$ & $2\ldotp 8\%$\\
\hline $R_c$ &$3\ldotp 8\%$ & $3\ldotp 0\%$&$2\ldotp 8\%$ &$2\ldotp 8\%$ &$3\ldotp 0\%$ &$3\ldotp 3\%$ &$4\ldotp 6\%$ &$5\ldotp 0\%$ \\
\hline
\end{tabular}\caption{Tiempos de ejecuci\'on y porcentajes por rutina del trabajo actual sin ruido t\'ermico.}\label{tabla:tiempos_actual_sinruido}
\end{table}

\begin{table}[h]
 \centering
\begin{tabular}{|c|c|c|c|c|c|c|c|}
\hline &\multicolumn{7}{|c|}{$\bar{n}$}\\
\hline $n$ & $0\ldotp0$ &$0\ldotp001$ &$0\ldotp005$ &$0\ldotp01$ &$0\ldotp05$ &$0\ldotp1$ &$0\ldotp5$\\
\hline 0 & 3654 & 0 & 0 & 0 & 0 &0 &0 \\
\hline 1 & 4513 & 0 & 0 & 0 & 0 &0 &0 \\
\hline 2 & 3853 & 11896 & 10974 & 0 & 0 &0 &0 \\
\hline 3 & 3627 & 3743 & 4630 & 15560 & 14910 &10906 &1492\\
\hline 4 & 4353 & 4361 & 4396 & 4440 & 5090 &9094 &17573 \\
\hline 5 & 0 & 0 & 0 & 0 & 0 &0 &935 \\
\hline
\end{tabular}\caption{N\'umero de correcciones $n$, en el m\'etodo de integraci\'on, $\dimA=\dimB=7$.}\label{tabla:numero_correcciones}
\end{table}
\clearpage
\subsubsection{Proporciones de demora los Algoritmos}\label{sec:prop_demora_algoritmos}

\qquad Ahora, otros datos de inter\'es son las proporciones, sobre el tiempo total, de demora de cada subrutina importante identificada en (\ref{sec:codigos_generales}). Resumimos \'estas proporciones de tiempo de ejecuci\'on en la Tabla \ref{tabla:tiempos_prop_por_rutina}.\\

\begin{table}[h]
\centering
\begin{tabular}{|c|c|c|c|c|c|c|c|c|c|c|}
\hline \multicolumn{11}{|c|}{$\bar{n}=0\ldotp 001$}\\
\hline $\dimA=\dimB$ & 2&3&4&5&6&7&8&9&10&Prom.\\
\hline $R_f$ & $ 90\ldotp 2$ & $ 92\ldotp 7$ & $ 93\ldotp 6$ & $ 93\ldotp 8$ & $ 91\ldotp 2$ & $ 89\ldotp 1$ & $ 89\ldotp 6$ & $ 90\ldotp 7$ & $ 90\ldotp 2$ & $ 91\ldotp 2$\\
\hline $R_e$ & $  3\ldotp 0$ & $  2\ldotp 6$ & $  2\ldotp 2$ & $  2\ldotp 1$ & $  3\ldotp 0$ & $  3\ldotp 4$ & $  2\ldotp 9$ & $  2\ldotp 2$ & $  2\ldotp 3$ & $  2\ldotp 6$\\
\hline $R_c$ & $  2\ldotp 6$ & $  1\ldotp 8$ & $  1\ldotp 7$ & $  1\ldotp 6$ & $  1\ldotp 7$ & $  2\ldotp 5$ & $  2\ldotp 9$ & $  3\ldotp 7$ & $  3\ldotp 7$ & $  2\ldotp 5$\\
\hline
\hline \multicolumn{11}{|c|}{$\bar{n}=0\ldotp 005$}\\
\hline $\dimA=\dimB$ & 2&3&4&5&6&7&8&9&10&Prom.\\
\hline $R_f$ & $ 90\ldotp 5$ & $ 92\ldotp 7$ & $ 93\ldotp 5$ & $ 93\ldotp 8$ & $ 92\ldotp 3$ & $ 89\ldotp 3$ & $ 90\ldotp 0$ & $ 89\ldotp 6$ & $ 90\ldotp 3$ & $ 91\ldotp 3$\\
\hline $R_e$ & $  2\ldotp 7$ & $  2\ldotp 5$ & $  2\ldotp 2$ & $  2\ldotp 1$ & $  2\ldotp 6$ & $  3\ldotp 4$ & $  2\ldotp 8$ & $  2\ldotp 6$ & $  2\ldotp 2$ & $  2\ldotp 6$\\
\hline $R_c$ & $  2\ldotp 5$ & $  1\ldotp 7$ & $  1\ldotp 7$ & $  1\ldotp 5$ & $  1\ldotp 7$ & $  2\ldotp 4$ & $  2\ldotp 7$ & $  3\ldotp 6$ & $  4\ldotp 1$ & $  2\ldotp 4$\\
\hline
\hline \multicolumn{11}{|c|}{$\bar{n}=0\ldotp 01$}\\
\hline $\dimA=\dimB$ & 2&3&4&5&6&7&8&9&10&Prom.\\
\hline $R_f$ & $ 90\ldotp 5$ & $ 92\ldotp 8$ & $ 93\ldotp 7$ & $ 93\ldotp 9$ & $ 94\ldotp 3$ & $ 90\ldotp 0$ & $ 89\ldotp 2$ & $ 90\ldotp 7$ & $ 90\ldotp 3$ & $ 91\ldotp 7$\\
\hline $R_e$ & $  3\ldotp 0$ & $  2\ldotp 6$ & $  2\ldotp 3$ & $  2\ldotp 1$ & $  2\ldotp 0$ & $  3\ldotp 4$ & $  3\ldotp 2$ & $  2\ldotp 5$ & $  2\ldotp 6$ & $  2\ldotp 6$\\
\hline $R_c$ & $  2\ldotp 2$ & $  1\ldotp 6$ & $  1\ldotp 5$ & $  1\ldotp 4$ & $  1\ldotp 5$ & $  2\ldotp 1$ & $  2\ldotp 6$ & $  3\ldotp 1$ & $  3\ldotp 3$ & $  2\ldotp 1$\\
\hline
\hline \multicolumn{11}{|c|}{$\bar{n}=0\ldotp 05$}\\
\hline $\dimA=\dimB$ & 2&3&4&5&6&7&8&9&10&Prom.\\
\hline $R_f$ & $ 90\ldotp 5$ & $ 90\ldotp 8$ & $ 93\ldotp 7$ & $ 94\ldotp 2$ & $ 94\ldotp 3$ & $ 87\ldotp 9$ & $ 89\ldotp 0$ & $ 89\ldotp 8$ & $ 87\ldotp 8$ & $ 90\ldotp 9$\\
\hline $R_e$ & $  3\ldotp 0$ & $  3\ldotp 8$ & $  2\ldotp 3$ & $  2\ldotp 1$ & $  2\ldotp 0$ & $  4\ldotp 2$ & $  3\ldotp 4$ & $  2\ldotp 8$ & $  2\ldotp 3$ & $  2\ldotp 9$\\
\hline $R_c$ & $  1\ldotp 9$ & $  1\ldotp 7$ & $  1\ldotp 4$ & $  1\ldotp 4$ & $  1\ldotp 5$ & $  2\ldotp 1$ & $  2\ldotp 4$ & $  3\ldotp 2$ & $  3\ldotp 2$ & $  2\ldotp 1$\\
\hline
\hline \multicolumn{11}{|c|}{$\bar{n}=0\ldotp 1$}\\
\hline $\dimA=\dimB$ & 2&3&4&5&6&7&8&9&10&Prom.\\
\hline $R_f$ & $ 90\ldotp 5$ & $ 93\ldotp 1$ & $ 93\ldotp 7$ & $ 94\ldotp 0$ & $ 92\ldotp 5$ & $ 89\ldotp 3$ & $ 90\ldotp 7$ & $ 88\ldotp 8$ & $ 87\ldotp 4$ & $ 91\ldotp 1$\\
\hline $R_e$ & $  3\ldotp 0$ & $  2\ldotp 5$ & $  2\ldotp 3$ & $  2\ldotp 2$ & $  2\ldotp 6$ & $  3\ldotp 6$ & $  2\ldotp 9$ & $  3\ldotp 2$ & $  2\ldotp 7$ & $  2\ldotp 8$\\
\hline $R_c$ & $  2\ldotp 1$ & $  1\ldotp 6$ & $  1\ldotp 4$ & $  1\ldotp 3$ & $  1\ldotp 4$ & $  2\ldotp 0$ & $  2\ldotp 4$ & $  2\ldotp 9$ & $  3\ldotp 2$ & $  2\ldotp 0$\\
\hline
\hline \multicolumn{11}{|c|}{$\bar{n}=0\ldotp 5$}\\
\hline $\dimA=\dimB$ & 2&3&4&5&6&7&8&9&10&Prom.\\
\hline $R_f$ & $ 90\ldotp 8$ & $ 92\ldotp 8$ & $ 93\ldotp 8$ & $ 94\ldotp 0$ & $ 92\ldotp 7$ & $ 89\ldotp 9$ & $ 89\ldotp 2$ & $ 88\ldotp 9$ & $ 88\ldotp 3$ & $ 91\ldotp 2$\\
\hline $R_e$ & $  3\ldotp 0$ & $  2\ldotp 9$ & $  2\ldotp 4$ & $  2\ldotp 2$ & $  2\ldotp 7$ & $  3\ldotp 4$ & $  3\ldotp 4$ & $  3\ldotp 3$ & $  2\ldotp 8$ & $  2\ldotp 9$\\
\hline $R_c$ & $  2\ldotp 0$ & $  1\ldotp 4$ & $  1\ldotp 3$ & $  1\ldotp 2$ & $  1\ldotp 3$ & $  2\ldotp 0$ & $  2\ldotp 4$ & $  2\ldotp 6$ & $  2\ldotp 8$ & $  1\ldotp 9$\\
\hline
\end{tabular}\caption{Porcentajes de tiempos de ejecuci\'on para diferentes dimensiones de los espacios modales.}\label{tabla:tiempos_prop_por_rutina}
\end{table}

\clearpage
\qquad En la Tabla \ref{tabla:proporciones_tiempo_F_lim_conmu}: '$R_f$' denota al Algoritmo \ref{algo:main_int_f}, mientras que 'Li', 'CH' denotan a los algoritmos \ref{algo:limbladiano2} y \ref{algo:conmuH} respectivamente, subrutinas de 'F'. 'Md', denota entre las l\'ineas \ref{algo:limb2_linea_inicial_loop_mod} y \ref{algo:limb2_linea_final_loop_mod} del Algoritmo \ref{algo:limbladiano2} involucra s\'olo operadores modales como aratMod; tambi\'en 'At' denota desde la l\'inea \ref{algo:limb2_linea_inicial_loop_At} y \ref{algo:limb2_linea_final_loop_At}, \'estas l\'ineas s\'olo involucran operadores at\'omicos, como aratAt. Seguidamente, 'AM' denota entre las l\'ineas \ref{algo:conmuH_linea_inicial_loop_AM} y \ref{algo:conmuH_linea_final_loop_AM} del Algoritmo \ref{algo:conmuH}, dichas l\'ineas operadores at\'omicos y de modos; 'AtC' denota entre las l\'ineas \ref{algo:conmuH_linea_inicial_loop_At} y \ref{algo:conmuH_linea_final_loop_At}, \'estas l\'ineas involucran \'unicamente operadores At\'omicos. \'Esta tabla contiene los porcentajes de demora de cada una de las secciones mencionadas, comparados sus tiempos versus la demora total del Algoritmo que las contiene.\\

\qquad De la Tabla \ref{tabla:proporciones_tiempo_F_lim_conmu}, podemos concluir, los Algoritmos que involucran operadores de los modos, demoran casi 8 veces en su c\'omputo que los Algoritmos que involucran operadores at\'omicos.

\begin{table}[h]
 \centering
\begin{tabular}{|c|c|c|c|c|c|c|c|}
\hline \multicolumn{2}{|c|}{} &\multicolumn{6}{|c|}{$\dimA=\dimB$}\\
\hline \'Ambito& Algor. & 2&4&6&8&10&12\\
\hline \multirow{2}{*}{$R_f$} 	& Li & 69\% & 75\% & 71\% & 59\% & 55\% & 54\%\\
\cline{2-8} 			& CH & 31\% & 25\% & 29\% & 41\% & 45\% & 46\%\\
\hline \multirow{2}{*}{Li} 	& Md & 93\% & 97\% & 97\% & 96\% & 96\% & 95\%\\
\cline{2-8} 			& At & 7\% & 3\% & 3\% & 4\% & 4\% & 5\%\\
\hline \multirow{2}{*}{CH} 	& AM & 87\% & 89\% & 91\% & 93\% & 93\% & 94\%\\
\cline{2-8} 			& AtC & 13\% & 11\% & 9\% & 7\% & 7\% & 6\%\\
\hline
\end{tabular}\caption{Proporciones de demora sobre el total (del resp. \'ambito)\\ de Algoritmos \ref{algo:main_int_f}, \ref{algo:limbladiano2} y \ref{algo:conmuH}.}\label{tabla:proporciones_tiempo_F_lim_conmu}
\end{table}

\subsubsection{Resultados paralelizaci\'on}
\qquad Para el c\'omputo en paralelo, se utilizaron los resultados expuestos en la secci\'on anterior. Se observa de aquellos resultados ha de ser conveniente calcular el Limbladiano (Li en \label{tabla:proporciones_tiempo_F_lim_conmu}) y el Conmutador del Hamiltoniano (CH en \label{tabla:proporciones_tiempo_F_lim_conmu}) en paralelo, \'esta paralelizaci\'on ser\'a denotada por P1. Adem\'as, se observa que las rutinas que m\'as costo computacional tienen, de las expuestas en (\ref{sec:computo_eficiente_operadores}), son las asociadas a operadores del campo electromagn\'etico (\ref{sec:expre_modales_orden2}), por ende, una segunda paralelizaci\'on es el c\'alculo en paralelo del bucl\'es, o \emph{loops} que aparecen en \'estos algoritmos (por ejemplo, Algoritmo \ref{algo:aratMod} l\'inea \ref{algo:aratMod_bucle0}, \'esta segunda paralelizaci\'on ser\'a denotada por P2. Finalmente, una tercera paralelizaci\'on, P3, estar\'a dada por la combinaci\'on de ambas paralelizaciones, mientras que la rutina original (pero con el n\'umero de correcciones fijadas a tres) ser\'a denotada por P0.\\

\qquad Para medir el rendimiento (\emph{performance}) de las paralelizaciones, expondremos los tiempos totales de demora de cada versi\'on (para diferentes dimensiones de los espacios modales), el factor de mejora $\alpha$ y la eficiencia lograda. Para \'esto \'ultimo; si $T_s$ es el tiempo de demora del programa \emph{serial}, esto es, el programa original; $T_p$ es el tiempo de demora \emph{real} del programa paralelizado, esto es, el tiempo que el usuario mide; $T_t$ es el tiempo total de demora, esto es, la suma de los tiempos en cada hilo de procesamiento. Entonces, $\alpha=T_s/T_p$ y la eficiencia $E_A$ de un algoritmo $A$ es calculada seg\'un:
$$E_A=100\cdot\frac{T_s}{T_t}\%.$$
\qquad Notemos que si $n$ es el n\'umero de hilos de procesamiento, entonces para obtener una eficiencia del 100$\%$ es necesario que cada hilo tenga un tiempo de procesamiento igual a $T_s/n$ y que ser\'a equivalente en dicho caso a $T_p$. En la Tabla \ref{tabla:rendimientos_paralelizaciones} se muestran tanto los factores de mejora de los tiempos, como las eficiencias logradas, mientras que en la Tabla \ref{tabla:tiempos_paralelizaciones} se muestran los tiempos respectivos. En todos \'estos casos, el n\'umero de correciones del m\'etodo de integraci\'on fue fijado a tres, \'esto para tener una mejor medida del producto de las paralelizaciones en s\'i, adem\'as el par\'ametro relacionado con el ruido t\'ermico fue fijado en $\bar{n}=0\ldotp 001$.\\

\begin{table}[h]
 \centering
\begin{tabular}{|c|c|c|c|c|c|c|c|c|c|c|c|c|}
\hline $\dimA=\dimB$&\multicolumn{2}{|c|}{2}&\multicolumn{2}{|c|}{4}&\multicolumn{2}{|c|}{6}&\multicolumn{2}{|c|}{8}&\multicolumn{2}{|c|}{10}&\multicolumn{2}{|c|}{12}\\
\hline Rutina&$\alpha$&$E_A\%$&$\alpha$&$E_A\%$&$\alpha$&$E_A\%$&$\alpha$&$E_A\%$&$\alpha$&$E_A\%$&$\alpha$&$E_A\%$\\
\hline P1 & 1.16 &59 &1.11 &57 &1.04 & 54&1.29 &68 &1.30 &{\color{red}69} &1.27 &{\color{red}69} \\
\hline P2 & {\color{red}0.98} &50 &1.25 &63 &1.25 & 64&1.21 &64 &1.2 &67 &1.23 &{\color{red}69} \\
\hline P3 & 1.32 &45 &1.68 &57 &1.63 & 57&{\color{red}1.85} &64 &1.84 &65 &1.73 &61 \\
\hline
\end{tabular}\caption{Rendimiento de las paralelizaciones realizadas.}\label{tabla:rendimientos_paralelizaciones}
\end{table}
\begin{table}[h]
 \centering
\begin{tabular}{|c|c|c|c|c|c|c|c|}
\hline \multicolumn{2}{|c|}{} &\multicolumn{6}{|c|}{$\dimA=\dimB$}\\
\hline Rutina& 2&4&6&8&10&12\\
\hline P0 & 24s& 4m03s& 18m49s& 1h19m& 3h59m& 8h13m\\
\hline P1 & 20s& 3m38s& 18m09s& 1h01m& 3h04m& 6h29m\\
\hline P2 & 24s& 3m14s& 15m03s& 1h05m& 3h16m& 6h41m\\
\hline P3 & 18s& 2m24s& 11m31s& 42m58s& 2h10m& 4h45m\\
\hline
\end{tabular}\caption{Tiempos de demora de rutinas paralelizadas.}\label{tabla:tiempos_paralelizaciones}
\end{table}

\Pag{CONCLUSIONES}
\sectionm{Conclusiones}
\subsection{Sobre los resultados con ruido t\'ermico}
\subsection{Sobre los tiempos de ejecuci\'on obtenidos} \qquad De la Tabla \ref{tabla:tiempos_modos}, es claro que los tiempos de ejecuci\'on del programa principal, Algoritmo \ref{algo:main_completo}, aumentan seg\'un lo hacen tambi\'en las dimensiones de los espacios de Hilbert involucrados $\mathcal{H}_a$ y $\mathcal{H}_b$, \'esto se explica claramente por la dependencia cuadr\'atica con dichas dimensiones, del n\'umero de operaciones de los algoritmos descritos en (\ref{sec:computo_eficiente_operadores}). Adem\'as, se aprecia c\'omo los tiempos de c\'omputo, tambi\'en aumentan a medida que el par\'ametro $\bar{n}$, el ruido t\'ermico, aumenta. \'Esto se explica dado que la variaci\'on de $\rho$ es mayor para mayores niveles de ruido (en la escala temporal considerada), por ende el m\'etodo corrector deber\'a realizar un mayor n\'umero de correciones, como lo confirman los datos expuestos en la Tabla \ref{tabla:numero_correcciones}. Tambi\'en, al comparar las tablas \ref{tabla:tiempos_gino} y \ref{tabla:tiempos_actual_sinruido} se aprecia claramente una mejora en los tiempos de ejecuci\'on, con respecto al trabajo anterior.\\

\qquad Tambi\'en podemos observar que el algoritmo de mayor coste computacional, est\'a asociada al c\'alculo de la funci\'on a F utilizada para el c\'alculo de la derivad temporal de la matriz de densidad, \'esto se aprecia claramente en la Tabla \ref{tabla:tiempos_prop_por_rutina}. Adem\'as, mediante los datos obtenidos en la Tabla \ref{tabla:proporciones_tiempo_F_lim_conmu}, podemos confirmar que el algoritmo con mayor coste computacional son los asociados a los modos del campo electromagn\'etico acoplados al \'atomo, aquello provoca que el Limbladiano tenga un mayor coste computacional frente al conmutador del Hamiltoniano, puesto que el primer operador est\'a compuesto por expresiones de segundo orden, mientras que el \'ultimo por expresiones de primer orden. \'Esta informaci\'on la utilizamos para establecer qu\'e rutinas deb\'ian tener prioridad en los programas escritos con directivas de paralelizaci\'on (OpenMP). Lo anterior nos lleva a establecer que de realizarse un aumento del n\'umero de niveles at\'omicos, el coste computacional no aumentar\'a en exceso, en comparaci\'on con el modelo expuesto en este trabajo, puesto que en dicho caso, habr\'a un aumento en el uso de operadores at\'omicos, como los expuestos en (\ref{sec:expre_atomicas_orden2}), dichos operadores mostraron tener un menor coste computacional que los asociados al los modos acoplados al \'atomo.\\

\qquad Las variaciones en las eficiencias obtenidas para distintas dimensiones modales, pueden ser explicadas debido al aumento de la memoria utilizada por el programa, \'esto tiene una \'intima relaci\'on con el rendimiento puesto que var\'ia la proporci\'on de instrucciones y datos del programa que pueden ser alojados en las memorias de mejor rendimiento o c\'aches, v\'ease por ejemplo Tabla \ref{tabla:cantidad_memoria_rho}. As\'i, las velocidades a las cuales un dato o instrucci\'on son accesados por el hilo de procesamiento pueden variar dram\'aticamente, para profundizaci\'on el lector puede revisar \cite{arquitectura_pcs}.\\

\qquad Adem\'as, en un trabajo futuro, la adaptaci\'on de los programas estar\'a facilitada, por la caracter\'istica de realizar los c\'alculos, utilizandos las diferentes expresiones expuestas en (\ref{sec:computo_eficiente_operadores}) como bloques de construcci\'on. Adem\'as, los par\'ametros del problema, como dimensiones, acoplamientos, constantes, han sido ingresados de manera din\'amica, esto es, son inclu\'idas en un m\'odulo aparte (llamado en particular ``datos.f90''), de manera tal que la modificaci\'on de la Ecuaci\'on Maestra, al incluir posibles nuevos niveles y acoplamientos sea posible de forma directa, afectando s\'olo los arreglos que contienen coeficientes asociados a los t\'erminos y las variables asociadas a dimensiones, como n\'umero de modos, n\'umero de fotones por modo, etc.
\section*{Nomenclatura y constantes}
\fancyhf{}
\fancyfoot[C]{\titlem\hspace*{5cm}\thepage}
% %\fancyfoot[L]{\titlem}
% %\fancyfoot[R]{\thepage} 
\fancyhead[R]{Nomenclatura y constantes}
\addcontentsline{toc}{section}{\bfseries Nomenclatura y constantes}
\begin{tabular*}{\textwidth}{ll}
\hline\multirow{2}{*}{\large{S�mbolo}}&\multirow{2}{*}{\large{Descripci�n}}\\
&\\
\hline\\
$\hbar$ & Constante de Planck.\\\\
$\omega_a$ & Frecuencia de resonancia del modo $a$ del campo electromagn\'etico. An\'alogo para $\omega_b$.\\\\
$\omega_i$ & Frecuencia de resonancia del nivel at\'omico $i$.\\\\
$\Omega_i(t)$ & Intensidad del l\'aser $i$ en el tiempo $t$.\\\\
$g_a$ & Factor de acoplamiento del modo $a$ a los niveles at\'omicos asociados. An\'alogo para $g_b$.\\\\
$\kappa_a$ & Factor de p\'erdida de energ\'ia en la cavidad asociado al modo $a$. An\'alogo para $\kappa_b$.\\\\
$\triangledown^2$ & Operador Laplaciano.\\\\
$[a,b]$ & Conmutador de $a$ y $b$: $ab-ba$\\\\
$\otimes$ & Producto tensorial de espacios vectoriales.\\\\
$\delta_{i,j}$ & Delta de Kronecker.\\\\
$a^\dag$ & Operador adjunto de $a$.\\\\
$tr\left(a\right)$ & Traza de la matriz representante del operador $\hat{a}$.\\\\
$nEm$ & Notaci\'on exponencial: $nEm=n\cdot 10^m,\;n,\,m\in \mathbb{R}$.\\\\
$\lceil x\rceil$ & Parte entera superior de $x\in \mathbb{R}$.\\\\
$n\bmod p$ & M\'odulo de congruencia de $n,\,p\in\mathbb{Z}$.
\end{tabular*} 

\begin{tabular*}{\textwidth}{ll}
\hline\multirow{2}{*}{\large{S�mbolo}}&\multirow{2}{*}{\large{Descripci�n}}\\
&\\
\hline\\
$e^t$ & Vector transpuesto de $e$.\\\\
$I^n$ & Matriz identidad de orden $n$.\\\\
$\mathbb{O}^n$ & Matriz nula de orden $n$.
\end{tabular*} 
\section*{Bibliograf�a}
\fancyhf{}
\fancyfoot[C]{\titlem\hspace*{5cm}\thepage}
%\fancyfoot[R]{\thepage} 
\fancyhead[R]{Bibliograf�a}
\addcontentsline{toc}{section}{\bfseries Bibliograf�a}
\begin{thebibliography}{2}

\bibitem{griffiths} David J. Griffiths. {\it Introduction to Quantum Mechanics}. Pearson Education. Second Edition, 2005.
\bibitem{eisberg} Robert Eisberg, Robert Resnick. {\it F�sica Cu�ntica}. Editorial Limusa. 2009.
\bibitem{cohen} Claude Cohen-Tannoudji, Bernard Diu, Frank Laloe. {\it Quantum Mechanics, Volume 1}. WILEY-VCH. 1991. 
\bibitem{dutra} Sergio M. Dutra. {\it  Cavity Quantum Electrodynamics, The Strange Theory of Light in a Box}. WILEY-INTERSCIENCE. 2005.
\bibitem{gino} Gino Montecinos. {\it C\'alculo de Funciones de Correlaci�n en la generaci�n de dos fotones por encargo. Trabajo para optar al T�tulo de Ingeniero Matem�tico}. UFRO. 2008.
\bibitem{robert} R. Guzm\'an. {\it Procesamiento de Informaci\'on en Sistemas Cu\'anticos}. Tesis Doctoral. USACH, 2004.
\bibitem{carmichael} Howard Carmichael. {\it An Open Systems Approach to Quantum Optics, Lectures presented at the Universit� Libre de Bruxelles, October 28 to November 4, 1991}. Springer-Verlag. 1993.
\bibitem{cohenphoton} Claude Cohen-Tannoudji, Jacques Dupont-Roc, Gilbert Grynberg. {\it  Photons \& Atoms, Introduction to Quantum Electrodynamics}. WILEY-VCH. 2004.
\bibitem{single-photon} Christian Maurer, Christoph Becker, Carlos Russo, J\"urgen Eschner and Rainer Blatt. {\it A single-photon source based on a single $Ca^+$ ion}. New Journal of Physics. 6(2004)94.
\bibitem{single-photon-walther} M. Keller, B. Lange, K. Hayasaka, W. Lange and H. Walther. {\it A calcium ion in a cavity as a controlled single-photon source}. New Journal of Physics. 6(2004)95.
\bibitem{electro} David J. Griffiths. {\it Introduction to Electrodynamics}. Prentice Hall. 3rd Edition. 1999.
\bibitem{martinelli} Marcelo Martinelli. {\it Luz e �tomos como ferramentas para Informa��o Qu�ntica}. Instituto de F\'isica, Laborat�rio de Manipula��o Coerente de �tomos e Luz. Universidade do S�o Paulo. Curso de Ver�o 2011.
\bibitem{openmp}Barbara Chapman, Gabriele Jost and Ruud Van Der Pas. {\it Using OpenMP, Portable Shared Memory Parallel Programming}. The MIT Press. 2008.
\bibitem{arquitectura_pcs} David A. Patterson, John L. Hennessy, Ram\'on Canal Corretger. {\it Estructura y dise�o de computadores, Vol 2.}. Editorial Revert\'e, 2000.
\bibitem{maravillas} George Gamow. {\it En el pa�s de las maravillas: relatividad y cuantos}. Fondo de Cultura Econ�mica. 1958.
%\bibitem{1} Autores. {\textit{T�tulo del Libro}}, Editorial, (A�o).
%\bibitem{2} Autores. {\textit{T�tulo del Articulo}}, Revista, N�mero(a�o)paginas.
\end{thebibliography}
\section*{Anexo}
\fancyhf{}
\fancyfoot[L]{\titlem}
\fancyfoot[R]{\thepage} 
\fancyhead[R]{Anexo}
\addcontentsline{toc}{section}{\bfseries Anexo}
\setcounter{section}{1}
\setcounter{subsection}{0}
\renewcommand{\thesection}{\Alph{section}}
A continuaci\'on hacemos una breve descripci\'on del software utilizado durante la realizaci\'on del presente trabajo.

\subsection{IFORT}\qquad Compilador de Intel para el lenguaje de programaci\'on Fortran, utilizado en este trabajo. Es un compilador de alto rendimiento pensado en especial para procesadores Intel, mayor informaci\'on puede ser obtenida en:
\begin{center}
 \url{http://software.intel.com/en-us/articles/intel-composer-xe/}.
\end{center}

\qquad En pruebas realizadas, super\'o siempre al popular compilador GNU \verb|gfortran|. Adem\'as se utiliz\'o la licencia de ``Software No Comercial'' (Non-Comercial Software), se puede encontrar informaci\'on acerca de este tipo de Software de Intel en:
\begin{center}
\url{http://software.intel.com/en-us/articles/non-commercial-software-download/}.\\
\end{center}

\subsection{OpenMP}\qquad Es una ``Interfaz de programaci\'on de aplicaciones'' o API, por sus siglas en ingl\'es. Permite la creaci\'on de programas que realicen c\'omputos en paralelo en arquitecturas de memoria compartida (v\'ease \cite{arquitectura_pcs}). La idea de su creaci\'on fue facilitar la creaci\'on de Software de c\'omputo en paralelo de manera sencilla y portable. OpenMP no es un lenguaje de programaci\'on, sino, consiste en una serie de directivas que son a\~nadidas a los c\'odigos fuentes de un programa secuencial, escrito en C/C++ o Fortran, de manera tal que describan c\'omo el trabajo es debe ser compartido entre los diferentes hilos de procesamiento que ser\'an ejecutados en diferentes procesadores o n\'ucleos y para ordenar el acceso a la memoria compartida por \'estos hilos. Mayor informaci\'on puede encontrada en \cite{openmp} o en:
\begin{center}
\url{http://www.openmp.org}\\
\end{center}

\subsection{Scilab}\qquad Su nombre resume su versatilidad: en espa\~nol, ``Laboratorio Cient\'ifico''. Es un software libre de c\'odigo abierto (Free Open Source Software) para Computaci\'on Num\'erica. Provee un ambiente de c\'omputo para aplicaciones en ingenier\'ia y ciencia. Es un lenguaje de programaci\'on de alto nivel, con cientas de funciones matem\'aticas, permite por ejemplo, el acceso a estructuras de datos complejas, generaci\'on de gr\'aficos en 2-D y 3-D, entre otras capacidades. \verb|Scilab| es distribuido bajo licencia CeCILL, compatible con GPL. En el presente trabajo se hizo uso de la versi\'on 5.3.3 por su estabilidad, f\'acil acceso a la ayuda y mejorada interface. Puede encontrarse mayor informaci\'on y realizar la descarga de esta poderosa herramienta en:
\begin{center}
\url{http://www.scilab.org}\\
\end{center}

\subsection{Dia} \qquad Programa de desarrollo de diagramas, basado en gtk+. Fue utilizado para la realizaci\'on de las figuras \ref{fig:decaimientos}  y \ref{fig:laseres}. Es distribuido bajo licencia GNU. Puede descargarse, y encontrarse documentaci\'on acerca de este software en:
\begin{center}
\url{http://projects.gnome.org/dia/}\\
\end{center}

\subsection{Kile}\qquad Excelente entorno de desarrollo de \LaTeX, integrado en el entorno de escritorio KDE. Es distribuido bajo licencia GNU. Puede descargarse, y encontrarse documentaci\'on acerca de este software en:
\begin{center}
\url{http://kile.sourceforge.net/}\\
\end{center}
%\include{cap_anexos/anexob}
\end{document}