\Pag{CONCLUSIONES}
\sectionm{Conclusiones}
\subsection{Sobre los resultados con ruido t\'ermico}
\subsection{Sobre los tiempos de ejecuci\'on obtenidos} \qquad De la Tabla \ref{tabla:tiempos_modos}, es claro que los tiempos de ejecuci\'on del programa principal, Algoritmo \ref{algo:main_completo}, aumentan seg\'un lo hacen tambi\'en las dimensiones de los espacios de Hilbert involucrados $\mathcal{H}_a$ y $\mathcal{H}_b$, \'esto se explica claramente por la dependencia cuadr\'atica con dichas dimensiones, del n\'umero de operaciones de los algoritmos descritos en (\ref{sec:computo_eficiente_operadores}). Adem\'as, se aprecia c\'omo los tiempos de c\'omputo, tambi\'en aumentan a medida que el par\'ametro $\bar{n}$, el ruido t\'ermico, aumenta. \'Esto se explica dado que la variaci\'on de $\rho$ es mayor para mayores niveles de ruido (en la escala temporal considerada), por ende el m\'etodo corrector deber\'a realizar un mayor n\'umero de correciones, como lo confirman los datos expuestos en la Tabla \ref{tabla:numero_correcciones}. Tambi\'en, al comparar las tablas \ref{tabla:tiempos_gino} y \ref{tabla:tiempos_actual_sinruido} se aprecia claramente una mejora en los tiempos de ejecuci\'on, con respecto al trabajo anterior.\\

\qquad Tambi\'en podemos observar que el algoritmo de mayor coste computacional, est\'a asociada al c\'alculo de la funci\'on a F utilizada para el c\'alculo de la derivad temporal de la matriz de densidad, \'esto se aprecia claramente en la Tabla \ref{tabla:tiempos_prop_por_rutina}. Adem\'as, mediante los datos obtenidos en la Tabla \ref{tabla:proporciones_tiempo_F_lim_conmu}, podemos confirmar que el algoritmo con mayor coste computacional son los asociados a los modos del campo electromagn\'etico acoplados al \'atomo, aquello provoca que el Limbladiano tenga un mayor coste computacional frente al conmutador del Hamiltoniano, puesto que el primer operador est\'a compuesto por expresiones de segundo orden, mientras que el \'ultimo por expresiones de primer orden. \'Esta informaci\'on la utilizamos para establecer qu\'e rutinas deb\'ian tener prioridad en los programas escritos con directivas de paralelizaci\'on (OpenMP). Lo anterior nos lleva a establecer que de realizarse un aumento del n\'umero de niveles at\'omicos, el coste computacional no aumentar\'a en exceso, en comparaci\'on con el modelo expuesto en este trabajo, puesto que en dicho caso, habr\'a un aumento en el uso de operadores at\'omicos, como los expuestos en (\ref{sec:expre_atomicas_orden2}), dichos operadores mostraron tener un menor coste computacional que los asociados al los modos acoplados al \'atomo.\\

\qquad Las variaciones en las eficiencias obtenidas para distintas dimensiones modales, pueden ser explicadas debido al aumento de la memoria utilizada por el programa, \'esto tiene una \'intima relaci\'on con el rendimiento puesto que var\'ia la proporci\'on de instrucciones y datos del programa que pueden ser alojados en las memorias de mejor rendimiento o c\'aches, v\'ease por ejemplo Tabla \ref{tabla:cantidad_memoria_rho}. As\'i, las velocidades a las cuales un dato o instrucci\'on son accesados por el hilo de procesamiento pueden variar dram\'aticamente, para profundizaci\'on el lector puede revisar \cite{arquitectura_pcs}.\\

\qquad Adem\'as, en un trabajo futuro, la adaptaci\'on de los programas estar\'a facilitada, por la caracter\'istica de realizar los c\'alculos, utilizandos las diferentes expresiones expuestas en (\ref{sec:computo_eficiente_operadores}) como bloques de construcci\'on. Adem\'as, los par\'ametros del problema, como dimensiones, acoplamientos, constantes, han sido ingresados de manera din\'amica, esto es, son inclu\'idas en un m\'odulo aparte (llamado en particular ``datos.f90''), de manera tal que la modificaci\'on de la Ecuaci\'on Maestra, al incluir posibles nuevos niveles y acoplamientos sea posible de forma directa, afectando s\'olo los arreglos que contienen coeficientes asociados a los t\'erminos y las variables asociadas a dimensiones, como n\'umero de modos, n\'umero de fotones por modo, etc.