\begin{center}
\section*{RESUMEN}
\end{center}
\quad En el presente Trabajo de T\'itulo se mejoraron los programas realizados en \cite{gino}, dichos programas solucionan la Ecuaci\'on de Sch\"odinger para el \'atomo de Calcio atrapado en una cavidad, con ciertas condiciones (condici\'on inicial, acoplamiento a modos del campo electromagn\'etico, acomplamientos a l\'aseres) de manera tal que se logran emitir, de manera determin\'istica dos fotones en ciertos modos. La mejora se obtuvo en tanto se redujeron dr\'asticamente los tiempos de c\'omputo y se obtuvieron programas capaces de resolver problemas m\'as gen\'ericos, en particular con ruido t\'ermico en la cavidad. A continuaci\'on resumimos los contenidos de los cap\'itulos de este documento.\\

\begin{description}
 \item [\emph{Ca\'pitulo 1:} ] Se hace una introducci\'on al problema resuelto desde el punto de vista f\'isico y se describen los objetivos del presente Trabajo de T\'itulo.
\item [\emph{Ca\'pitulo 2:} ] Se hace una breve introducci\'on, desde el punto de vista de la F\'isica, de los temas relevantes involucrados en el presente trabajo. Se comienza desde la idea de de Broglie sobre las ondas de materia, se solucionan ciertos casos fundamentales de la Ecuaci\'on de Sch\"odinger desde el punto de vista conceptual; como lo es por ejemplo, el caso del oscilador arm\'onico. Luego, utilizando la notaci\'on de Dirac, se presenta la \emph{primera cuantizaci\'on del campo electromagn\'etico}, donde aparecer\'a el fot\'on, concepto fundamental que en la parte final del cap\'itulo, ser\'a ligado el modelo cu\'antico de un \'atomo a trav\'es de la emisi\'on y absorci\'on de materia. \'Esto \'ultimo es de vital importancia, puesto que es el fundamento mediante el cual podremos emitir fotones de manera determin\'istica.
\item [\emph{Ca\'pitulo 3:} ] En este cap\'itulo, se presenta el modelo matem\'atico utilizado para el estudio de la evoluci\'on del Calcio II, una vez que es llevado al primer estado excitado. Se definen los espacios de Hilbert que compondr\'an el espacio total $\mathcal{H}$, luego se presentan los operadores de $\mathcal{H}$ en $\mathcal{H}$ involucrados en el modelo utilizado y que tendr\'an importancia fundamental en la mejora de los tiempos de c\'omputo del programa. Tambi\'en se presenta formalmente la adici\'on de ruido t\'ermico en la cavidad, a trav\'es de la adici\'on de ciertos t\'erminos en el Limbladiano, operador que se modela las p\'erdidas de energ\'ia del sistema. Finalmente presentamos el m\'etodo num\'erico, que ser\'a utilizado para integrar la ecuaci\'on maestra (nombre de una forma particular de la Ec. de Sch\"odinger: el m\'etodo de Heun.
\item [\emph{Ca\'pitulo 4:} ] \'Este es el cap\'itulo m\'as extenso, y con raz\'on puesto que contien gran parte del trabajo realizado. Se comienza definiendo variables importantes desde el punto de vista computacional e identificando puntos importantes de los algoritmos generales, utilizados para computar la soluci\'on del problema. Luego, se estudian las estructuras de los operadores introducidos el Cap\'itulo 3, y se escriben los algoritmos de c\'omputo de la acci\'on, de cada operador, sobre un eventual elemento de $\mathcal{H}$. Tambi\'en se estudia la acci\'on de las composiciones de estos operadores, que sean importantes en el modelo. Aqu\'i es donde se obtuvo el primero de los logros, puesto que gracias a estos estudios, el n\'umero de operaciones de procesador necesarias se redujo en un orden, lo que signific\'o una gran ganancia en tiempo de c\'omputo. Finalmente, se escriben los principales algoritmos que componen el programa general, utilizando como bloques de construcci\'on los obtenidos anteriormente, esto se hizo de manera tal de lograr la generalidad necesaria, de manera tal que \'estos algoritmos puedan ser reutilizados en modelos m\'as complejos, inclusive del mismo problema f\'isico.
\item [\emph{Ca\'pitulo 5:} ] Presentamos en este cap\'itulo los resultados, cuantitativos y cualitativos de los algoritmos obtenidos en el Cap\'itulo  5. Tambi\'en se presentan los resultados del programa anterior con el fin de realizar una comparaci\'on de los tiempos de c\'alculo, una vez que se establece que las soluciones obtenidas son las mismas.
\item [\emph{Ca\'pitulo 6:} ] En \'este cap\'itulo se realiza el an\'alisis, de los resultados y de sus consecuencias en tanto logros obtenidos e impacto en trabajos futuros.
\item [\emph{Anexo:} ] Sobre el Software utilizado en el presente trabajo.
\end{description}

\quad Como validaci\'on de los resultados obtenidos, se utilizaron, tanto los resultados del trabajo anterior, conocimiento de la Teor\'ia y resultados de publicaciones estudiadas, como \cite{single-photon} y \cite{single-photon-walther}. Los logros m\'as importantes de este trabajo, se obtuvieron gracias a la identificaci\'on de las distintas estructuras de los operadores involucrados, y por ende de la acci\'on que \'estos realizan sobre alg\'un elemento del espacio de Hilbert $\mathcal{H}$, inclusive no siendo necesario el c\'omputo de \'estos operadores sino, obteniendo directamente el resultado de su operaci\'on y el de sus composiciones.